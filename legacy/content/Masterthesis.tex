\documentclass[11pt]{report}
\usepackage[english]{babel}
\usepackage{amsmath,amsfonts,amsthm,amssymb}
\usepackage{hyperref}
\usepackage{epigraph}
\usepackage{cleveref}
\usepackage{fancyhdr}
\usepackage{graphicx}
\usepackage{setspace}
\usepackage{verbatim}
\usepackage{tikz-cd}
\usepackage{enumitem}
\usepackage{makeidx}
\usepackage{MA_Titlepage}
\usepackage[T1]{fontenc}
\usepackage[a4paper,left=2.5cm,right=1.5cm,top=3cm,bottom=2cm]{geometry}
\pdfpagewidth 12in
\pdfpageheight 15in
\setenumerate{itemsep=0pt,topsep=3pt}
%%%%%%%%%%%%%%%%%%%preamble%%%%%%%%%%%%%%%


%Name of the author of the thesis 
\authornew{Zhicheng HAN}
%Date of birth of the Author
\geburtsdatum{14th August 1992}
%Place of Birth
\geburtsort{Jingdezhen, Jiangxi, People Republic of China}
%Date of submission of the thesis
\date{\today}

%Name of the Advisor
% z.B.: Prof. Dr. Peter Koepke
\betreuer{Advisor: Prof. Dr. Wolfgang L\"uck}
%Name of the Insitute of the advisor
\zweitgutachter{Second Advisor: Prof. Dr. Matthias Lesch}
%name of the second advisor of the thesis
%z.B.: Mathematisches Institut
\institut{Mathematical Institute}
%Title of the thesis 
\title{Analytic Aspects of $L^2$-Invariants}
%Do not change!
\ausarbeitungstyp{Master's Thesis  Mathematics}
\pagestyle{fancy}
\fancyhf{}
\renewcommand{\headrulewidth}{0pt}
\headheight 5pt
\rhead{\rightmark}
\lhead{}
\rfoot{}
\cfoot{\thepage}
\lfoot{}
%\newcounter{nnumber}[section]
%\newcounter{apnnumber}[chapter]

\theoremstyle{definition}

\newtheorem{apDef}{Definition}
\newtheorem{Def}{Definition}[chapter]
\newtheorem{Fact}{Fact}

\newtheorem{Lemma}[Def]{Lemma}
\newtheorem{Cor}[Def]{Corollary}
\newtheorem{Rmk}{Remark}[chapter]
\newtheorem{Exe}[Rmk]{Example}

\theoremstyle{plain}
\newtheorem{Prop}[Def]{Proposition}
\newtheorem{Theo}[Def]{Theorem}
\newtheorem*{Theo*}{Theorem}


%\renewcommand\thesection{\arabic{subsection}}
\DeclareMathOperator*{\Span}{Span}
\DeclareMathOperator{\tr}{tr}
\DeclareMathOperator{\sign}{sgn}
\DeclareMathOperator{\id}{id}
\DeclareMathOperator{\im}{Im}
\DeclareMathOperator{\inv}{Inv}
\DeclareMathOperator{\codim}{codim}
\DeclareMathOperator{\ind}{ind}
\DeclareMathOperator{\defect}{def}
\DeclareMathOperator{\nul}{nul}
\DeclareMathOperator{\Ad}{Ad}
\DeclareMathOperator{\ad}{ad}
\DeclareMathOperator{\Ric}{Ric}
\DeclareMathOperator{\Mab}{Mab}
\DeclareMathOperator{\osc}{osc}
\DeclareMathOperator{\grad}{grad}
\DeclareMathOperator{\dom}{Dom}
\DeclareMathOperator{\spec}{spec}
\DeclareMathOperator{\vol}{vol}
\DeclareMathOperator{\cone}{cone}
\DeclareMathOperator{\colim}{colim}
\DeclareMathOperator{\cyl}{cyl}
\DeclareMathOperator{\Wh}{Wh}
\DeclareMathOperator{\rk}{rank}
\DeclareMathOperator{\Ext}{Ext}
\DeclareMathOperator{\Hom}{Hom}
%%%%%%%%%%%%%%%%%%%%%%%%%%%%%%%%%%%%\right %%%%%%%%%%%%%%%%%%%%%%%%
\newcommand{\Defautorefname}{Definition}
\newcommand{\Lemmaautorefname}{Lemma}
\newcommand{\Corautorefname}{Corollary}
\newcommand{\Theoautorefname}{Theorem}
\newcommand{\Rmkautorefname}{Remark}
\newcommand{\Propautorefname}{Proposition}
\newcommand{\Exeautorefname}{Example}
\newcommand{\Factautorefname}{Fact}
\renewcommand{\equationautorefname}{Equation}
\newcommand*{\bigchi}{\mbox{\Large$\chi$}}% big chi
%%%%%%%%%%%%%%%%%%%%%%%
\newcommand{\nat}{\mathbb{N}}
\newcommand{\real}{\mathbb{R}}
\newcommand{\complex}{\mathbb{C}}
\newcommand{\rational}{\mathbb{Q}}
\newcommand{\integer}{\mathbb{Z}}
\newcommand{\field}{\mathbb{F}}
\newcommand{\inj}{\hookrightarrow}
\newcommand{\bdd}{\mathcal{B}}
\newcommand{\vna}{\mathcal{N}}
\newcommand{\even}{\mathrm{even}}
\newcommand{\odd}{\mathrm{odd}}
%%%%%%%%%%%%%for this document%%%%%%%%%%%%%%%%%%%%%%
\newcommand{\overbar}[1]{\mkern 1.5mu\overline{\mkern-1.5mu#1\mkern-1.5mu}\mkern 1.5mu}
\newcommand{\brac}[1]{\langle #1 \rangle}
\renewcommand{\bar}{\overbar}
\renewcommand{\hat}{\widehat}
\newcommand{\disk}{\mathbb{D}}
\newcommand{\torus}{\mathbb{T}}
\newcommand{\diff}[2]{\frac{\partial #1}{\partial #2}}
\newcommand{\norm}[1]{\lVert #1 \rVert}
\renewcommand{\tilde}{\widetilde}
\newcommand{\mass}[1]{\mathop{}\mathrm{d}{#1}}
\renewcommand{\frak}{\Lie}
\newcommand{\Lie}[1]{\mathfrak{#1}}
\renewcommand{\Re}{\mathrm{Re}}

\hypersetup{%
	colorlinks=true,% hyperlinks will be black
	linkcolor=cyan,
	citecolor=magenta, % hyperlink borders will be red
	pdfborderstyle={/S/U/W 1}% border style will be underline of width 1pt
}

\bibliographystyle{alpha}

\makeatletter
\def\th@plain{%
	\thm@notefont{}% same as heading font
	\itshape % body font
}
\def\th@definition{%
	\thm@notefont{}% same as heading font
	\normalfont % body font
}
\makeatother
\setlength\epigraphwidth{8cm}
\setlength\epigraphrule{0pt}
\begin{document}
	\setlength{\parskip}{.3em}
	\maketitle
	\tableofcontents
	\chapter*{Abstract}
	In this article we make a detailed study of different $L^2$-invariants mainly from the perspective of analysis. 
	\par In \Cref{chapter1} we introduce the readers some basic concepts pertaining to our following discussion of $L^2$-invariants. In particular, this includes an quick survey of Borel functional calculus, Hodge theory and equivariant CW-complexes, together with an heuristic approach to Whitehead torsion. Readers who are familiar with these terms can skip this section and refer back in times of need.
	\par In \Cref{chapter2} we define all the $L^2$-invariants that are central to our discussions. All of them have been approached from both the topological and analytic angles, with some essential properties discussed along the way. In the ending sessions of this chapter we shall see the both approaches give us the same invariant.
	\par In \Cref{chapter3} we study the $L^2$-invariants of symmetric spaces. To do so, one needs a systematic inspection of heat kernel of the underlying manifolds. Two pillars of this machinery are Harish-Chandra's Plancherel formula, which allows us to express the heat kernel in forms of global characters of irreducible unitary representations, and continuous $(\Lie{g}, K)$-cohomology, which allows us to investigate dimension of each irreducible unitary representations. In such case we have many vanishing results, which help us to further reduce the amount of work. In the last section of \Cref{chapter3} we bring the analytic data captured by Plancherel formula and the representation data captured by continuous cohomology together to prove the theorem.
	\par To stay on the main focus of this article is on $L^2$-invariants, we skip most discussion on theory of von Neumann algebra and $C^*$-algebra. Well-known results, such as Gelfand-Naimark theorem and Double Commutant theorem are often cited rather than properly stated. Readers who are interested in such can consult classical textbooks on such realms such as \cite{conway2013}, and also \cite{murphy2014} and \cite{dixmier1982} for a more algebraic approach.
	
	\section*{Convention of Notations}
	Throughout this article we denote Hilbert spaces over complex number $\complex$ as $H$, with inner product of the Hilbert space $\brac{-,-}$. $\bdd(H)$ is used to denote the space of all the bounded linear operators on $H$ and $\mathcal{L}(H)$ the space of all linear operators on $H$. 
	\par There are also cases we need to compare the asymptotic behaviour of two functions. If for two functions on $\real$, with $g\geq 0$, we uses the Big $O$-notation:
	\begin{equation*}
	f=\mathcal{O}(g) \text{ as }x\to \infty
	\end{equation*}
	if there exists a constant $C>0$ such that $|f(x)|\leq C\cdot g(x)$ for all $x$ sufficient large. 
	\par We also fix the notation $C_*^{(2)}$ to be the $\ell^2(G)\otimes C_*$ when $C_*$ is a chain complex of $\complex G$-modules. The context will specify which the specific group is.
	\par Throughout this article we cross-referred the theorems, propositions, etc. with both term and numbering. Formulae are referred solely with numbers.
	
	
\chapter*{Acknowledgement}
\epigraph{Wir m\"ussen wissen, wir werden wissen.}{David Hilbert}
\par I am indebted to my thesis adviser Herr. Wolfgang L\"uck for ushering him into the realm of $L^2$-invariants. His guidance and moral support are invaluable to me, and his amicability and the liberal atmosphere of his research group are will be part of my fondest memories in Bonn.
\par I would also like to thank Herr. Werner M\"uller for explaining to me, with much patience, Harish-Chandra's Plancherel formula and the harmonic analysis on symmetric spaces. I would also like to thank him for ushering me the heat kernel method on manifolds, which is the starting point of this thesis.
\par Last but not least, I would like to express my deepest gratitude to all of my math friends in Bonn, amongst whom my particular thanks to F. Henneke, S.X. Huang, M. Lackmann, M. Pieper, Y.Q Shi, Q.D. Wang, Y.X. Wang, Y.Z. Wu and A.F. Zhou. These two year-and-a half master studies would not be such a bliss without them.


\chapter{Preliminaries} \label{chapter1}
In this chapter we will briefly introduce all pertinent tools that are vital in our later discussion of $L^2$-invariants. 


\section{Survey on Borel Functional Calculus}
\par In this section we briefly survey Borel functional calculus, one of the most important tools in ensuing discussions. In general terms functional calculus is used to study the following question: given a linear operator $T$ on Hilbert space $H$, how does $f(T)$  behave on $H$ for a function $f:\complex \to \complex$. It turns out when the spectrum of $T$ is well-behaved enough, we can define $f(T)$ for a rather large class of functions. For our purpose we only consider (possibly unbounded) normal operators.

\begin{Def}
	A linear operator $T$ on $H$ is \textbf{normal} if $T$ is closed, densely defined linear operator on $H$ and $H^*H=HH^*$\footnote{Note the definition inexplicitly says $\dom N^*N=\dom NN^*$, but it is not necessary that $\dom N^*N=\dom N$.}
\end{Def}
\begin{Def}\label{spectral measure}
	If $X$ is a set, $\Omega$ is a $\sigma$-algebra of subsets of $X$ and $H$ is a Hilbert space. Then a \textbf{spectral measure} for $(X, \Omega, H)$ is a function $E:\Omega\to \bdd(H)$ satisfying:
	\begin{enumerate}
		\item For each $S\in \Omega$, $E(S)$ is a projection;
		\item $E(\emptyset)=0$ and $E(X)=\id_H$;
		\item $E(S_1\cap S_2)=E(S_1)E(S_2)$ for $S_1, S_2\in \Omega$;
		\item If $\{S_n\}_{n=1}^\infty$ are pairwise disjoint sets from $\Omega$, then $E(\sqcup_{n=1}^\infty S_n)=\sum_{n=1}^{\infty}E(S_n)$.
	\end{enumerate}
\end{Def}
Note for each $h,k\in H$ fixed, we can defined a complex-valued measure $E_{h,k}$ on $X$ by letting:
\begin{equation}
\text{For each $S\in \Omega$} \qquad E_{h,k}(S):=\brac{E(S)h,k}
\end{equation} 
Consequently for any $\Omega$-measurable function $\phi:X\to \complex$, we can define a normal operator $T_\phi$ on $H$ as follows:
\begin{align*}
\dom(T_\phi)=\{h\in H\mid \int|\phi|^2\mass{E_{h,h}}<\infty\}
\end{align*}
Then the operator $\int\phi \mass{E}$ is uniquely determined by the property that for all $h\in \dom(T_\phi)$ and $f\in H$,
\begin{equation}\label{1.65}
\brac{(\int\phi \mass{E})h, f}=\int\phi \mass{E_{h,f}}
\end{equation}
In particular $E^T_\lambda$ can be retrieved as $\chi_{(-\infty, \lambda]}(T)$, where $\chi_S$ is the characteristic function of $S$.
\par The following spectral theorem now asserts that each normal operator determines a unique spectral measure on $\complex$:
\begin{Theo}
	[Spectral Theorem]\textnormal{\cite[Chapter~X, Theorem~4.11]{conway2013}}\label{normalspec} If $T$ is a normal operator on $H$, then there is a unique spectral measure $E$ defined on the Borel subsets of $\complex$ satisfying:
	\begin{enumerate}
		\item  $T=\int z \mass{E^T_z}$;
		\item E(S)=0 if $S$ is a Borel subset of $\complex$ and $S\cap \spec(T)=\emptyset$;
		\item If $U$ is open subset of $\complex$ and $U\cap \spec(T)\neq \emptyset$, then $E(S)\neq 0$;
		\item If $A\in \bdd(H)$ commutes with $N$ and $N^*$, then $A(\int\phi \mass{E})\subset (\int\phi \mass{E})A$ for every Borel function $\phi$ on $\complex$.
	\end{enumerate}
\end{Theo}
Using spectral theorem we can derive many useful terms related to $f$. Define $f\in \mathcal{L}(H)$ to be \textbf{positive} if $f$ is self-adjoint and $\brac{f(v),v}\in \real_+\cup \{0\}$ for all $v\in \dom(T)$.In such case we may define a spectral measure on $\real_+\cup \{0\}$. In particular, we could define the ``square root'' of such positive operators as the following self-adjoint operator:
\begin{equation}
f^{1/2}:=\int_0^\infty \lambda^{1/2} \mass{E^f_\lambda}
\end{equation} 
where $E^f_\lambda$ is the unique spectral measure associated to $f$. It is easy to see $f=f^{1/2}\circ f^{1/2}$.  For an arbitrary operator $f$, it is easy to check $f^*f$ is a positive operator, and we define the ``absolute value'' of $f$ to be:
\begin{equation*}
|f|:=(f^*f)^{1/2}=\int_{0}^{\infty}\lambda^{1/2} \mass{E^{f^*f}_\lambda}
\end{equation*}
As one might expect, there is also an analogue of polar decomposition for unbounded operators:
\begin{Theo}[Polar decomposition]\label{polar decomposition}
	If $T:\dom(T)\subset H_1\to H_2$ is a closed, densely defined unbounded operator between Hilbert space, then it has a (unique) polar decomposition
\begin{equation}
T=U|T|
\end{equation}
	where $U$ is a \textbf{partial isometry}, i.e., it is an isometry on the orthogonal complement of $\ker U$. Moreover, $U$ vanishes on the orthogonal complement of the range $\im(|T|)$.
\end{Theo}
\begin{Def}
	Let $T:\dom(T)\subset H_1\to H_2$ be a linear operator defined on a dense linear subspace $\dom(T)$ of $H_1$.Define an \textbf{closed extension} $S$ of $T$ to be a closed operator $S$ satisfying $\dom(S)\supset \dom(T)$ and $S|_{\dom(T)}=T$. In such case we denote $S\supseteq T$. 
Define  \textbf{minimal closed extension} of $T$ is then a closed extension $T_{\min}$, with domain:
	\begin{equation}
	\dom(T_{\min}):=\{x\in H_1\mid \exists \{x_n\}_{n=1}^\infty\subset \dom(T) \text{ such that}\lim_{n\to \infty}x_n=x \text{ and} \lim_{n\to \infty}Tx_n \text{ exists}\}
	\end{equation}
	and we define $T_{\min}(x)=\lim_{n\to \infty}Tx_n$.
\end{Def}
The reader is to note that $T_{\min}$ corresponds closed extension minimal with respect to all possible closed extensions, ordered under $\supseteq$. 
\begin{Def}
The \textbf{adjoint} of an linear operator $T$ is the operator $T^*$, whose domain is defined by:
\begin{equation}
\dom(T^*)=\Big\{v\in H_2\mid \exists u\in H_1 \text{ such that } \forall u'\in \dom(T), \brac{u', u}=\brac{T(u'),v}\Big\}
\end{equation}
When $H_1=H_2$, we call $T$ \textbf{symmetric} if $T\subset T^*$, and \textbf{self-adjoint} if $T=T^*$. We call $T$ \textbf{essentially self-adjoint} if $T_{\min}$ is self-adjoint, that is, $T$ admits unique self-adjoint extension.
\end{Def}
As one might expect, for general unbounded operators, we can also define an maximal closed extension that is maximal amongst all closed extensions. Nonetheless, within our scope of discussions, in which cases the operators are often positive and symmetric densely defined, we deduce from the following lemma that they admit unique extensions to a self-adjoint operator. 
\begin{Lemma}\label{langa1}
	\cite[A2, \S 1]{lang2012} Let $A:\dom(A)\subset H\to H$ be a symmetric, closed operator with densely defined domain. If for all $\lambda\in \complex\backslash \real$ we have $\bar{\im (A\pm \lambda\cdot \id)}=H$, then $A$ is essentially self-adjoint.
\end{Lemma}
Hence it makes no difference to distinguish different closed extensions. For practical purposes, we concern ourself mostly with minimal extensions. The only significant cases of non-symmetric operators are $d_p$ and $\delta^p$ on $p$-forms, and for this we appeal to the result by Br\"uning and Lesch:
\begin{Lemma}\label{unique closed extension}
	\cite{lesch1992} Let $M$ be a complete Riemannian manifold without boundary. Then there exists a unique closed extension of $d^p:\Omega^p_c(M)\to L^2\Omega^p(M)$ and of $\delta^p:\Omega^p_c(M)\to L^2\Omega(M)$.
\end{Lemma}
We conclude this section by the following construction of bounded operator on $L^2$-spaces. Define the \textbf{essential range} of a function $\phi:X\to \complex$ for some measure space $(X, \Omega, \mu)$ to be:
\begin{equation}
\mathrm{essran}(\phi):=\{\bar{\phi(S)}\mid S\in \Omega, \mu(X\backslash S)=0 \}
\end{equation}
\begin{Prop}\label{conix.2.6}
	Let $(X, \Omega, \mu)$ be any $\sigma$-finite measure space and $H=L^2(X, \Omega, \mu)$ be the underlying Hilbert space. Now for $\phi\in L^\infty(\mu):= L^\infty(X, \Omega, \mu)$, we define $M_\phi$ on $H$ by $M_\phi f=\phi\cdot f$. Now:
	\begin{equation}
	L^\infty(\mu)\to \mathcal{B}(H) \qquad \phi\mapsto M_\phi
	\end{equation}
	gives an isometric representation of $C^*$-algebra, with $\spec(M_\phi)=\mathrm{essran}(\phi)$.
\end{Prop}
\begin{proof}
	Clearly this map is a $*$-homomorphism, hence it suffices to prove $\norm{\phi}_\infty=\norm{M_\phi}$. Note $\norm{M_\phi}\leq \norm{\phi}_\infty$ is straightforward. For the other direction, Take:
	\begin{equation*}
	S:=\{x\in X\mid |\phi(x)|\geq \norm{\phi}_\infty-\epsilon\}\qquad f:=(\mu(S))^{-1/2}\chi_S
	\end{equation*}
	One readily checks $\norm{M_\phi}\geq \norm{\phi f}_2^2\geq \norm{\phi}_\infty-\epsilon$. Take $\epsilon\to 0$, we have $\norm{M_\phi}\geq \norm{\phi}_\infty$.
	\par Next we prove $\spec(M_\phi)=\mathrm{essran}(\phi)$. First assume $\lambda\notin \mathrm{essran}(\phi)$, hence we can find some $S\in \Omega$ such that $\mu(X\backslash  S)=0$ with $\mathrm{dist}(\lambda, \phi(S))>\delta>0$. Now take $\psi=(\phi-\lambda)^{-1}\in L^\infty(\mu)$ with $M_\psi=(M_\phi-\lambda)^{-1}$ is again bounded, and we have $\lambda\notin \spec(M_\phi)$.
	\par Conversely, if $\lambda\in \mathrm{essran}(\phi)$, choose $\{S_n\}_{n\in \nat}$ a sequence in $\Omega$ with $\mathrm{dist}(\lambda, \phi(S_n))<1/n$. Set $f_n:=\mu(S_n)^{-1/2}\chi_{S_n}\in L^2(\mu)$, then we see $\norm{f_n}_2=1$ and $\norm{(M_\phi-\lambda)f_n}_2\leq \frac{1}{n}$, hence $\lambda\in \spec(M_\phi)$ by approaching $n\to \infty$.
\end{proof}
\section{Hilbert $\vna(G)$-Module, Trace and von Neumann Dimension}
In this section we introduced von Neumann dimension of Hilbert $\vna(G)$-module. Note that in \cite[Chapter~6.1]{luck2013} there is an extended dimension function, defined over all arbitrary $\vna(G)$-module, which only depends only on the ring structure of $\vna(G)$. For the sake of simplicity we will not discuss it here.
\par Let $G$ be a discrete group. Define $\ell^2(G)$ to be the completion of $\complex G$ under the pre-Hilbert norm:
\begin{equation}
\brac{\sum_{g\in G}\lambda_gg, \sum_{g\in G}\mu_gg}:=\sum_{g\in G}\lambda_g\bar{\mu_g}
\end{equation}
\begin{Rmk}\label{ell2fct}
Given an element in $\complex G$ one can always identify it with a function $f:G\to \complex$ with compact support via:
	\begin{equation}
	 	\complex G\longrightarrow C_c(G)\qquad 
	 	\sum_{g\in G}\lambda_g\cdot g\longmapsto \Big(f: h\mapsto \lambda_{h}\Big)
	 	\end{equation}
with the inverse map being $f\mapsto \sum_{g\in G}f(g)g$. Note this map is $G$-equivariant, i.e.:
\begin{equation*}
	 (g_0\circ\sum_{g\in G}\lambda_g g)(h)=(\sum_{g\in G}\lambda_{g}g_0\circ g)(h)=(\sum_{g\in G}\lambda_{g_0^{-1}g}g)(h)=\lambda_{g_0^{-1}h}=(\sum_{g\in G}\lambda_g g)(g_0^{-1}h)
\end{equation*}
Hence we may also define $\ell^2(G)$ to be:
\begin{equation}
\ell^2(G):=\Big\{f:G\to \complex\mid \sum_{g\in G}|f(g)|^2<\infty\Big\}
\end{equation}
where $G$ acts on $f$ via $(g\circ f)(h)=f(g^{-1}h)$. The second definition also applies to general $\ell^p(G)$-spaces and when $G$ is a Lie group.
\end{Rmk}
\begin{Def}
The \textbf{group von Neumann algebra $\vna(G)$} of a discrete group $G$ is defined as the $C^*$-subalgebra of $\bdd(\ell^2(G))$ that contains all $G$-equivariant bounded operators.
\end{Def}

\begin{Def}
	A \textbf{Hilbert $\vna(G)$-module} is a Hilbert space $V$ together with a isometric linear $G$-action such that there exists an isometric linear $G$-embedding of $V$ into $H\otimes \ell^2(G)$ for some Hilbert space $H$. A \textbf{morphism} of Hilbert $\vna(G)$-module is a bounded $G$-equivariant operator. A Hilbert $\vna(G)$-module is called \textbf{finitely generated} if there is a surjection: 
	\begin{equation*}
	\bigoplus_{i=1}^n \ell^2(G)\twoheadrightarrow V
	\end{equation*}
\end{Def}
\begin{Rmk}\label{1.6}
	Note in the definition of Hilbert $\vna(G)$-module only the isometric $G$-action is intrinsic. We only request the existence of $H$ for such $G$-embedding. Meanwhile, the Hilbert space structure also gives the chain complexes of Hilbert $\vna(G)$-modules special structures, for instance:
	\begin{enumerate}
		\item From \textbf{Inverse Mapping Theorem} (c.f. \cite[Chapter~III, Theorem 12.5]{conway2013}) we see every bounded injective map between Hilbert spaces have inverse map bounded. Hence a map $f:U\to V$ of Hilbert $\vna(G)$-modules is an isomorphism if and only if it is bijective;
		\item If $f:V\to W$ a \textbf{weak isomorphism} between Hilbert $\vna(G)$-modules, i.e., $f$ is injective and has dense image; then the unitary part is by \nameref{polar decomposition} an isometric isomorphism  between $V$ and $W$. 
		\item Recall any closed subspace $U$ of a Hilbert space $V$ is a closed subspace is a Hilbert space with orthogonal complement $W\subseteq V$ another Hilbert subspace such that $V=U\oplus W$. Hence every exact sequence of Hilbert $\vna(G)$-module splits. 
	\end{enumerate}
\end{Rmk}
It is natural to consider the trace of a positive operator. Nonetheless, as the group $G$ can be arbitrarily large, the trace of identity operator is often infinite. For this reason we want some suitable trace function, which nicely encodes the $G$-action:
\begin{Def}\label{1.8}
	For a positive endomorphism $f:V\to V$ of Hilbert $\vna(G)$-module, we define $\bar{f}$ to be the composite map:
\begin{equation}
\begin{tikzcd}
\bar{f}:H\otimes\ell^2(G)\rar[two heads]{\pi}  & V \rar{f} &V \rar[hook]{\iota} &H\otimes \ell^2(G)
\end{tikzcd}
\end{equation}
where $H$ is a Hilbert space into which $V$ admits an isometric $G$-embedding $\iota$,  and $\pi$ is the orthogonal projection of $H\otimes \ell^2(G)$ onto $V$. Now define the \textbf{von Neumann trace} of $f$ as:
\begin{equation}
\tr_{\vna(G)}(f):=\sum_{i\in I}\brac{\bar{f}(b_i\otimes e), b_i\otimes e}
\end{equation}
where $\{b_i\mid i\in I\}$ is a Hilbert basis for $H$. 
\end{Def}
\begin{Rmk}
	Note the definition of von Neumann trace is independent of the choice of Hilbert space $H$, the choice of basis $\{b_i\mid i\in I\}$, as well as the choice of $G$-orthogonal projection $\pi$. 
\end{Rmk}
\begin{Def}
	We define the \textbf{von Neumann dimension} of a Hilbert $\vna(G)$-module to be the trace of identity, i.e.:
	\begin{equation*}
		\dim_{\vna(G)}(V):=\tr_{\vna(G)}(\id:V\to V)
	\end{equation*}
\end{Def}
\begin{Rmk}\label{1.12}
	It is easy to derive from the definition that for any map $f$ between Hilbert $\vna(G)$-modules, $\tr_{\vna(G)}(f)=0 $ if and only $f=0$. Consequently $\dim_{\vna(G)}(V)=0$ if and only if $V=0$.
\end{Rmk}
\section{Hodge Theory}
\par In this section we will have a brief review of the theory of harmonic forms, with other pertinent topics. For a detailed discussion on this topic in the case of compact manifolds one can refer to \cite[Chapter~4]{morita2001}.
\par Let $M$ be a smooth manifold without boundary (possibly noncompact). Denote $\Omega^p(M)$ the space of smooth $p$-forms on $M$. One can take that as the spaces of sections of the exterior algebra bundle $\Lambda^p(TM\otimes_\real \complex)$ over $M$. Recall the wedge product of the exterior algebra can be extended to a map: 
\begin{align}
\wedge: \Omega^p(M) \times \Omega^q(M)\to \Omega^{p+q}(M)
\end{align}
Furthermore the $p$-th differential on the exterior algebra can also be extended to a map $d^p:\Omega^p(M)\to \Omega^{p+1}(M)$ which is uniquely determined by the following properties:
\begin{enumerate}
	\item $d^p$ is $\complex$-linear;
	\item $d^0f=\grad f$ for any $f\in \Omega^0(M)=C^\infty(M)$;
	\item (\textbf{product rule}) for all $\omega\in \Omega^p(M)$ and $\eta\in \Omega^q(M)$, we have:
	\begin{align*}
	d^{p+q}(\omega\wedge \eta)=(d^p\omega)\wedge \eta+(-1)^p\omega \wedge d^q\eta
	\end{align*}
	\item $d^{p+1}\circ d^p=0$
\end{enumerate}
then the differential $d^*$ together with $\Omega^*(M)$ defined s a cochain complex:
\begin{equation*}
\begin{tikzcd}
\cdots\rar{d^{p-2}} & \Omega^{p-1}(M) \rar{d^{p-1}} &\Omega^p(M) \rar{d^p}&\cdots
\end{tikzcd}
\end{equation*}
the cohomology of which we defined as \textbf{de Rham cohomology} of $M$ and denote by $H^p_{dR}(M)$.
\par If we further endow $M$ with a Riemannian metric and an orientation, we can define an inner product on forms with respect to the metric. Let $n=\dim(M)$. Define the Hodge star operator $\star:\Omega^p(M)\to \Omega^{n-p}(M)$ by a $C^\infty(M)$-linear operator such that:
\begin{equation}
\star (f_{i_1\cdots i_k}dx_{i_1}\wedge \cdots\wedge dx_{i_k})= \sign(I,J) f_{i_1\cdots i_k}dx_{j_1}\wedge \cdots\wedge dx_{j_{n-k}}
\end{equation}
where $j_1<\cdots <j_{n-k}$ is the rearrangement of the complement of $i_1<\cdots <i_k$ and $\sign(I, J)$ the sign of the permutation $i_1, \cdots, i_k, j_1, \cdots, j_{n-k}$.
\par Define the adjoint of the exterior differential $\delta^p=(-1)^{np+n+1}\star d \star:\Omega^p(M)\to \Omega^{p-1}(M)$, and \textbf{Laplace operator} on $p$-forms of the $M$ is defined to be:
\begin{equation}
\Delta_p:=d^{p-1}\circ \delta^p+\delta^{p+1}\circ d^p:\Omega^p(M)\to \Omega^p(M)
\end{equation}
Denote now $\Omega_c^p(M)$ the subspace of smooth $p$-forms with compact support. Then $\Omega^p_c(M)$ forms a pre-Hilbert space under the inner product:
\begin{equation}
\brac{\omega, \eta}_{L^2}:=\int_M\omega\wedge \star\eta=\int_M\brac{\omega_x, \eta_x}\mass{x}
\end{equation}
by the integrating the inner product of each fibre over the whole manifold. Denote $L^2\Omega^p(M)$ to be the completion of $\Omega^p_c(M)$ with respect to this inner product. Further define the space of \textbf{$L^2$-integrable harmonic smooth $p$-forms} we have:
\begin{equation}\label{1.55}
\mathcal{H}^p_{(2)}(M):=\{\omega\in L^2\Omega^p(M)\cap \Omega^p(M)\mid \Delta_p(\omega)=0\}
\end{equation}

\section{Survey on $G$-CW Complexes and Torsion Invariants}
In this section we provide the concepts pertaining to $G$-CW complexes and Whitehead torsion of a group $G$. These concepts are central to our discussions on $L^2$-chain complexes as well as on $L^2$-torsions. For a thorough discussion on these topics the reader is advised to consult \cite[Chapter~II]{tom1987}.
\par First we recall some basic chain constructions. 
\begin{Def}
	Given $f: C_*\to D_*$ be a chain map between chain complexes. Then the \textbf{mapping cylinder} $\cyl_*(f_*)$ to be the chain complex $C_{*-1}\oplus C_*\oplus D_*$, with differential being:
	\begin{equation}
	\begin{bmatrix}
	-c_{n-1} &0 &0\\ -\id & c_n & 0 \\ f_{n-1} & 0 & d_n
	\end{bmatrix}:C_{n-1}\oplus C_n\oplus D_n\longrightarrow C_{n-2}\oplus C_{n-1}\oplus D_{n-1}
	\end{equation}
	Define the \textbf{mapping cone} $\cone_*(f_*)$ to be the chain complex $C_{*-1}\oplus D_*$, with the differential being:
	\begin{equation}
	\begin{bmatrix}
	-c_{n-1} &0\\ f_{n-1} &d_n
	\end{bmatrix}:C_{n-1}\oplus D_n\longrightarrow C_{n-2}\oplus D_{n-1}
	\end{equation}
\end{Def}
From the definition one can easily derive the a canonical short exact sequence:
\begin{equation}\label{1.33}
\begin{tikzcd}
0 \rar &C_*(X)\rar &\cyl(C_*(f))\rar &\cone(C_*(f))\rar &0
\end{tikzcd}
\end{equation}
\begin{Def}\label{1.25}
	A \textbf{$G$-CW complex} $X$ is a $G$-space together with a $G$-invariant filtration:
	\begin{equation*}
	\emptyset=X_{-1}\subset X_0\subset \cdots \subset X_n\subset \bigcup_{n\geq 0}X_n=X
	\end{equation*}
	with $X_n$ obtained from $X_{n-1}$ by attaching equivariant $n$-dimensional cells via the following $G$-pushout:
	\begin{equation}
	\begin{tikzcd}[column sep=huge]
	\coprod_{i\in I_n} G/H_i\times S^{n-1} \dar[hook]\rar{\coprod_{i\in I}q_i} &X_{n-1}\dar{\iota}\\
	\coprod_{i\in I_n}G/H_i\times D^n\rar{\coprod_{i\in I}Q_i}&X_n
	\end{tikzcd}
	\end{equation}
	where $\{H_i\}_{i\in I}$ is a family of closed subgroups of $G$ and all maps in this pushout are $G$-equivariant, where $\iota$ is a closed embedding. Note $X$ is endowed with the colimit topology with respect to the filtration above.
\end{Def}
\begin{Rmk}
One may divest the $G$-setting by noting there is a bijection between equivariant maps $\phi:G/H\times D^n\to X_{n}$ and non-equivariant maps $\phi':D^n\to X^H$ via the following assignment:
	\begin{equation}
	\phi(gH, x)=g\cdot \phi'(x) \qquad \forall x\in X_n, g\in G
	\end{equation}
	with $X^H$ the $H$-fixed point set. More generally if $G$ is a Lie group, with $H\subseteq G$ compact subgroup, then $X^H$ inherits a $WH$-CW complex structure. Note the \textbf{Weyl group} $WH$ of $H\subseteq G$ is:
	\begin{equation}
	WH:=NH/H=\{g\in G\mid gHg^{-1}=H\}/H
	\end{equation}
\end{Rmk}
We say a $G$-space is \textbf{proper} if for all $x,y\in X$, there are open neighbourhoods $U_x$ and $U_y$ respectively such that the closure of $\{g\in G\mid gU_x\cap U_y\neq \emptyset \}$ is compact in $G$. The reader can refer to \cite[Chapter~I.3]{tom1987} for various equivalent definitions of proper action. In particular, we have a free $G$-CW complex is proper. However, not every free $G$-space is proper since $G$ due to bizarre topology.
\par A $G$-CW complex is \textbf{finite} if it is cocompact, and is of \textbf{finite type} if each $n$-skeleton is cocompact. A $G$-map $f:X\to Y$ is a \textbf{$G$-homotopy equivalence} if $f$ is a homotopy equivalence and the homotopy itself is a $G$-equivariant map (note $G$ acts on $[0,1]$ trivially). In the case when $X, Y$ are $G$-CW complexes, this means for any isotropy group of $X$ or $Y$ the induced map $f^H:X^H\to Y^H$ is a \textbf{weak homotopy equivalence}, i.e., $f_*$ induces a bijection on all homotopy groups (See e.g. \cite[Chapter~II, Proposition~2.7]{tom1987}). Similar to ordinary homotopy theory we have a equivariant version of cellular approximation theorem: 
\begin{Theo}\cite[Chapter~II, Theorem~2.1]{tom1987}\label{tom2.2.1}
Let $f: X\to Y$ a $G$-map between $G$-CW complexes. then there exists a $G$-homotopy $H:X\times [0,1]\to Y$ such that $H_0=f$ and $H_1$ is cellular.
\end{Theo}
Now suppose $G$ is discrete. The cellular $\integer G$-chain complex $C_*(X)$ of a $G$-CW complex is defined as in ordinary cellular homology. Note if one has chosen a $G$-pushout as in \Cref{1.25}, we then have chosen a preferred $\integer G$-isomorphism:
\begin{equation}
(\bigoplus_{i\in I_n}(Q_i, q_i))_*:\bigoplus_{i\in I_n} \integer[G/H_i]\longrightarrow C_n(X)
\end{equation}
by sending each $(gH_i)_{i\in I_n}$ to the element in $C_n(X)$ representing $(Q_i, q_i)(gH_i, (D^n, S^{n-1}))$. If we choose a different $G$-pushout, we obtain another isomorphism, which are, up to a sign change, differed by the composition of an automorphism which permutes the summands:
\begin{equation}\label{1.27}
\bigoplus_{i\in I_n}\epsilon_i R_{g_i}:\bigoplus_{i\in I_n}\integer [G/H_i] \longrightarrow \bigoplus_{i\in I_n} \integer[G/H_i]
\end{equation}
with $g_i\in G, \epsilon_i\in \{\pm 1\}$ and $R_{g_i}$ sends $gH_i$ to $gg_iH_i$.
\par The above discussion motivates the following discussion. Let $f:X\to Y$ a $G$-homotopy equivalence of finite $G$-CW complexes. It then induces a $G$-equivariant chain homotopy equivalence between cellular $\integer G$-chain complexes.Now we see the mapping cone $\cone_*(f_*)=:Z^*$  is a contractible finite free $\integer G$-chain complex. Choose a chain contraction $\gamma_*:Z_*\to Z_{*+1}$ and denote the $Z_{\odd}$ and $Z_{\even}$ to be the direct sum of all the odd and even degree terms of $Z_*$ respectively, we then have an isomorphism:
\begin{equation}
(d^Z+\gamma)_{\odd}: Z_{\odd}\to Z_{\even}
\end{equation}
Choose an equivariant cellular basis of $Z_*$, we can represent $(d^Z+\gamma)_*$ as an element in $GL(n, \integer G)$ with $n$ the cardinality of cellular basis. But such an isomorphism is not canonical chosen for the following reasons:
\begin{enumerate}
	\item $Z$ should be allowed to add redundant cells without changing the isomorphism, up to a certain equivalence class;
	\item $Z$ do not have a preferred ordering of the cellular basis;
	\item \ref{1.27} shows the choice of a cellular $\integer G$-basis is not quite unique, subject to a group action and a sign change
\end{enumerate}
Hence we want to define an invariant associated to $(d^Z+\gamma)_{\odd}$ that resolves the listed problems. To remedy the first problem, one can consider the isomorphism instead in 
\begin{equation*}
GL(\integer G):=\colim_{n\to \infty}GL(n, \integer G)
\end{equation*}
with the colimit is taken by embedding $GL(n, \integer G)$ into the upper-left block of $GL(n+1, \integer G)$.
\par To resolve the second problem, we require our invariant to be a \textbf{simple $G$-homotopy invariance} (see \cite[Chapter~I.4]{luck2006} for definitions and general discussions). Heuristically speaking, this allows the $G$-homotopy equivalence to be constituted by a series of elementary expansions and collapses, each of which take place at individual cells. These actions correspond to elementary matrices in $GL(\integer G)$, hence we want to modulo these elements as well. By \textbf{Whitehead Lemma} \cite[Lemma~3.1]{milnor2016} the elementary matrices are exactly the derived group of $GL(\integer G)$, hence it motivates the following abelianization of $GL(\integer G)$:
\begin{equation}
K_1(\integer G):=GL(\integer G)/[GL(\integer G), GL(\integer G)]
\end{equation}
Now the last problem gives rise to the following definition:
\begin{Def}\label{3.1.2}
	Given $f:X\to Y$ a $G$-homotopy equivalence of finite free $G$-CW complexes as above. We define the \textbf{Whitehead group} $\Wh(G)$ to be the cokernel of the following map:
	\begin{equation}
	G\times \{\pm 1\}\to K_1(\integer G) \qquad (g, \pm 1)\mapsto [(\pm g)]
	\end{equation}
	where $(\pm g)$ the class of $1\times 1$-matrix. We also define \textbf{Whitehead torsion} $\tau(f)$ of $f$ to be an element in $\Wh(G)$ that is the image of the class $(d^{\cone_*(C_*(f))}+\gamma)_{\odd}\in K_1(\integer G)$ under the canonical projection $K_1(\integer G)\twoheadrightarrow \Wh(G)$.
\end{Def}




\chapter{Analytic and Topological $L^2$-Invariants}\label{chapter2}
In this chapter we will use spectral density function, Borel functional calculus and heat kernel to elicit a unified approach to various $L^2$-invariants. This approach renders more insight into the Riemannian structure of the underlying manifold, whereas a careful analysis of spectrum of the Laplace operators will be conducive. In the second part of this chapter we will have a glimpse of the topological aspects of $L^2$-invariants, and conclude this chapter by bridging the topological world with their analytic counterparts.

\section{Spectral Density Function}
We first deal with spectral density function, which is central to the definition of Novikov-Shubin invariant and analytic definition of $L^2$-torsion.
\begin{Def}\label{2.1}
	Let $f:\dom(f)\in U\to V$ be $G$-equivariant closed densely defined operator between Hilbert $\vna(G)$-modules. For $\lambda>0$, define:\begin{equation*}
	\mathcal{L}(f,\lambda):=\{L\subset \dom(f)  \text{ is a Hilbert }\vna(G)\text{-submodule} \mid \forall x\in L, \norm{f(x)}\leq \lambda \norm{x}\}
	\end{equation*}
	The \textbf{spectral density function} of $f$ is defined to be:
	\begin{align*}
	F(f):[0, \infty)\longrightarrow[0, \infty] \qquad \lambda \mapsto \sup\{\dim_{\vna(G)}(L)|L\in \mathcal{L}(f,\lambda)\}
	\end{align*}
\end{Def}
The spectral density function of a $f$ captures the dimension of space on which $f$ has operator norm less than a constant $c$. In the case of compact operators (or more generally of those operators with only discrete spectrum), this is just the direct sum of eigenspaces with corresponding eigenvalues in the ball centred at zero of radius $c$: $B_0(c)\subset \complex$. 
\par The spectral density function can be related to the spectral measure in \Cref{spectral measure} via following lemma:
\begin{Lemma}\label{2.3}
	Let $U$ and $V$ be Hilbert $\vna(G)$-modules. Let $f:\dom(f)\subset U\to V$ be a $G$-equivariant closed densely defined operator. Then for $\lambda\in \real$ and $x\in \dom(f)$:
	\begin{align*}
	\norm{f(x)}\begin{cases}
	>|\lambda|\cdot\norm{x}\qquad &\text{if } E_{\lambda^2}^{f^*f}(x)=0, x\neq 0;\\
	\leq |\lambda|\cdot \norm{x}\qquad & \text{if } E_{\lambda^2}^{f^*f}(x)=x
	\end{cases}
	\end{align*}
	where $\{E_{\lambda^2}^{f^*f}\}$ is the spectral family associated to $f^*f$. Furthermore, the spectral projections $E_{\lambda^2}^{f^*f}$ are $G$-equivariant and
	\begin{equation}
	F(f)(\lambda)=\dim_{\vna(G)}(\im (E_{\lambda^2}^{f^*f}))
	\end{equation}
\end{Lemma}
\begin{proof}
To prove the first part, pick $x\in U$ nonzero such that $E^{f^*f}_{\lambda^2}(x)=0$. Since $\dom(f)$ is dense in $U$, hence is $\dom(f^*f)$ dense in $U$. Also $\dom(f^*f)$ is a \textbf{core} for $\dom(f)$, that is, for any $x\in \dom(f)$ there is a sequence $x_n\in \dom(f^*f)$ such that $\lim_{n\to \infty}x_n=x$ and $\lim_{n\to \infty}f(x_n)=f(x)$. Note the spectral projections are right-continuous with respect to spectrum, hence we have $\lim_{\mu\downarrow\lambda^2}E_{\mu}^{f^*f}(x)=E^{f^*f}_{\lambda^2}(x)=0$ and there exists a $\epsilon>0$ such that:
\begin{equation*}
\brac{E_{\lambda^2+\epsilon}^{f^*f}(x), x}<\frac{1}{2}\brac{x,x}
\end{equation*}
Recall now \ref{1.65}, we have:
\begin{align*}
\norm{f(x_n)}&=\int_0^\infty\mu \mass{\brac{E^{f^*f}_\mu(x_n), x_n}}\\
&\geq \int_{(\lambda^2, \lambda^2+\epsilon]}\mu \mass{\brac{E^{f^*f}_\mu(x_n), x_n}}+\int_{(\lambda^2+\epsilon, \infty)}\mu \mass{\brac{E^{f^*f}_\mu(x_n), x_n}}\\
&\geq \lambda^2\cdot \brac{E^{f^*f}_{\lambda^2+\epsilon}(x_n), x_n}-\brac{E^{f^*f}_{\lambda^2}(x_n), x_n}+(\lambda^2+\epsilon)\cdot(\brac{x_n, x_n}-\brac{E^{f^*f}_{\lambda^2+\epsilon}(x_n), x_n})\\
&=(\lambda^2+\epsilon)\cdot\norm{x_n}-\epsilon\cdot\brac{E^{f^*f}_{\lambda^2+\epsilon}(x_n), x_n}-\lambda^2\cdot \brac{E^{f^*f}_{\lambda^2}(x_n), x_n}
\end{align*}
Taking the limit $n\to \infty$ on both sides of the inequality, we have, by the above choice of $\epsilon$:
\begin{align*}
\norm{f(x)}^2&\geq (\lambda^2+\epsilon)\cdot \norm{x}^2-\epsilon\cdot\brac{E^{f^*f}_{\lambda^2+\epsilon}(x), x}-\lambda^2\cdot \brac{E^{f^*f}_{\lambda^2}(x), x}\\
&> (\lambda^2+\epsilon)\cdot \norm{x}^2-\frac{\epsilon}{2}\cdot \norm{x}^2-0>\lambda^2\cdot\norm{x}^2
\end{align*}
Hence we have proved the first part of the lemma. Note the first part directly implies that $\im(E^{f^*f}_{\lambda^2})\in \mathcal{L}(f,\lambda)$, which by taking supremum we have $\dim_{\vna(G)}(\im (E_{\lambda^2}^{f^*f}))\leq F(f)(\lambda)$. 
\par Moreover, the first part also implies $E^{f^*f}_{\lambda^2}|_L=\id_L$ for all $L\in \mathcal{L}(f, \lambda)$. Since $L\subseteq \im(E^{f^*f}_{\lambda^2})$ and from the additivity of von Neumann dimension with respect to weakly exact sequence of Hilbert $\vna(G)$-modules, we have the $\dim_{\vna(G)}(L)\leq \dim_{\vna(G)}(\im (E_{\lambda^2}^{f^*f}))$ for $L\in \mathcal{L}(f, \lambda)$.
\end{proof}
\begin{Rmk}
	As a special case of this lemma, we can easily derive from a similar argument that if$f$ being a positive operator, then $F(f)(\lambda)=\dim_{\vna(G)}(\im (E^{f}_{\lambda}))$.
\end{Rmk}
More generally we can define \textbf{density function} $F:[0, \infty)\to [0, \infty]$ to be a monotone non-decreasing and right-continuous function. To each density function one can associate a unique Borel measure via the following procedure: First define a pre-measure on all the half-open sets as:
\begin{equation*}
\mu((a,b]):=F(b)-F(a)
\end{equation*}
then extend it to all Borel sets on the real line via Carath\'eodory extension theorem (See for instance \cite[Theorem~1.41]{klenke2013}), this then defines a Borel measure, which is in fact the spectral measure associated to the trace function of $f$, in the sense of \Cref{1.8}:
\begin{Lemma}\label{F=dim}
	For $f:\dom(f)\subset U\to V$ a $G$-equivariant positive closed densely defined operator between Hilbert $\vna(G)$-modules, we have, for any Borel subset $S$ of $\real$:
	\begin{equation}
	\mu^f(S)=\tr_{\vna(G)}E^f(S)
	\end{equation}
	where $E^f$ is the spectral measure associated to $f$ and $\mu^f$ is the measure defined using spectral density function, as above:
\end{Lemma}
\begin{proof}
	Since the half-open sets $\{(-\infty, \lambda]\mid \lambda\in \real\}$ forms a base for Borel sets, it suffices to prove the statement for these sets. Now from \Cref{2.3}:
	\begin{equation*}
	\mu^f((-\infty, \lambda])=F(f)(\lambda)-F(f)(-\infty)=F(f)(\lambda)=\dim_{\vna(G)}(\im E^f_\lambda)
	\end{equation*}
	Now since $E^f_\lambda$ is a projection, so in particular its image is closed \cite[Chapter~II, Proposition~3.2]{conway2013}, hence we can choose a basis $I=I'\sqcup I''$ of $U$ such that $\{e_i\mid i\in I'\}$ form a base of $\ker E^f_\lambda$ and $\{e_j\mid j\in I''\}$ form a base of $\im E^f_\lambda=(\ker E^f_\lambda)^\perp$.
	\par Now by considering the canonical isometric linear $G$-embedding of $V\inj V\otimes \ell^2(G)$, we have $\bar{f}|_{V\otimes e}=f$ where $\bar{f}$ is as defined in \Cref{1.8}. Now
	\begin{equation}
	\begin{split}
	\dim_{\vna(G)}(\im E_\lambda^f)=\sum_{i\in I''}\brac{E^f_\lambda e_i, e_i}=\sum_{i\in I}\brac{E^f_\lambda e_i, e_i}=\tr_{\vna(G)}(E^f_\lambda)
	\end{split}
	\end{equation}
\end{proof}
\par An important invariant associated to each density function, which measures its asymptotic behaviour when approaching zero:
\begin{Def}\label{2.8}
	Let $F$ be a \textbf{Fredholm} density function, i.e., there exists a $\lambda>0$ such that $F(\lambda)<\infty$. We then define its \textbf{Novikov-Shubin invariant} to be $\alpha(F)$, as follows:
	\begin{equation}
	\alpha(F):=\begin{cases}
	\liminf_{\lambda\downarrow 0}\frac{\ln(F(\lambda)-F(0))}{\ln(\lambda)}\in [0,\infty], \qquad &\text{if } \forall \lambda>0, F(\lambda)>F(0);\\
	\infty^+\qquad &\text{if otherwise}
	\end{cases}
	\end{equation}
\end{Def}
\begin{Rmk}
	Note here the use of $\infty^+$ is a convention we adopted to make Novikov-Shubin invariant better fitting into settings such as additivity and also to distinguish the cases in which the density functions behaves abnormally from those of which that the density function takes a constant value around zero. For further details, please refer to \cite[Notation~2.10]{luck2013}.
\end{Rmk}
The reader might observe only the asymptotic behaviour determines the value of Novikov-Shubin invariant, and this motivates a comparison of density function near $0$. Indeed we write $F\preceq G$ if there are $C>0$ and $\epsilon>0$ such that:
\begin{equation}
\forall \lambda\in [0, \epsilon], \qquad F(\lambda)\leq G(C\cdot \lambda)
\end{equation}
Moreover, we write $F\simeq G$ if $F\preceq G$ and $G\preceq F$. Some immediate consequences are recorded down here:
\begin{Lemma}\label{2.11}
	If $F$ and $F'$ are density functions with $F'$ be Fredholm. Let $f:U\to V$ be a morphism of $\vna(G)$-Hilbert modules. Then:
	\begin{enumerate}
		\item If $F\preceq F'$, then $F$ is Fredholm and $b^{(2)}(F)\leq b^{(2)}(F')$; If further $b^{(2)}(F)=b^{(2)}(F')$, then $\alpha(F)\geq \alpha(F')$. In particular, if $b^{(2)}(F)=b^{(2)}(F')$ and $F\simeq F'$, we have $\alpha(F)=\alpha(F')$;
		\item If $i:V\to V'$  injective with closed image, and $p:U\to U'$ surjective and $b^{(2)}(p)$ is finite, then $f$ is Fredholm if and only if $i\circ f\circ p$ is Fredholm, in which case $\alpha(i\circ f\circ p)=\alpha(f)$;
		\item $f$ is an isomorphism, then $b^{(2)}(f)=0$, $f$ is Fredholm, and $\alpha(f)=\infty^+$;
		\item If $F, F'$ are Fredholm, then $\alpha(F+F')=\min\{\alpha(F), \alpha(F')\}$.
	\end{enumerate} 
\end{Lemma}
\begin{proof}
	1. follows directly from definition; 2. is a consequence of Inverse Mapping Theorem, which implies
	\begin{equation}\label{2.11[9]}
	F(i\circ f\circ p)(\lambda)\simeq F(f)(\lambda)-\dim_{\vna(G)}(\ker(p))
	\end{equation}
		3. follows from the fact that each isomorphism is bounded from below, hence the spectrum is disjoint from $0$. Then the spectral density function $F(f)$ is constant in a neighbourhood of $0$ hence $\alpha(f)=\infty^+$. This can be seen by applying Borel functional calculus to $f^*f$.
	\par To prove 4., we may assume without loss of generality that $b^{(2)}(F)=b^{(2)}(F')=0$. Note $\alpha(F+F')\leq \min\{\alpha(F), \alpha(F') \}$ is direct from the first part. To prove the reverse inequality, It suffices to consider the case when $\alpha(F)\leq \alpha(F')$ and $0<\alpha(F)\leq \infty$. Choose any $\alpha>0$ such that $\alpha(F)>\alpha$. Then we have for sufficiently small $\lambda$, we have: $F(\lambda), F'(\lambda)\leq K\lambda^\alpha$ for some constant $K$. Then $(F+F')(\lambda)\leq 2K\lambda^\alpha$ which shows $\alpha(F+F')\geq \alpha$. 
	\end{proof}

\section{Fugelede-Kadison determinant}
Based on spectral density function we can define an notion of determinant on morphisms of Hilbert $\vna(G)$ modules. This will be crucial to our later study of $L^2$-torsion, where one need to distinguish isomorphisms from each other.
\begin{Def}\label{3.11}
	Let $f:U\to V$ be a morphism of finite dimensional Hilbert $\vna(G)$-modules. We call $f$ is of \textbf{determinant class} if:
	\begin{equation}
	\int_{0^+}^\infty\ln(\lambda)\mass{F(\lambda)}>-\infty
	\end{equation}
	Define its \textbf{(generalized) Fuglede-Kadison determinant} by:
	\begin{equation}
	\det{}_{\vna(G)}(f):=\begin{cases}
	\exp(\int_{0^+}^\infty\ln(\lambda)\mass{F})\qquad &\text{if }\int_{0^+}^\infty\ln(\lambda)\mass{F}>-\infty\\
	0 \qquad &\text{if otherwise}
	\end{cases}
	\end{equation}	
\end{Def}
A few properties directly derived from definition and \Cref{2.11} are listed here:
\begin{Lemma}\label{3.15}
	Let $f:U\to V$ be a morphism of finite dimensional Hilbert $\vna(G)$-modules with respective spectral density function. Then:
\begin{enumerate}
	\item If $f$ is invertible, we get $\det(f)=\exp (\frac{1}{2}\tr(\ln(f^*f)))$;
	\item If $f^\perp: \ker(f)^\perp \to \bar{\im(f)}$ the induced weak isomorphism, then $\det(f)=\det(f^\perp)$.
	\item $\det(f)=\det(f^*)=\sqrt{\det(f^*f)}=\sqrt{\det(ff^*)}$;
\end{enumerate}
\end{Lemma}
\begin{proof}
	1. is a direct consequence of definition and \Cref{F=dim}:
	\begin{equation*}
	\tr(\ln(f^*f))=\int^\infty_{0+}\mass{\tr(E^{f^*f}_\lambda)}=2\cdot \int^\infty_{0+}\ln(\lambda)\mass{F(f)}
	\end{equation*}
	2. and 3. are directly from \Cref{2.11}. Note since the determinant exclude point $0$. We may then suffice to consider $F(f^\perp)=F(f)-F(0)$.
\end{proof}
The reader is to observe that the Fuglede-Kadison determinant behaves similarly as conventional determinant on finite-dimensional matrices.
\begin{Lemma}\label{3.14}
	Let $f:U\to V$ and $g:V\to W$ be morphisms of finite-dimensional Hilbert $\vna(G)$-modules such that $f$ has dense image and $g$ is injective. Then:
	\begin{equation*}
	\deg(g\circ f)=\det(g)\cdot \det(f)
	\end{equation*}
	Meanwhile, given $f_1:U_1\to V_1, f_2:U_2\to V_2$ and $f_3:U_2\to V_1$ be morphisms of finite-dimensional Hilbert $\vna(G)$-modules, such that $f_1$ has dense image and $f_2$ is injective. Then:
	\begin{equation*}
	\det\begin{pmatrix}
	f_1 &f_3\\ 0 &f_2
	\end{pmatrix}=\det(f_1)\cdot \det(f_2)
	\end{equation*}
\end{Lemma}
\begin{proof}[Sketch of Proof]
	Proof of the first assertion can be found in \cite[Theorem~3.14]{luck2013}. The basic idea is as follows: By \nameref{polar decomposition} and the fact $f^*$ and $g$ both are injective, it suffices to establish the case when $f$ and $g$ are injective positive morphisms. First consider the case when both are invertible, then by holomorphic functional calculus (c.f. \cite[Lemma~3.18]{luck2013}) to integrate a path between $\tr(\ln(gf^2g))$ and $\tr(\ln(g^2))$, so as to prove:
	\begin{equation*}
	\tr(\ln(gf^2g))=2\tr(\ln(g)+\ln(f))
	\end{equation*}
	Next consider injective positive morphisms, in which case the operator norms are bounded from below. Hence there is a gap in the spectrum around $0$, now we can choose $\epsilon$ and $\delta$ small enough such that $f-\epsilon\cdot \id$ and $g-\delta\cdot \id$ are both invertible positive morphisms, hence apply the first scenario. Then approach both $\epsilon$ and $\delta$ to $0$, in which we get:
	\begin{equation*}
	\det(gf^2g)=\det(f)^2\cdot \det(g)^2
	\end{equation*}
	and now $\det(g\circ f)=\sqrt{\det(gf)^2}=\sqrt{\det(gf^2g)}=\sqrt{\det(f)^2\det(g)^2}=\det(f)\det(g)$. 
	\par Now the second assertion follows from the first, and the following manipulation:
\begin{equation*}
\begin{pmatrix}
f_1 &f_3\\0 &f_2
\end{pmatrix}=\begin{pmatrix}
1 &0\\ 0& f_2
\end{pmatrix}\begin{pmatrix}
1 &f_3\\ 0& 1
\end{pmatrix}\begin{pmatrix}
f_1 &0\\ 0 &1
\end{pmatrix}
\end{equation*}
 and the fact that we can write:
 \begin{equation*}
 \begin{pmatrix}
 1 &f_3 &0\\ 0 &1 & 0\\ 0 &0 & 1
 \end{pmatrix}=\Bigg[\begin{pmatrix}
 1 &0 &0\\ 0 &1 &0\\ 0 &-f_3 &1
 \end{pmatrix}, \begin{pmatrix}
 1 &0 &1\\ 0&1 &0\\ 0 &0 &1
 \end{pmatrix}\Bigg]
 \end{equation*}
 where $[A, B]=ABA^{-1}B^{-1}$ is the commutator.
\end{proof}
\section{Definition of Analytic $L^2$-Invariants}\label{def}
\textbf{Throughout this section we adopt the settings as in \nameref{l2hdr}, i.e., Let $M$ be a cocompact free proper $G$-manifold without boundary with $G$-invariant Riemannian metric.} We then define the \textbf{heat operator} $e^{-t\Delta_*}$ of Laplace operator to be:
\begin{equation}
e^{-t\Delta_p}:M\times M\to \Hom(\pi^*_1\Lambda^p(TM), \pi_2^*\Lambda^p(TM))
\end{equation}
a smooth section, with $\pi_i:M\times M\to M$ the canoncial projection to the $i$-th factor. Note by functional calculus we can define, for $x\in M$, an bounded linear operator $e^{-t\Delta_p}(x):L^2\Omega^p(M)\to L^2\Omega^p(M)$ by:
\begin{equation}
e^{-t\Delta_p}(x)(\omega)=\int_M K_p(t,x,y)(\omega_y)\mass{\vol_y}
\end{equation}
where $K_p(t,x,y)$ is the \textbf{heat kernel}, which is the smooth Schwartz kernel corresponds to the integral operator $e^{-t\Delta_p}(x,y)$.
\begin{Rmk}
	The reader is to note in general cases of noncompact manifold the heat operator is not of trace class, i.e., $\tr_\complex(K_p(t,x,y))$ is not integrable over $M$, or the heat kernel may not exist (even as a distribution) \textit{a priori}. Nonetheless, For our case when $M$ has a properly discontinuous group of isometries $\Gamma$ acting on $M$ such that the quotient is compact, then the heat kernel $K_M$ of $M$ exists, and is related to heat kernel $K_{\Gamma\backslash M}$ of the compact quotient $\Gamma\backslash M$, via:
	\begin{equation}
	K_{\Gamma\backslash M}(t, x, y)=\sum_{\gamma\in \Gamma}K_M(t, x, \gamma\cdot y)
	\end{equation}
	For more details in this aspect, the readers is advised to consult \cite[Chapter VI.4]{chavel1984} for the compact case and \cite{donnelly1979} for the noncompact case.
\end{Rmk}
We are now ready to define the analytic $L^2$-Betti number:
\begin{Def}\label{1.60}
	The \textbf{analytic $p$-th $L^2$-Betti number}, which we also denote as $b_p^{(2)}(M)$, is defined by:
		\begin{equation}
		b^{(2)}_p(M):=\lim_{t\to \infty}\int_{\mathcal{F}}\tr_{\complex}(e^{-t\Delta_p}(x,x))\mass{\vol}
		\end{equation}
		where $\mathcal{F}$ is a \textbf{fundamental domain} for the $G$-action, i.e., an open subset $\mathcal{F}\subset M$ such that: \begin{enumerate}
			\item $M=\bigcup_{g\in G}g\cdot \bar{\mathcal{F}}$;
			\item $g\cdot \mathcal{F}\cap \mathcal{F}\neq \emptyset$ if and only if $g=1$;
			\item  the topological boundary $\partial \mathcal{F}$ has measure zero.
		\end{enumerate}
\end{Def}
Hence in our case integrating over $\mathcal{F}$ is same as integrating over $\bar{\mathcal{F}}$, which is actually a compact manifold (possibly with boundary). Thus there is no ambiguity in the heat kernel not being of trace class.
\begin{Def}\label{2.64}
Given $d^p:\Omega^p_c(M)\to L^2\Omega^{p+1}(M)$ and $\Delta_p: \Omega_c^p(M)\to L^2\Omega^p(M)$. Denote:
\begin{equation*}
(d^p_{\min})^\perp: \dom(d^p_{\min})\cap \im (d^{p-1}_{\min})^\perp \to \im(\delta^{p+2}_{\min})^\perp 
\end{equation*}	
the operator induced by $d^p_{\min}$. We then define the \textbf{analytic $p$-th spectral density function} of $M$ by:
\begin{equation*}
F_p(M):=F((d^p_{\min})^\perp) \qquad F^\Delta_p(M):=F((\Delta_p)_{\min})
\end{equation*}
and define their respective \textbf{analytic $p$-th Novikov-Shubin invariant} of $M$ by:
\begin{equation*}
\alpha_p(M):=\alpha(F_{p-1}(M)) \qquad \alpha_p^{\Delta}(M):=\alpha(F^\Delta_p(M))
\end{equation*}
\end{Def}
Lastly we define the $L^2$-analytic torsion. This, as an analog of Whitehead torsion, is a secondary invariant, i.e., it is not a $G$-homotopy invariant. To prepare for the definitions, We need first an analytic version of determinant class. Later in \Cref{determinant class} we prove this is equivalent to \Cref{3.11}. 
\par Recall the \textbf{Laplace transform} of a function $f:\real\to \real$ is defined by:
\begin{equation}
\theta_f(t)=\int_{0}^{\infty}e^{-t\lambda}f(\lambda)\mass{\lambda}
\end{equation}
\begin{Def}\label{3.128}
	$M$ is of \textbf{analytic determinant class} if for each $0\leq p\leq \dim (M)$, there exists a $\epsilon>0$ such that:
	\begin{equation}
	\int_{\epsilon}^{\infty}t^{-1}\cdot (\theta_p(t)-b_p^{(2)}(M))\mass{t} <\infty
	\end{equation}
	where
	\begin{equation}
	\theta_p(t):=\theta_{F^{\Delta}_p}(t)=\int_{0}^{\infty}e^{-t\lambda}\mass{F^\Delta_p(\lambda)}
	\end{equation}
\end{Def}
\begin{Rmk}\label{3.138}
As a direct consequence of \Cref{F=dim}, we see:
\begin{equation}
\begin{split}
\theta_p(t)&=\int_{0}^{\infty}e^{-t\lambda}\mass{(\tr_{\vna(G)}(E^{\Delta_p}_\lambda))}\\
&=\tr_{\vna(G)}(\int_{0}^{\infty}e^{-t\lambda}\mass{E^{\Delta_p}_\lambda})\\
&=\tr_{\vna(G)}(e^{-t(\Delta_p)_{\min}})\\
&=\int_{\mathcal{F}}\tr_{\complex}(e^{-t\Delta_p}(x,x))\mass{\vol}
\end{split}
\end{equation}
where the second equality follows from \ref{1.65} or the general fact that the trace function is ultra-weakly continuous. So this function indeed is the trace of heat kernel.
\end{Rmk}
\begin{Lemma}\label{3.139}
	For a Fredholm spectral density function $F$, we have, for all $\lambda>0$:
	\begin{equation}
	F(\lambda)\leq \theta_F(t)\cdot e^{-t\lambda}
	\end{equation}
	Moreover, $\theta_F(t)<\infty$ for all $ t>0$ if and only if for all $t>0$, there is a constant $C(t)$ such that $F(\lambda)\leq C(t)\cdot e^{-t\lambda}$ holds for all $\lambda\geq 0$. In this case we have:
	\begin{equation}\label{3.140}
	\theta_F(t)=t\cdot \int_{0}^\infty e^{-t\lambda}\cdot F(\lambda)\mass{\lambda}
	\end{equation}
\end{Lemma}
\begin{proof}
	Since $e^{-t\lambda}$ is absolutely continuous and $F$ is integrable on $[0, K]$ \footnote{Note $F$ is not in general integrable, but for our case the von-Neumann dimension of $L^2$-cohomology are finite at each dimension due to the the manifold being co-compact, whence we may take $F$ being integrable on every bounded set.}, we can then use integration by parts and monotone convergence theorem for $0<\epsilon<K<\infty$,
\begin{equation*}
\begin{split}
\int_{\epsilon^+}^K e^{-t\lambda} \mass{F(\lambda)}&=e^{-tK}\cdot F(K)-e^{-t\epsilon}\cdot F(\epsilon)+t\int_{\epsilon}^{K}e^{-t\lambda}\cdot F(\lambda)\mass{\lambda}\\
&\overset{\epsilon\to 0^+}{\longrightarrow}e^{-tK}\cdot F(K)-F(0)+t\int_{0}^{K}e^{-t\lambda}\cdot F(\lambda)\mass{\lambda}\\
\end{split}
\end{equation*}
Adding $F(0)$ on both sides, we then have:
\begin{equation}\label{3.141}
e^{-tK}\cdot F(K)=(1-t)\cdot \int_{0}^{K}e^{-t\lambda}\cdot F(\lambda)\mass{\lambda}\leq (1-t)\theta_p(t)
\end{equation} 
Hence the first part of the lemma is proved.
\par Now assume $\theta_F(t)$ is finite, we have $F(\lambda)\in \mathcal{O}(e^{t\lambda})$ for all fixed $t$. To see the other direction is true as well, assume $F(\lambda)\leq C(t/2)\cdot e^{-\frac{t}{2}\lambda}$, and we then have for $t>0$, $\lim_{t\to \infty}e^{-t\lambda}\cdot F(\lambda)=0$. Now by tending $K\to \infty$ in \ref{3.141} and we then see \ref{3.140} readily follows from Monotone convergence theorem. Moreover,
\begin{equation}
\theta_F(t)\leq t\cdot \int_0^\infty e^{-t\lambda}\cdot C(t/2)\cdot e^{\frac{t}{2}\lambda}\mass{\lambda}=2C(t/2)
\end{equation}
So the other direction is proved as well.
\end{proof}
From this lemma we can conclude the following:
\begin{Prop}\label{3.136}
	If $\alpha_p^\Delta(M)=\infty^+$, then there exists a $\epsilon>0$ and a constant $C(\epsilon)>0$ such that for all $t>0$, 
	\begin{equation}
	\theta_p(M)\leq C(\epsilon)\cdot e^{\epsilon t}
	\end{equation} 
	If $\alpha^{\Delta}_p(M)\neq \infty^+$. Then:
	\begin{equation}\label{3.143}
	\alpha^\Delta_p(M)=\liminf_{t\to \infty}\frac{-\ln (\theta_p(t)-b_p^{(2)}(M))}{\ln(t)}
	\end{equation}
\end{Prop}
\begin{Rmk}
	Before giving the proof we need to justify that right-hand side of \ref{3.143} indeed gives a positive number. First observe $\theta_p(t)$ is monotone non-increasing with respect to $t$, and by dominated convergence theorem we have $\lim_{t\to \infty}\theta_p(t)=F(0)=b^{(2)}_p(M)$. Hence we see $-\ln(\theta_p(t)-F(0))$ and $\ln(t)$ always have the same sign for sufficiently large $t$.
\end{Rmk}
\begin{proof}
	Taking $F=F^\Delta_p$ and first consider the case when $\alpha^\Delta_p(M)=\infty^+$. Recall \Cref{2.8}, this means there is a gap in the spectrum of $\Delta_p$ at zero, i.e., $F(\lambda)=F(0)$ holds for $0<\lambda<\varepsilon$ for some $\varepsilon>0$. For $t> 2$:
	\begin{equation}
	\begin{split}
	\theta_p(t)-F(0)&=t\cdot \int_{0}^{\infty}e^{-t\lambda}(F(\lambda)-F(0))\mass{\lambda}\\
	&\leq t\cdot \int_{\epsilon}^{\infty}e^{-t\lambda}\cdot F(\lambda)\mass{\lambda}\\
	&\leq t\cdot \int_{\epsilon}^{\infty}e^{-t\lambda}\cdot \theta_p(1)\cdot e^\lambda\mass{\lambda}\\
	&=\theta_p(1)t\cdot \int_{\epsilon}^{\infty}e^{(-t+1)\lambda}\mass{\lambda}\\
	&=2\theta_p(1)e^\epsilon\cdot e^{-t\epsilon}
	\end{split}
	\end{equation}
	Note the second inequality follows from the first part of the lemma. Hence $\theta_p(t)-F(0)\leq C(\epsilon)\cdot e^{-t\epsilon}$.
	\par Now if $\alpha^\Delta_p(M)\neq \infty^+$, by possibly replacing $F$ by $F(\lambda)-F(0)$ we may assume $F(0)=0$. Then \ref{3.143} amounts to say:
	\begin{equation}
	\liminf_{\lambda\downarrow 0}\frac{\ln F(\lambda)}{\ln(\lambda)}= \liminf_{t\to \infty}\frac{-\ln(\theta_p(t))}{\ln(t)}
	\end{equation}
	We may again assume the left-hand side is great than 0, for the other case is trivially true. Then consider $\liminf_{\lambda\downarrow 0}\frac{\ln(F(\lambda))}{\ln(\lambda)}>\alpha>0$. This means we can find a $\epsilon>0$ such that $F(\lambda)\leq \lambda^\alpha$ for all $\lambda\in (0, \epsilon)$. From the lemma above we yield:
	\begin{equation*}
	\begin{split}
	\theta_p(t)&\leq t\cdot \int_{0}^{\epsilon}e^{-t\lambda}\cdot \lambda^\alpha\mass{\lambda}+t\cdot \int_{\epsilon}^\infty e^{-t\lambda}\theta_F(1)\cdot e^\lambda\mass{\lambda}\\
	&\leq t\cdot \int_0^\infty e^{-t\lambda}\cdot \lambda^\alpha\mass{\lambda}+t\theta_p(t)\cdot \int_{\epsilon}^{\infty}e^{(-t+1)\lambda}\mass{\lambda}\\
	&\leq t\cdot \int_{0}^{\infty}e^{-t\lambda}\cdot \lambda^\alpha\mass{\lambda}+\theta_p(1)\frac{t}{t-1}\cdot e^{(-t+1)\epsilon}\\
	&=\Gamma(\alpha+1)\cdot t^{-\alpha}+\theta_p(1)\frac{t}{t-1}\cdot e^{(-t+1)\epsilon}
	\end{split}
	\end{equation*}
	From the above equality we see the $\theta_p(t)=\mathcal{O}(t^{-\alpha})$ as $t\to \infty$. Now:
	\begin{equation} 
	\ln(\theta_p(t))=\mathcal{O}(\ln(t^{-\alpha}))=\mathcal{O}(-\alpha\ln(t))
	\end{equation}
	as $t\to \infty$. Hence we conclude $\alpha\leq \liminf_{t\to \infty} \frac{-\ln(\theta_p(t))}{\ln(t)}$. This proves $\liminf_{\lambda\downarrow 0}\frac{\ln F(\lambda)}{\ln(\lambda)}$ is less than $\liminf_{t\to \infty}\frac{-\ln(\theta_p(t))}{\ln(t)}$.
	\par To prove $\liminf_{\lambda\downarrow 0}\frac{\ln F(\lambda)}{\ln(\lambda)}\geq \liminf_{t\to \infty}\frac{-\ln(\theta_p(t))}{\ln(t)}$, again we suffices to treat the case when the right hand side is larger than zero. Then we can find a $K>0$ such that $\theta_p(t)\leq t^{-\alpha}$ for all $t>K$. From first part of the lemma one has for $t\geq\max(K, \lambda^{-1})$:
	\begin{equation*}
	F(\lambda)\leq e^{-t\lambda}\cdot \theta_p(t)\leq e^{-t\lambda}\cdot t^{-\alpha}\leq e\cdot \lambda^\alpha
	\end{equation*}
	Then we have $\liminf_{t\to \infty}\frac{\ln(F(\lambda))}{\ln\lambda}\geq \alpha$, and the equality is proved. 
\end{proof}
\begin{Rmk}
	A sufficient condition for a manifold to be of analytic determinant class is $\alpha^{\Delta}_p(M)>0$ for all $0\leq p\leq \dim(M)$. This fact follows readily from the proposition. For the case $\alpha^\Delta_p(M)=\infty^+$, we have $\theta_p(t)$ bounded by $e^{-\epsilon t}$, and $t^{-1}e^{-t\epsilon}$ is integrable; On the other hand when $0<\alpha^\Delta_p(M)<\infty^+$, we then may assume that there is a $\varepsilon>0$ such that $\alpha^\Delta(M)>\varepsilon$. Now from the second part of \Cref{3.136} we see there is a $K$ such that for all $t>K$, 
	\begin{equation}
	\theta_p(t)-F^\Delta_p(0)\leq t^{-\varepsilon}
	\end{equation}
	Now $\int_{\epsilon}^{\infty}t^{-1}\cdot(\theta_p(t)-F^\Delta_p(0))\mass{t}\leq \int_{\epsilon}^{\infty}t^{-1-\varepsilon}\mass{t}<\infty$ for some $\epsilon>0$. Hence we have in this case the manifold is of analytic determinant class as well.
\end{Rmk}
As a final remark we note the analytic determinant class (\Cref{3.128}) is indeed equivalent to the determinant class (\Cref{3.11}), via the following lemma:
\begin{Lemma}\label{3.15eq}
	For any $F:[0, \infty)\to [0, \infty)$ finite spectral density function, we have:
	\begin{equation}
	\int_{\epsilon^+}^{a}\ln(\lambda)\mass{F}=-\int_{\epsilon}^{a}\lambda^{-1}\cdot (F(\lambda)-F(0))\mass{\lambda}+\ln(a)\cdot (F(a)-F(0))-\ln(\epsilon)\cdot (F(\epsilon)-F(0))
	\end{equation}
	Moreover, we have: 	$\int_{\epsilon^+}^{a}\ln(\lambda)\mass{F}>-\infty$ if and only if $\int_{\epsilon}^{a}\lambda^{-1}\cdot (F(\lambda)-F(0))\mass{\lambda}<\infty$, in which case:
	\begin{gather*}
	\lim_{\lambda\to \infty}\ln(\lambda)\cdot (F(\lambda)-F(0))=0;\\
	\int^{a}_{0+}\ln(\lambda)\mass{F}=-\int_{0}^{a}\lambda^{-1}\cdot (F(\lambda)-F(0))\mass{\lambda}+\ln(a)\cdot (F(a)-F(0))
	\end{gather*}
\end{Lemma}
\begin{proof}
	The first part comes directly from integration by parts. Assume either $\int_{0}^{1}\frac{1}{\lambda}\cdot F(\lambda)\mass{\lambda}<\infty$ or $\int_{0+}^a \ln(\lambda)\mass{F}>\infty$, then apply Monotone Convergence Theorem to get:
	\begin{equation*}
	\int_{0^+}^{a}\ln(\lambda)\mass{F}\geq-\int_{0}^{a}\lambda^{-1}\cdot (F(\lambda)-F(0))\mass{\lambda}
	\end{equation*}
	To see the other direction, we assume $\int_{0+}^1 \ln(\lambda)\mass{F}>-\infty$. Suppose to the contrary \begin{equation}
	\lim_{\lambda\to \infty}\ln(\lambda)(F(\lambda)-F(0))\neq 0
	\end{equation} 
	Then we can find $C<0$ with a sequence of $1>\lambda_1>\cdots \to 0$ such that: $\ln(\lambda_i)(F(\lambda_i)-F(0))\leq C$. Since $F(\lambda_i)\to F(0)$ as $i\to \infty$, we may then assume, by possibly passing to a subsequence, that $F(\lambda_{i+1})-F(0)\leq 1/2(F(\lambda_i)-F(0))$. 
	\par Consequently for any $n\in \nat$ and any $\lambda\in (0, 1)$, one has $\ln(\lambda)\leq \sum_{i=1}^n\ln(\lambda_i)\cdot \chi_{(\lambda_i, \lambda_{i+1}]}(\lambda)$ holds, then use this inequality to derive $\int_{0+}^1\ln(F)\mass{F}\leq n\cdot \frac{C}{2}$. Now since $C$ is arbitrarily chosen, we have $\int^1_{0+}\ln(F)\mass{F}=-\infty$. 
\end{proof}
\begin{Lemma}\label{determinant class}
	Suppose that $\theta_F(t)<\infty$ for all $t>0$. Then:
	\begin{equation}
	\int_{0^+}^1\ln(\lambda)\mass{F(\lambda)}>-\infty \qquad \iff \qquad \int_{1}^{\infty}t^{-1}\cdot (\theta_F(t)-F(0))<\infty
	\end{equation}
\end{Lemma}
\begin{proof}
	Again it suffices to assume without loss of generality that $F(0)=0$. Now \ref{3.140} implies:
\begin{equation*}
\int_1^\infty t^{-1}\cdot \theta_F(t)\mass{t}=\int_1^\infty\Big(\int_0^\infty e^{-t\lambda}\cdot F(\lambda)\mass{\lambda}\Big)\mass{t}
\end{equation*}
Now by the assumption $\theta_F(t)$ is finite for all $t$ and \Cref{3.139}, we have $F(\lambda)\leq C(1)\cdot e^{\lambda}$ for all $\lambda$, so indeed $\int_1^\infty\Big(\int_1^\infty e^{-t\lambda}\cdot F(\lambda)\mass{\lambda}\Big)\mass{t}<\infty$, and it suffices to prove the rest part is finite:
\begin{equation*}
\begin{split}
\int_1^\infty\Big(\int_0^1 e^{-t\lambda}\cdot F(\lambda)\mass{\lambda}\Big)\mass{t}&=\lim_{K\to \infty}\int_1^K\Big(\int_0^1 e^{-t\lambda}\cdot F(\lambda)\mass{\lambda}\Big)\mass{t}\\
&=\lim_{K\to \infty} \int_1^K\Bigg(\frac{d}{dt}\int_0^1\frac{e^{-t\lambda}}{-\lambda}\cdot F(\lambda)\mass{\lambda}\Bigg)\mass{t}\\
&=\lim_{K\to \infty}\Big(\int_{0}^{1}\frac{e^{-K\lambda}}{-\lambda}\cdot F(\lambda)\mass{\lambda}\Big)-\int_{0}^{1}\frac{e^{-\lambda}}{-\lambda}\cdot F(\lambda)\mass{\lambda}\\
&=\int_{0}^{1}\frac{e^{-\lambda}}{\lambda}\cdot F(\lambda)\mass{\lambda}
\end{split}
\end{equation*}
where the first equality follows from Monotone Convergence Theorem, and the third from Dominated Convergence Theorem. Now the last term is bounded by:
\begin{equation*}
\int_{0}^{1}\frac{e^{-1}}{\lambda}\cdot F(\lambda)\mass{\lambda}\leq\int_{0}^{1}\frac{e^{-\lambda}}{\lambda}\cdot F(\lambda)\mass{\lambda}\leq \int_{0}^{1}\frac{1}{\lambda}\cdot F(\lambda)\mass{\lambda}
\end{equation*}
Hence $\int_1^\infty t^{-1}\cdot \theta_F(t)\mass{t}<\infty$ if and only if $\int_{0}^{1}\frac{1}{\lambda}\cdot F(\lambda)\mass{\lambda}<\infty$. Now the lemma follows from \Cref{3.15eq}.
\end{proof}
We are now prepared to define the analytic $L^2$-torsion of $M$:
\begin{Def}
	If $M$ is of analytic determinant class, we then define the \textbf{analytic $L^2$-torsion} of $M$ to be:
	\begin{equation}\label{3.131}
	\begin{split}
	\rho^{(2)}_{\mathrm{an}}(M):=&\frac{1}{2}\cdot \sum_{p\geq 0}(-1)^p\cdot p\cdot \Bigg(\frac{d}{ds}\frac{1}{\Gamma(s)}\int_{0}^{\epsilon}t^{s-1}\cdot \left(\theta_p(t)-b_p^{(2)}(M)\right)\mass{t}\Big|_{s=0}\\
	&+\int_{\epsilon}^{\infty}t^{-1}\cdot \left(\theta_p(t)-b_p^{(2)}(M)\right)\mass{t}\Bigg)
	\end{split}
	\end{equation}
\end{Def}
\begin{Rmk}
One should note $\frac{1}{\Gamma(s)}\int_{0}^{\epsilon}t^{s-1}\cdot \left(\theta_p(t)-b_p^{(2)}(M)\right)\mass{t}\Big|_{s=0}$ is holomorphic on $\{s\in \complex \mid \Re(s)>\dim(M)/2\}$ and it admits meromorphic extension to $\complex$ with no poles on $0$. This can be proved by comparing the small $T$ behaviour of von Neumann trace of heat kernel of $M$ and the ordinary trace of heat kernel of $G \backslash M$, and then by appealing to the classical theory of extension of Zeta function. For the details, refer to \cite[Lemma~3]{lott1992}.
\end{Rmk}
\begin{Rmk}
	It may seems \ref{3.131} depends on the choice of $\epsilon$. But in fact the value is independent of $\epsilon$. To see so, choose $\delta\geq \epsilon>0$, and abbreviate $\theta^\perp_p(t)=\theta_p(t)-b_p^{(2)}(M)$:
	\begin{equation*}
	\begin{split}
	\frac{d}{ds}\frac{1}{\Gamma(s)}\int_{\epsilon}^{\delta}t^{s-1}\cdot \theta_p^\perp(t)\mass{t}\Big|_{s=0}&=\frac{d}{ds}\frac{s}{\Gamma(s+1)}\int_{\epsilon}^{\delta}t^{s-1}\cdot \theta_p^\perp(t)\mass{t}\Big|_{s=0}\\
	&=\frac{d}{ds}s\Big|_{s=0}\cdot \frac{1}{\Gamma(s+1)}\int_{\epsilon}^{\delta}t^{s-1}\cdot \theta_p^\perp(t)\mass{t}\Big|_{s=0}+0\\
	&=\int_{\epsilon}^{\delta}t^{-1}\cdot \theta_p^\perp(t)\mass{t}
	\end{split}
	\end{equation*}
since $\Gamma(1)=0!=1$.
\end{Rmk}
\section{Topological $L^2$-Invariants: A Concise Introduction}
 For the integrity of discussion of the upcoming session we will briefly discuss the topological notions here, whichever we deem as indispensable. Many interesting properties and ramifications of the topological aspects of $L^2$-invariants are omitted here. For an encyclopedic view the reader is referred to \cite{luck2013}. 
\par We begin with the general discussion of what a $L^2$-(co)homology is. Throughout this section $X$ will be a $G$-CW complex of finite type, unless otherwise stated.
\begin{Def}
	Define the \textbf{cellular $L^2$-chain complex} and the \textbf{cellular $L^2$-cochain complex} of a free $G$-CW complex $X$ to be:
\begin{align*}
C^{(2)}_*(X)&:=\ell^2(G)\otimes_{\integer G} C_*(X)\\
C_{(2)}^*(X)&:=\Hom_{\integer G}(C_*(X), \ell^2(G))
\end{align*}
where $C_*(X)$ is the cellular $\integer G$-chain complex, with $d^*_{(2)}$ and $d_*^{(2)}$ are the map induced by differential and the $L^2$-(co)chain level.
\end{Def}
Now since $X$ is of finite type, we can then fix a finite $G$-equivariant cellular basis for $C_n(X)$, from which we obtain explicit $G$-isometric isomorphisms:
\begin{equation}\label{1.29}
C^{(2)}_n(X)\cong C^n_{(2)}(X)\cong \bigoplus^k_{i=1} \ell^2(G)\cong \complex^k\otimes_\complex \ell^2(G)
\end{equation}
which indeed gives a structure of finitely generated Hilbert $\vna(G)$-module structure on both $L^2$-(co)chain complexes. One can also readily check $d^p_{(2)}$ and $d_p^{(2)}$ are $G$-equivariant.
\par To see $d^*_{(2)}$ and $d_*^{(2)}$  indeed give morphisms between Hilbert $\vna(G)$-modules it suffices to verify the boundedness. First note $\bigoplus_{i=1}^k \ell^2(G)\cong \complex^n\otimes \ell^2(G)$ is furnished with the tensor product norm. This is resolved more generally by the following lemma:
	\begin{Lemma}
		Let $\phi: (\complex G)^n\to (\complex G)^m$ be a morphism of $\complex G$-modules. Then the induced operator:
		\begin{equation*}
		\tilde{\phi}:=\phi\otimes_{\complex G}\ell^2(G): (\ell^2(G))^n\to (\ell^2(G))^m
		\end{equation*}
		is a bounded operator.
	\end{Lemma}
\begin{proof}
	Write $\phi$ as a $m\times n$-matrix $[\phi_{ij}]$ with $\phi_{ij}=\sum_{g\in G}c_{ij}(g)g$, where $c_{ij}(g)\in \complex$. Consider the $\ell^1$-norm of $\phi$ in the sense of \Cref{ell2fct}, i.e.: $\norm{\phi_{ij}}_{\ell^1}=\sum_{g\in G}|c_{ij}(g)|$ and then for $(f_1, \cdots f_n)\in (\ell^2(G))^n$, by Cauchy-Schwarz inequality, we have:
	\begin{equation*}
	\norm{\tilde{\phi}(f_1, \cdots, f_n)}_{\ell^2}^2=\sum_j\norm{\sum_i f_i\phi_{ij}}^2_{\ell^2}\leq \sum_{i,j}\norm{\phi_{ij}}^2_{\ell^1}\norm{f_i}^2_{\ell^2}\leq (\sum_{i,j}\norm{\phi_{ij}}^2_{\ell^1})\norm{(f_1, \cdots, f_n)}^2_{\ell^2}
	\end{equation*}
	Hence $\tilde{\phi}$ is a bounded operator. 
\end{proof}
\begin{Rmk}\label{ell2fct2}
	Analogous to \Cref{ell2fct} we can also identify $C^{(2)}_n(X)$ with the Hilbert space of summable chains:
	\begin{equation}
	\{\sum_{\sigma\in I_n}f(\sigma)\sigma\mid f(\sigma)\in \complex, \sum_{\sigma\in I_n} |f(\sigma)|^2<\infty\}
	\end{equation}
	where $\{\sigma\}_{\sigma\in I_n}$ form an orthonormal basis with $I_n$ the set of $n$-cell of $X$. Similarly, we can take:
	\begin{equation}
	C^n_{(2)}(X)=\{\phi:C_i(X)\to \complex\mid \sum_{\sigma\in I_n}|\phi(\sigma)|^2<\infty\}
	\end{equation}
	From this we see $C^n_{(2)}(X)=\Hom_{cont}(C_n^{(2)}(X), \complex)$. Now from the Hilbert space structure of $C_n^{(2)}(X)$ and Riesz representation theorem \cite[Chapter~I, Theorem~3.4]{conway2013}, we have the duality:
	\begin{equation}
	\Lambda: C^{(2)}_n(X)\to C^n_{(2)}(X)\qquad h\mapsto \brac{h,-}
	\end{equation}
	where $\brac{-,-}$ is the inner product of $C^{(2)}_n(X)$. Hence we see $d_p^{(2)}$ is the dual map of $d^{p+1}_{(2)}$ in the topological sense. In fact, they are also adjoint of each others as bounded operators, if we identify $C^{(2)}_*(X)$ with $C_{(2)}^*(X)$ using the isomorphism $\Lambda$: Given $x\in C^{(2)}_p, y\in C^{(2)}_{p-1}$, we have:
\begin{equation}
\Lambda((d_p^{(2)})^*(y))(x)=\brac{(d_p^{(2)})^*(y),x}=\brac{y,(d_p^{(2)})(x)}=\Lambda(y)(d_p^{(2)}(x))=(d^{p-1}_{(2)}(\Lambda(y))(x)
\end{equation}
where $d^*$ is the adjoint operator of $d$. Hence from now onwards when speaking of Hilbert chain complexes we do not differ the dual of differential map from its adjoint.
\end{Rmk}
When passing from $L^2$-cochain complex to homology, one should caution that the image of a bounded linear operator between Hilbert spaces may not be a closed subspace, thus when passing to homology, as did in singular chain complex the Hilbert space may be sacrificed. To remedy this, we define the quotient by the closure of its image:
\begin{Def}
	Given $X$ a free $G$-CW complex of finite type. Define its \textbf{(reduced) $p$-th $L^2$-(co)homology} to be:
	\begin{equation}
	\begin{split}
	H_p^{(2)}(X; \vna(G))&:=\ker (d_p^{(2)}:C_p^{(2)}\to C_{p-1}^{(2)})/\bar{\im(d_{p+1}^{(2)}:C_{p+1}^{(2)}\to C_{p}^{(2)})}\\
	H^p_{(2)}(X; \vna(G))&:=\ker (d^p_{(2)}:C^p_{(2)}\to C^{p+1}_{(2)})/\bar{\im(d^{p-1}_{(2)}:C^{p-1}_{(2)}\to C^{p}_{(2)})}
	\end{split}	
	\end{equation}
	Moreover, we define the \textbf{$p$-th Laplace operator} associated to the space $X$ to be:
	\begin{equation}
	\Delta_p:=d^{(2)}_{p+1}(d^{(2)}_{p+1})^*+(d^{(2)}_{p})^*d^{(2)}_{p}:C^{(2)}_p\to C^{(2)}_p
	\end{equation}
	with $d^*$ the adjoint of $d$ as before.
\end{Def}
One ought to expect from \Cref{ell2fct2} that the $L^2$-homology should be $G$-isometrically isomorphic to the $L^2$-cohomology in a canonical sense. It become clearer by the following lemma:
\begin{Lemma}\label{1.18}
	The $L^2$-chain complex admits an orthogonal decomposition of Hilbert $\vna(G)$-modules:
	\begin{equation}
	C^{(2)}_p(X)=\ker(\Delta_p)\oplus \bar{\im(d^{(2)}_{p+1})}\oplus \bar{\im((d^{(2)}_p)^*)}
	\end{equation}
	with the natural map:
	\begin{equation}
	\ker (d_p^{(2)})\cap \ker ((d^{(2)}_{p+1})^*)=\ker(\Delta_p)\to H_p^{(2)}(X; \vna(G))
	\end{equation}
	is an isometric $G$-isomorphism.
\end{Lemma}
\begin{proof}
	First note $H_p^{(2)}(X)$ is isometrically $G$-isomorphic to $\ker(d^{(2)}_p)\cap \im(d_{p+1}^{(2)})^\perp$ and $\ker(d_p^{(2)})^\perp=\bar{\im((d_p^{(2)})^*)}$ and $\im(d_{p+1}^{(2)})^\perp=\ker((d_{p+1}^{(2)})^*)$, whence they constitute an orthogonal decomposition of $C^{(2)}(X)$. It remains to show the first equality. This follows directly from the following equality:
	\begin{equation*}
	\brac{\Delta_p(v), v}=\norm{d_p^{(2)}(v)}^2+\norm{(d_{p+1}^{(2)})^*(v)}^2
	\end{equation*}
\end{proof}
Now the duality $\Lambda_*$ in \Cref{ell2fct2} gives the desired isomorphism between $H^*_{(2)}(X)$ and $H_*^{(2)}(X)$ since the $p$-th Laplace operator in $L^2$-cochain complex is the same with that in $L^2$-chain complex. So we are entitled to define $p$-th Betti number of $X$ without specifying the cohomology or homology:
\begin{Def}\label{1.30}
	Let $X$ be a free $G$-CW complex of finite type. Define its \textbf{topological $L^2$-Betti number} as $b_p^{(2)}(X;\vna(G)):=\dim_{\vna(G)}(C^{(2)}_*(X))$.
\end{Def}
We now set off to define the topological version of Novikov-Shubin invariants from \Cref{2.1} and \Cref{2.8}:
\begin{Def}\label{2.54}
	Let $X$ be a free $G$-CW complex of finite type. Define its \textbf{topological $p$-th spectral density function} of $d_p$ and of $\Delta_p$ respectively as:
	\begin{equation}
	F_p(X)\equiv F_p(C^{(2)}_*(X)):=F\Big(d_p|_{\im(d^{(2)}_{p+1})^\perp}:\im(d^{(2)}_{p+1})^\perp \to C_{p-1}\Big)  \qquad  F^\Delta_p(X):=F(\Delta_p)
	\end{equation}
	Respectively we define the \textbf{$p$-th Novikov-Shubin invariant} of $X$ to be:
	\begin{equation}
	\alpha_p(X):=\alpha(F_p(X)) \qquad \alpha_p^\Delta(X):=\alpha(\Delta_p)
	\end{equation}
\end{Def}
\begin{Rmk}
	Note for general Hilbert chain complex the Novikov-Shubin invariant can be defined in the same fashion, but one requires every differential map to be Fredholm in the sense of \Cref{2.8}. However, since in our case every $C_p^{(2)}(X)$ have finite von-Neumann dimension, we omit the part on Fredholmness.
\end{Rmk}
Lastly we define the topological $L^2$-torsion. This has imposed with a stronger restriction to the space than $L^2$-Betti numbers and Novikov-Shubin invariant, that is, we require the $G$-CW complex further to be finite, i.e., $X$ is of finite type and $C^{(2)}_p(X)=0$ for all $p\geq N$ for some positive integer $N$. Moreover, we requires $X$ to be of det-$L^2$-acyclic, in the following sense:
\begin{Def}
	A finite $G$-CW complex is said to be \textbf{det-$L^2$-acyclic} if each of its differential map $d_p^{(2)}$ is of determinant class, and the $L^2$-chain complex is weakly acyclic, i.e., $H_p^*(X)$ vanishes for all $p$.
\end{Def}
\begin{Def}
	If $X$ is a det-$L^2$-acyclic finite free $G$-CW complex , we define its \textbf{cellular $L^2$-torsion} to be:
	\begin{equation}
	\rho^{(2)}(X):=-\sum_{p\in \integer}(-1)^p\cdot \ln(\det(d_{p}^{(2)}))
	\end{equation}
\end{Def}
We will end this section by citing a few theorems with regard to Hilbert $\vna(G)$-chain complexes. These shall be of use in our later discussion. Due to limit in volume we omit some of the proofs here, all of which could be retrieved from \cite[Chapter~I \& II]{luck2013}.
\par First Recall a Hilbert $\vna(G)$-chain complex is Fredholm if all its differentials are Fredholm in the sense of \Cref{2.8}. We say a sequence $C_*$ of Hilbert spaces is \textbf{weakly exact} if $\ker (d_p:C_p\to C_{p-1})=\bar{\im(d_{p+1}:C_{p+1}\to C_p)}$.
\begin{Theo}\label{1.21}
	If the following is an exact sequence of Fredholm Hilbert $\vna(G)$-chain complexes:
	\begin{equation*}
	\begin{tikzcd}
	0 \rar &C_*\rar{i_*} &D_*\rar{p_*} & E_*\rar &0
	\end{tikzcd}
	\end{equation*}
	Then it induces a weakly exact long homology sequence:
	\begin{equation}
	\begin{tikzcd}
	\cdots \rar{H^{(2)}_{n+1}(p_*)} &H^{(2)}_{n+1}(E_*)\rar{\partial_{n+1}}&H_n^{(2)}(C_*) \rar{H_n^{(2)}(i_*)} &H_n^{(2)}(D_*)\rar{H^{(2)}_n(p_*)}&H_n^{(2)}(E_*)\rar{\partial_n}&\cdots
	\end{tikzcd}
	\end{equation}
\end{Theo}
\begin{proof}
	See \cite[Theorem~1.21]{luck2013}. The proof resembles that of ordinary long homology exact sequence, except that in order to prove the weak-exactness, we prove certain space $V$ vanishes using the injectivity of dimension function (c.f. \Cref{1.12}).  To prove that we use the `outer regularity' of dimension function and construct a sequence of spaces which upper-bounds $V$ and converge to $0$ via spectral projections. 
\end{proof}
Next we prove that the Novikov-Shubin invariant as an invariance of chain homotopy equivalence:
\begin{Prop}\label{2.19}
If $f:C_*\to D_*$ is a chain homotopy equivalence of Hilbert $\vna(G)$-chain complexes, the for all $p\in \integer$ we have:
\begin{equation*}
F_p(C_*)\simeq F_p(D_*)
\end{equation*}
In particular $C_*$ is Fredholm if and only if $D_*$ is Fredholm. In this case,
\begin{equation*}
\alpha_p(C_*)=\alpha_p(D_*)
\end{equation*}
\end{Prop}
\begin{proof}
First recall \Cref{1.6} that every exact sequence of Hilbert $\vna(G)$-module splits. We want to extend it to the direct sum to a chain level:
	\par Given a short exact sequence of chain complexes of Hilbert $\vna(G)$-modules:
	\begin{equation}
	\begin{tikzcd}
	0\rar &C_*\rar{j_*} &D_* \rar{q_*} &E_* \rar &0
	\end{tikzcd}
	\end{equation}
	with $E_*$ contractible. Then choose a chain contraction $\epsilon_*$ for $E_*$ and for each $p$ a morphism $t_p:E_p\to D_p$ such that $q_p\circ t_p=\id_{E_*}$, we put:
	\begin{equation*}
	s_p=d_{p+1}\circ t_{p+1}\circ \epsilon_p+t_p\circ \epsilon_{p-1}\circ e_p
	\end{equation*}
	and this $s_*$ defines a left split, i.e., $q_*\circ s_*=\id_{C_*}$.  Hence $j_*\oplus s_*$ gives the desired chain isomorphism. 
	\par Apply this construction to the following short exact sequences:
	\begin{equation*}
	\begin{tikzcd}[row sep=tiny]
	0\rar &C_*\rar &\cyl_*(f_*) \rar &\cone_*(f_*) \rar &0\\
	0\rar &D_*\rar &\cyl_*(f_*) \rar &\cone_*(C_*) \rar &0
	\end{tikzcd}
	\end{equation*}
	and we see $C_*\oplus \cone_*(f_*)$ is chain isomorphic to $D_*\oplus \cone_*(C_*)$. 
	\par Now we claim $F_p(C_*)\simeq F_p(D_*)$ for all $p$. To see this, we note for general contractible $E^*$ with chain contraction $\epsilon_*$, first note $e_p$ and $\epsilon_p$ induces invertible morphisms between $\im(e_{p+1})^\perp$ and $E_{p-1}$. Hence we see $\alpha_p(E_*)=\infty^+$ and Now we have:
	\begin{equation*}
	F_p(C_*)+F_p(\cone_*(f_*))=F_p(C_*\oplus \cone_*(f_*))=F_p(D_*\oplus \cone_*(C_*))=F_p(D_*)+ F_p(\cone_*(C_*))
	\end{equation*}
	with the second component of both sides remain constant in a neighbourhood of $0$ by the general discussion above. Hence we have $F_p(C_*)\simeq F_p(D_*)$ and the second statement follows from \Cref{2.11}.
\end{proof}
The case of $L^2$-torsion of a chain complex is more subtle, and since it is not weak homotopy invariance, as was the other two. Indeed the difference is detected by the $L^2$-torsion of map between $L^2$-cohomology, as revealed by the following theorem:
\begin{Theo}\label{3.35(5)}
	Let $C_*$ and $D_*$ be dim-finite Hilbert $\vna(G)$-chain complexes of determinant class and $f_*:C_*\to D_*$ be a weak homotopy equivalence. Then $f_*$ is of determinant class if and only if $H_p^{(2)}(f_*)$ is of determinant class for all $p\in \integer$, and:
	\begin{equation}
	\rho^{(2)}(\cone_*(f_*))=\rho^{(2)}(D_*)-\rho^{(2)}(C_*)+\sum_{p\in \integer}(-1)^p\cdot \ln(\det(H_p^{(2)}(f_*)))
	\end{equation}
\end{Theo}
\begin{proof}
	See \cite[Theorem~3.35(5)]{luck2013} for related statements and \cite[Section~3.3.3]{luck2013} for a proof.
\end{proof}
\begin{Rmk}
	The reader can easily observe that when we replace weak homology equivalence by homology equivalence, then the assumption in $f_*$ being of determinant class can be dropped. This is because $H_p^{(2)}(f_*)$ is then isomorphisms between Hilbert spaces and is hence bounded from below. Hence by a common argument we see the spectrum is disjoint from $0$, and is hence of determinant class.
\end{Rmk}
We also show here the $L^2$-torsion of a contractible space is independent of the choice of contraction one choose:
\begin{Lemma}\label{3.41}
	Let $C_*$ be a contractible dim-finite Hilbert $\vna(G)$-chain complex of determinant class. Let $\gamma_*$ and $\delta_*$  be two chain contractions. Then the maps $(c+\gamma)_{\odd}$ and $(c+\gamma)_\even$ are weak isomorphisms of determinant class with:
	\begin{equation}
	\ln(\det((c+\gamma)_\odd))=-\ln(\det((c+\delta)_\even))\qquad 	\rho^{(2)}(C_*)=\ln(\det(c+\gamma)_{\odd})
	\end{equation}
\end{Lemma}
\begin{proof}
	First note by the fact that $\gamma_*$ is a contraction, we have:
	$(c+\gamma)_\even\circ (c+\gamma)_\odd=c_\even\circ c_\odd+c_\even\circ \gamma_\odd+\gamma_{\even}\circ c_\odd+\gamma_{\even}\circ \gamma_{\odd}=0+\id_{C_{\odd}}+(\gamma^2)_\odd$. which takes the following form:
	\begin{equation*}
	\begin{pmatrix}
	\ddots & \vdots &\vdots &\vdots &\ddots\\ \cdots &1 &0 &0 &\cdots\\ \cdots & \gamma^2 & 1 &0 &\cdots \\ \cdots &0 &\gamma^2  &1 &\cdots \\ \ddots & \vdots & \vdots & \vdots & \ddots
	\end{pmatrix}:C_\odd\to C_\odd
	\end{equation*}
is hence an isomorphism. Consequently $c+\gamma$ is isomorphism for both even and odd part. Moreover, by \Cref{3.14} we note all three maps are of determinant class. Next setting $\Theta_*=\id_{C_*}+(\delta\circ \gamma)_*$, observe the following composition:
\begin{equation*}
\begin{tikzcd}[column sep=huge]
C_\odd \rar{(c+\gamma)_\odd}&C_\even \rar{\Theta} &C_\even\rar{(c+\delta)_\even} &C_\odd
\end{tikzcd}
\end{equation*}
being a lower triangular matrix with diagonal being identity. Hence we have from \Cref{3.14} the first equality.
\par To prove the second equality. We first make the observation that when $C_*$ is contractible, then $\Delta_p$ is invertible for all $p$. To see this, first note since $H^{(2)}_p(C_*)$ vanishes, we have by \Cref{1.18} that $\ker(\Delta_p)=0$ for all $p$. In particular, $\Delta_p$ is injective  for all $p$. Hence the chain contraction shows $\im(d_{p+1})=\ker (d_p)$ which is a closed subspace, hence $d_p$ has closed image for all $p$. Together this implies $\Delta_p$ is invertible, in particular, it is of determinant class in our case.
\par Next we need to modify $\rho^{(2)}$ in the forms of $\Delta_p$. We claim the following identity:
\begin{equation}\label{3.30}
\rho^{(2)}(C_*)=-\frac{1}{2}\sum_{p\in \integer}(-1)^p\cdot p\cdot \ln(\det(\Delta_p))
\end{equation}
To prove the claim, first note with respect to the orthogonal decomposition we can write 
\begin{equation*}
\Delta_p=0\oplus ((c^\perp_p)^*c_p^\perp)\oplus ((c_{p+1}^\perp)^*c_{p+1}^\perp)
\end{equation*}
 Now by \Cref{3.14}, we have:
 \begin{equation*}
 \det(\Delta_p)=\det(0)\cdot\det((c^\perp_p)^*c_p^\perp)\cdot\det(((c_{p+1}^\perp)^*c_{p+1}^\perp))=\det(c_p)^2\cdot \det(c_{p+1})^2
 \end{equation*}
 Now the claim readily follows by the applying $\ln$ to each component. 
 \par Lastly we note for all $k\in \integer$, $\Delta^k_p\circ c_{p+1}=c_{p+1}\circ \Delta^k_{p+1}$, and $\Delta_p^{k}\circ(\Delta_p^{-1}\circ c_{p}^*)=(\Delta_p^{-1}\circ c_{p}^*)\circ \Delta_{p-1}^k$. To take degree into consideration, and denote $\bigoplus_{p\text{ odd}}(\Delta_p)^p$ and $\bigoplus_{p\text{ even}}(\Delta_p)^p$ to be $\Delta_{\odd}$ and $\Delta_{\even}$ respectively, and:
 \begin{equation*}
 \Delta_\odd\circ(c+\Delta^{-1}\circ c^*)_\even= (c^*+\Delta^{-1}\circ c)_\even\circ \Delta_\even=((c+\Delta^{-1}\circ c^*)_\odd)^*\circ \Delta_{\even}
 \end{equation*}
 So now via \ref{3.30} we write $2\cdot \rho^{(2)}(C_*)=\sum_{p \text{ odd}}\ln(\det(\Delta_p^p))-\sum_{p \text{ even}}\ln(\det(\Delta^p_p))=\ln(\det(\Delta_{\odd}))-\ln(\det(\Delta_{\even}))$, then use the equality above, the first formula of this lemma, and \Cref{3.14} one yield:
 \begin{equation*}
 \begin{split}
 \ln(\det((c+\Delta^{-1}\circ c^*)_\odd))&=-\ln(\det((c+\Delta^{-1}\circ c^*)_\even))\\
 &=-\ln(\det(\Delta_\odd))-\ln(\det((c+\Delta^{-1}\circ c^*)_\even))+\ln(\det(\Delta_\odd))\\
 &=-\ln(\det(((c+\Delta^{-1}\circ c^*)_\odd)^*))-\ln(\det(\Delta_{\even}))+\ln(\det(\Delta_\odd))\\
 &=-\ln(\det((c+\Delta^{-1}\circ c^*)_\odd))+2\rho^{(2)}(C_*)
 \end{split}
 \end{equation*}
 Hence we have the desired formula.
\end{proof}
\begin{Rmk}
	Note \Cref{3.41} can be extended to more general cases of weak-acyclic dim-finite Hilbert $\vna(G)$-modules. In such case one need to replace chain contractions by weak chain contraction, while the general idea can be carried forth with a bit extra work. For details, see \cite[Section~3.3.2]{luck2013}.
\end{Rmk}
\section{Equivalence between Topological and Analytic $L^2$-Invariants}
This section is offered as a gargantuan black-box in which we quote all relative theorems that bridge the analytic $L^2$-invariants with their topological counterparts. Before stating these theorems, we still need to transfer the above settings to the case when $M=X$ being a cocompact free proper $G$-manifold without boundary and with $G$-invariant Riemannian metric. This entails an equivariant smooth triangulation, hence the $L^2$-invariants may depend on the choice of triangulation and may subject to the perturbation of $G$-homotopy. Fortunately $L^2$-Betti numbers and Novikov-Shubin invariants are $G$-homotopy invariances (In fact they are even invariants of weak homotopy equivalences):
\begin{Theo}[Weak Homotopy Invariance]\label{homopty invariance}
	Let $f:X\to Y$ be a $G$-equivariant map of free $G$-CW complexes of finite type. If the map induced on homology with complex coefficients $f_*:H_p(X; \complex)\to H_p(Y; \complex)$ is bijective $p\leq d-1$, then for $p\leq d$:
	\begin{equation}
	 F_p(X)\simeq F_p(Y) \qquad \alpha_p(X)=\alpha_p(Y)
	\end{equation}
	Moreover, if $f_p$ is surjective for $p=d$, then:
	\begin{equation}
	\forall p<d,\qquad b_p^{(2)}(X)=b_p^{(2)}(Y); \qquad b_d^{(2)}(X)\geq b_d^{(2)}(Y)
	\end{equation}
	In particular, if $f$ is a weak homotopy equivalence, then for all $p\geq 0$:
	\begin{equation}
	b_p^{(2)}(X)=b_p^{(2)}(Y) \qquad F_p(X)\simeq F_p(Y) \qquad \alpha_p(X)=\alpha_p(Y)
	\end{equation}
\end{Theo}
\begin{proof}
	First note we may assume $f$ to be cellular by \Cref{tom2.2.1}. Recall the exact sequence \ref{1.33}, which we denote $C_*(X), \cyl_*(C_*(f; \complex))$ and $\cone_*(C_*(f; \complex))$ as $C_*, D_*, E_*$ respectively.
	\par To prove the statement regarding Betti numbers, we see first $E_*$ is free (whence projective) $\complex G$-chain complex and the homology with complex coefficients $H_p(E_*;\complex)$ vanishes for $p\leq d$, it is $\complex G$-chain homotopy equivalent to a free $\complex G$-chain complex of finite type $E'_*$ with $E'_p$ is trivial for $p\leq d$ (See \cite[Chapter~I, Corollary 7.7]{brown2012} for proof). Consequently by tensoring with $\ell^2(G)$, we have the $E^{(2)}_*$ is chain homotopy equivalent to a Hilbert $\vna(G)$-chain complex $E'^{(2)}_*$  with $H_p^{(2)}(E'^{(2)}_*)=0$ for all $p\leq d$. Now the part regarding Betti number follows from \Cref{1.21}.
	\par Next we prove the part regarding Novikov-Shubin invariants. Note the canonical inclusion $C_*(Y)\to \cyl_*(C_*(f; \complex))$ is a $\complex G$-chain homotopy equivalence, and hence induces a chain homotopy equivalence of Hilbert $\vna(G)$-chain complexes of finite type $C_*^{(2)}(Y)\to \ell^2(G)\otimes_{\complex G}D_*$. Hence in view of \Cref{2.19} it suffices to establish the following equivalence:
	\begin{equation*}
	F_p(\ell^2\otimes_{\complex G} C_*)\simeq F_p(\ell^2(G)\otimes_{\complex G} D_*) \qquad \forall p\leq d
	\end{equation*}
	where $p<d$-cases are straight forward. So it suffices to build the case with $p=d$. Since $H_p(E_*)=0$ for $p\leq d-1$, we may consider the following truncated $\complex G$-exact sequence:
	\begin{equation*}
	0\to P\to E_d\to E_{d-1}\to \cdots \to E_0\to 0
	\end{equation*}
	with $P:=\ker (e_d:E_d\to E_{d-1})$. Since each $E_i$ are finitely generated free $\complex G$-module, we claim $P$ is a direct summand in $E_d$, and we may find finitely generated free $\complex G$-module $F, F'$ such that $F=P\oplus F'$.\footnote{To see the claim is true, first note $P$ is a projective $\complex G$-module. Next by an easy induction from right, we see one can indeed make both $F$ and $F'$ finitely generated free.}
	\par Now denote $d[W]_*$to be the $\complex G$-chain complex concentrated in dimension $d$ with $W$. By truncating everything in dimension great or equal to $d+1$, we yield a commutative diagram of $\complex G$-chain complexes with exact rows:
	\begin{equation}
	\begin{tikzcd}[column sep=large]
	0 \rar &C_* \rar{i_*} &D_*\oplus d[F']_*\rar{p_*\oplus \id_{d[F']_*}} & E_*\oplus d[F']_*\rar &0\\
	0 \rar &C_*\rar{i_*'} \uar[equal] &D'\uar[hook]{j_*} \rar{p_*'} &d[F]_* \rar \uar[hook]{k_*} &0
	\end{tikzcd}
	\end{equation}
	where the vertical maps are inclusions, and $D'_*$ is the submodule of $D_*\oplus d[F']_*$ that makes the diagram commutes. Now since $\id_{C_*}$ and $k_*$ are $\complex G$-homotopy equivalences, so is $j_*$ a $\complex G$-homotopy equivalence as well. Hence for $p\leq d$, we have:
	\begin{equation*}
	F_p(\ell^2(G)\otimes_{\complex G}D'_*)\simeq F_p(\ell^2(G)\otimes_{\complex G}(D_*\oplus d[F']_*))\simeq F_p(\ell^2(G)\otimes_{\complex G}D_*)
	\end{equation*}
	Hence $ C_*\oplus d[F]_* \cong D'_*$. Next we need to identify two chain complexes via a $\complex G$-chain isomorphism $g_*$ and a $\complex G$-map $u:F\to C_{d-1}$ such that the following diagram commutes with vertical maps are $\complex G$-module isomorphisms:
	\begin{equation}
	\begin{tikzcd}[column sep= large]
		C_{d}\oplus F\dar{g_d} \rar{c_d\oplus u} &C_{d-1} \dar{i'_{d-1}} \rar{c_{d-1}} &C_{d-2} \dar{i'_{d-2}} \rar{c_{d-2}} &\cdots\\
		D'_{d}\rar{d'_d} &D'_{d-1}\rar{d'_{d-1}} &D'_{d-2} \rar{d'_{d-2}}\rar &\cdots		
	\end{tikzcd}
	\end{equation}
that is, $g_p=i'_p$ for $p<d$. Now $c_{d-1}\circ u=0$ by the exactness of upper chain complex, so by the projectivity of $F$, we can solve the extension problem by a $\complex G$-map $v:F\to C_{d}$ such that the follow diagram commutes:
\begin{equation}
\begin{tikzcd}[column sep= large]
&F\ar[dl, dotted, "v"]\dar{u}\ar{dr}{0}\\ C_d \rar[swap]{c_d} &C_{d-1} \rar[swap]{c_{d-1}} &C_{d-2}
\end{tikzcd}
\end{equation}
passing to Hilbert $\vna(G)$-modules, we have the following map:
\begin{equation}
\begin{tikzcd}[column sep=huge]
(C_d\oplus F)^{(2)}\ar[rr, bend right, "(c_d\oplus u)^{(2)}"] \rar[two heads]{(\id_{C_d}\oplus v)^{(2)}} &C_d^{(2)} \rar{c_d^{(2)}} &C_{d-1}^{(2)}
\end{tikzcd}
\end{equation}
Now from \ref{2.11[9]} we have:
\begin{equation*}
F((c_d\oplus u)^{(2)})\simeq F(c_{d}^{(2)}\circ(\id_{C_d}\oplus v)^{(2)}) \simeq F(c_d^{(2)})
\end{equation*}
so the statement regarding spectral density function is proved. Now the invariance of Novikov-Shubin invariant follows readily from \Cref{2.11}.
\end{proof}
Lastly we want to investigate the $L^2$-torsion, which we have forewarned is not a $G$-homotopy invariance. Instead it will produce a residual constant in $\real$ from Whitehead torsion (c.f. \Cref{3.1.2}):
\begin{Def}
	Given any discrete group $G$, we define the following map:
	\begin{equation*}
	\Phi^G:\Wh(G)\to \real
	\end{equation*}
	by sending an element $A\in M_n(\integer G)$ to $\int_{0^+}^\infty \ln(\lambda) \mass{F(R_A)}$ with $R_A:\ell^2(G)^n\to \ell^2(G)^n$ the right multiplication by $A$. Note this map is well-defined on $\Wh(G)$ by the properties of Kadison-Fuglede determinant listed in \Cref{3.15} and \Cref{3.14}.
\end{Def}
\begin{Theo}\label{3.93(1)}
	Let $f:X\to Y$ be a $G$-homotopy equivalence of finite free $G$-CW complexes. Suppose $X$ or $Y$ is det-$L^2$-acyclic, then so is the other, and:
	\begin{equation}
	\rho^{(2)}(Y)-\rho^{(2)}(X)=\Phi^G(\tau(f))
	\end{equation}
	where $\tau(f)$ is Whitehead torsion we defined in \Cref{3.1.2}
\end{Theo}
\begin{proof}
The det-$L^2$-acyclicity is a direct consequence of \Cref{3.35(5)}. Now use the formula in \Cref{3.35(5)}, we note:
\begin{equation*}
\rho^{(2)}(\cone_*(f_*))=\rho^{(2)}(C_*^{(2)}(Y))-\rho^{(2)}(C_*^{(2)}(X))+\sum_{p\in \integer}(-1)^p\cdot \ln(\det(H_p^{(2)}(f_*)))
\end{equation*}
where left hand side is by \Cref{3.41} is $\ln(\det(c+\gamma)_\odd)$ for some chain contraction $\gamma_*$. Now by choose the representation of $\tau(f)\in \Wh(G)$ to be $(c+\gamma)_\odd$ again, we see the proof completes once we show that $\sum_{p\in \integer}(-1)^p\ln(\det H^{(2)}_p(f_*))=0$, but this follows from the weak-acyclicity of $C_*$ or $D_*$. 
\end{proof}
Now we are entitled to define all topological $L^2$-invariants on manifolds. 
\begin{Def}
	An \textbf{$G$-equivariant smooth triangulation} $K$ of $M$ consists of a simplicial complex $K$ with simplicial $G$-action such that for each open simplex $\sigma$ and $g\in G$ with $g\sigma\cap \sigma\neq \emptyset$, then $g$ induces identity on $\sigma$. Take $|K|$ as the geometric realization, then there is a $G$-homeomorphism $f:|K|\to M$. We then define the \textbf{topological $L^2$-Betti number} and \textbf{Novikov-Shubin invariant} of a cocompact free proper $G$-manifold $M$ to be that of any of its equivariant triangulation.
\end{Def}
Now The core of relating $L^2$-integrable harmonic smooth $p$-forms \ref{1.55} with the topological $L^2$-cohomology is the following $L^2$-version of Hodge-de Rham Theorem, due to Dodziuk \cite{dodziuk1977}:
\begin{Theo}[$L^2$-Hodge-de Rham Theorem]
	\label{l2hdr} Let $M$ be a cocompact free proper $G$-manifold with $G$-invariant Riemannian metric and let $K$ be an equivariant smooth triangulation of $M$. Suppose that $M$ has no boundary. Then the integration defines an isomorphism of finitely generated Hilbert $\mathcal{N}(G)$-modules:
	\begin{align*}
	\mathcal{H}_{(2)}^p(M)\overset{\cong}{\longrightarrow} H^p_{(2)}(K)
	\end{align*}
\end{Theo}
As a immediate corollary we see the cellular $L^2$-Betti number in \Cref{1.30} equals to the analytic $L^2$-Betti number in \Cref{1.60} via the following simple computation:
\begin{equation*}
\dim_{\vna(G)}(\mathcal{H}_{(2)}^p(M))=\dim_{\vna(G)}(\ker((\Delta_p)_{\min}))=F((\Delta_p)_{\min})(0)=\lim_{t\to \infty}\theta_p(M)(t)
\end{equation*}
where the last equality is direct from \Cref{3.138} and Dominated Convergence Theorem.
\par Next with regard to the Novikov-Shubin invariants, we have the following result due to Efremov \cite{efremov1991}:
\begin{Theo}\label{2.68}
	Let $M$ be a cocompact free proper $G$-manifold with $G$-invariant Riemannian metric and let $K$ be an equivariant smooth triangulation of $M$. Suppose that $M$ has no boundary. Then for the cellular spectral density function $F_p(K)$ in \Cref{2.54} and the analytic spectral density function $F_p(M)$ in \Cref{2.64}, we have for each dimension $p$, 
	\begin{equation*}
	F_p(K)\simeq F_p(M)
	\end{equation*}
	Consequently, we have $\alpha_p(M)=\alpha_p(K)$.
\end{Theo}
Lastly we define the topological $L^2$-torsion. First note in \cite[Theorem~III]{illman2000} Illman shows that for general Lie group $G$,  each cocompact free proper $G$-manifold $M$ has a unique simple $G$-homotopy type, that is for any two equivariant smooth triangulation $f: K\to M$ and $g:L\to M$, we have the Whitehead torsion of $g^{-1}\circ f: K\to L$ to be a simple $G$-homotopy equivalence. Hence in the light of \Cref{3.93(1)} the $L^2$-torsion is well-defined, once we have chosen a smooth triangulation. 
\par Nonetheless in the assumption of \Cref{3.93(1)} one requires the weak-acyclicity of the simplicial complex, which is a rather strong assumption for manifolds. Nonetheless if one examine the proof of \Cref{l2hdr}, there is an isomorphism $A_K^p:\mathcal{H}^p_{(2)}(M)\to H^p_{(2)}(K)$ (see \cite[Lemma~1.76ff]{luck2013} for definition), and if we take this into consideration and use \Cref{3.35(5)}, one might remove this assumption:
\begin{Def}\label{3.120}
	Let $M$ be a cocompact free proper $G$-manifold without boundary and with a $G$-invariant Riemannian metric. Let $f:K\to M$ be an equivariant smooth triangulation. We define $M$ is of \textbf{determinant class} if any of the equivariant smooth triangulation (hence all) is of determinant class. Hence $\rho^{(2)}:=\rho^{(2)}(C_*^{(2)}(K))$ is defined. Let $A_K^p:\mathcal{H}^p_{(2)}(M)\to H^p_{(2)}(K)$ be the $L^2$-Hodge de Rham isomorphism, we then define \textbf{topological $L^2$-torsion} of $M$ to be:
	\begin{equation*}
	\rho^{(2)}_{\mathrm{top}}(M)=\rho^{(2)}(K)-\sum_{p \geq 0}(-1)^p\cdot \ln\Bigg(\det\big(A_K^p:\mathcal{H}^p_{(2)}(M)\to H^p_{(2)}(K)\big)\Bigg)
	\end{equation*}
\end{Def}
Next we want to check the definition is independent of choice of equivariant smooth triangulations. Choose $f:K\to M$ and $g:L\to M$ to be two such triangulations, we have $A_K^p\circ H^p_{(2)}(g^{-1}\circ f)=A_L^p$. Now by \Cref{3.14}, we have $\ln(\det(A^p_K))=\ln(\det(A_L^p))+\ln\bigg(\det (H^p_{(2)}(g^{-1}\circ f))\bigg)$, and consequently by \Cref{3.35(5)} and the fact $g^{-1}\circ f$ is a simple $G$-homotopy equivalence, one has:
\begin{equation*}
\rho^{(2)}(L)-\rho^{(2)}(K)=-\sum_{p \geq 0}(-1)^p\cdot \ln\Bigg(\det (H_p^{(2)}(g^{-1}\circ f))\Bigg)
\end{equation*}
Next observe from the discussion after \Cref{ell2fct} that the map between finitely generated Hilbert cochain complexes can be identified with that adjoint between their respective Hilbert chain complexes. Hence we have $\ln\bigg(\det (H^p_{(2)}(g^{-1}\circ f))\bigg)=\ln\bigg(\det (H_p^{(2)}(g^{-1}\circ f))\bigg)$. Summing up all this results, we have:
\begin{equation*}
\rho^{(2)}(L)-\rho^{(2)}(K)=\sum_{p \geq 0}(-1)^p\bigg(\ln(\det(A^p_L))-\ln(\det(A^p_k))\bigg)
\end{equation*}
Hence we have proved the topological $L^2$-torsion defined above is well-defined. Note $\rho^{(2)}_{\mathrm{top}}(M)$ is dependent on the choice of Riemannian metric, as captured by the Hilbert $\vna(G)$-structure of $\mathcal{H}^p_{(2)}(M)$.
\par Lastly the analytic $L^2$-torsion and the topological $L^2$-torsion agrees by the following deep result due to Burghelea, Friedlander, Kappeler and McDonald \cite{burghelea1996}:
\begin{Theo}\label{3.149}
	Let $M$ be a cocompact free proper $G$-manifold without boundary and with a $G$-invariant Riemannian metric. Then  $M$ is of analytic determinant class in the sense of \Cref{3.128} if and only if it is of determinant class in the sense of \Cref{3.120}. In such case, the analytic $L^2$-torsion $\rho^{(2)}_{\mathrm{an}}(M)$ as defined in \Cref{3.131} is the same as the topological $L^2$-torsion $\rho^{(2)}_{\mathrm{top}}(M)$ as defined in \Cref{3.120}.
\end{Theo}
Having established the topological $L^2$-invariants of manifolds, we conclude this section with some immediate consequences:
\begin{Theo}[Poincar\'e Duality]\label{poincare}
	Let $M$ be a cocompact free proper $G$-manifold without boundary of dimension $n$ which is orientable. Then:
	\begin{equation}
	b^{(2)}_p(M)=b^{(2)}_{n-p}(M) \quad F_p(M)\simeq F_{n+1-p}(M) \quad \alpha_p(M)=\alpha_{n+1-p}(M)
	\end{equation}
	Furthermore if $n$ is even, then $\rho^{(2)}(M)=0$.
\end{Theo}
\begin{proof}
	First note there is a subgroup $G_0$ of $G$ (of index 1 or 2) which acts orientation preserving on $M$. By applying the restriction functor, we have $b^{(2)}_p(M; \vna(G_0))=[G:G_0]b^{(2)}(M; \vna(G))$. So we can assume without loss of generality that $G\backslash M$ is orientable. Now \cite[Theorem~2.1]{wall1999} gives the Poincar\'e $\integer G$-chain homotopy equivalence:
	\begin{equation}
	\frown[G\backslash M]: C^{n-*}(M)\to C_*(M)
	\end{equation}
	which, after applying $\otimes_{\integer G}\ell^2(G)$, induces a homotopy equivalence of finitely generated Hilbert $\vna(G)$-chain complexes $C^{n-*}_{(2)}(M)\to C^{(2)}(M)$. Now the claim of $L^2$-Betti number follows from \Cref{homopty invariance} and the remark prior to \Cref{1.30}. The statement of Novikov-Shubin invariants now follow suit again by \Cref{homopty invariance} and $F_p(C_*)=F_p((C_*)^*)$.
	\par To preserve the statement of $L^2$-torsion,first choose a smooth triangulation $f:K \to M$, and let $[G\backslash K]:=((G\backslash f^{-1})_*:H_n(G\backslash M)\to H_n(G\backslash K))([G\backslash M])$. Now \cite[Theorem~2.1]{wall1999} also asserts $\tau(\frown[G\backslash K])=0$ with respect the cellular basis. In particular, together with the isomorphism $\Lambda$ we construct in \Cref{ell2fct2} it induces an homotopy equivalence fo finite Hilbert $\vna(G)$-chain complexes:
	\begin{equation*}
	g_*:\ell^2(G)\otimes_{\integer G} C^{n-*}(K)\to C_*^{(2)}(K)
	\end{equation*}
	with $\rho^{(2)}(\cone_*(g_*))=0$. Now by \Cref{3.35(5)} we see:
	\begin{equation}
	\rho^{(2)}(\ell^2(G)\otimes_{\integer G} C^{n-*}(K))=\rho^{(2)}(C_*^{(2)}(K))
	\end{equation}
	Now apply the error term $-\sum_{p \geq 0}(-1)^p\cdot \ln\Bigg(\det\big(A_K^p:\mathcal{H}^p_{(2)}(M)\to H^p_{(2)}(K)\big)\Bigg)$ in \Cref{3.120} to both sides, further note $\det(f)=\det(f^*)$ and a shift in degree when dualizing the chain complex, we have: $\rho^{(2)}(M)=(-1)^{n+1}\rho^{(2)}(M)$. In particular, when $n$ is even, the $L^2$-torsion vanishes and we have finished the proof.
\end{proof}

\section{Proof of $L^2$-Hodge de Rham Theorem}\label{proof of l2hdr}
In this section we shall prove \Cref{l2hdr} and \Cref{2.68} in an unified approach. Proof of \Cref{3.149} will require more technical tools and is way more complicated, for which reason we shall omit here. We shall follow the notation of previous section.
\par The proof of \nameref{l2hdr} will process very much like the compact case. The general steps is as follows. First we begin with constructing appropriate $L^2$-chain complexes on which the differential maps are bounded operators. Between this chain complex and $C_{(2)}^*(K)$ one then construct inverse maps which induces bijective map on $L^2$-cohomology. The later step takes some step as our construction is not global as the compact case. We begin ourselves by understanding harmonic $p$-forms as $L^2$-cohomology.
\par First recall the \textbf{Sobolev $k$-norm} of $p$-forms,  
\begin{equation}
\norm{\omega}_k:=\norm{(1+\Delta_p)^{k/2}\omega}_{L^2}=\brac{\omega, (1+\Delta_p)^k\omega}_{L^2}
\end{equation}
Consequently, we can define the \textbf{$k$-th Sobolev spaces} of $p$-forms on $M$ as the completion of $\Omega^p_c(M)$ with respect to $\norm{\cdot}$. Now we see moreover it is by \Cref{langa1} an elliptic operator which admits unique closed extension, whence we could identify it with the following space:
\begin{equation*}
H^{k}\Omega^p(M):=\{\omega\in L^2\Omega^p(M)\mid (1+\Delta_p)^{k/2}\omega \in L^2\Omega^p(M)\}
\end{equation*}
where $(1+\Delta_p)^{k/2}$ is defined using functional calculus and $(1+\Delta_p)^k\omega$ is defined in the sense of distribution. Note the $G$-action on $M$ gives an $\vna(G)$-structure on $L^2\Omega^p(M)$. To see this, one choose a fundamental domain $\mathcal{F}\subset M$ of $G$, which gives the following isomorphism:
\begin{equation}
L^2\Omega^p(M)\cong \ell^2(G)\otimes L^2\Omega^p(\mathcal{F})\cong \ell^2(G)\otimes L^2\Omega^p(G\backslash M)
\end{equation}
where $G$ acts on $\ell^2(G)$ via left-regular representation, and on $L^2\Omega^p(M)$ trivially. Consequently, we have a Hilbert $\vna(G)$-module structure on $L^2\Omega^p(M)$, and since $G$-action commutes with $(1+\Delta_p)^k$, we deduce $H^k\Omega^p(M)$ are Hilbert $\vna(G)$-modules for all $k,p\geq0$. Now since $d^p$ is a linear differential operator of order $1$ and hence can be extended to a bounded linear operator $d^p:H^{k+1}\Omega^p(M)\to H^k\Omega^{p+1}(M)$ for all $k,p\geq 0$. Consequently, we have the following Hilbert $\vna(G)$-cochain complex:
\begin{equation}
\begin{tikzcd}
\cdots \rar &H^l\Omega^0(M)\rar{d^0} &H^{l-1}\Omega^1(M)\rar{d^2} &\cdots \rar{d^{n-1}} &H^{l-n}\Omega^n(M)\rar &0\rar &\cdots
\end{tikzcd}
\end{equation}
Moreover, since $\Delta_p$ vanish on $\mathcal{H}_{(2)}^p(M)$, we have canonical inclusion $\iota:\mathcal{H}^p_{(2)}(M)\inj H^{l-p}\Omega(M)$ for all $l\geq p$. The following lemma identified it with the cohomology:
\begin{Lemma}\label{1.75}
	Let $l\geq \dim(M)$. Then $\iota$ is a $G$-equivariant isometric embedding and induces a $G$-equivariant isometric isomorphism:
	\begin{equation}
	\mathcal{H}^p_{2}(M)\longrightarrow H^p_{(2)}(H^{l-*}\Omega^*(M); d^p)
	\end{equation}
\end{Lemma}
\begin{proof}
	Let $\omega\in \mathcal{H}^p_{(2)}(M)$. One observe $\norm{\omega}_k=\norm{\omega}_0<\infty$ for all $k\geq 0$. Hence $d\omega, \delta\omega, d\delta\omega, \delta d\omega\in L^2(M)$. Also since $M$ is complete Riemannian manifold, we can apply $L^2$-Stokes theorem:
	\begin{equation}
	\brac{\omega, \omega}_{L^2}=\brac{(1+\Delta_p)\omega, \omega}_{L^2}=\norm{\omega}_{L^2}^2+\norm{d^p\omega}_{L^2}^2+\norm{\delta^p\omega}_{L^2}^2
	\end{equation}
	forcing $d\omega=\delta\omega=0$. Next let $\eta\in H^{l-p+1}\Omega^{p-1}(M)$, one has:
	\begin{equation*}
	\brac{\omega, d^{p-1}\eta}_{l-p}=\brac{\omega, (1+\Delta_p)^{l-p}d^{l-p}\eta}_{L^2}=\brac{\delta^p\omega, (1+\Delta_{p-1})^{l-p}\eta}_{L^2}=0
	\end{equation*}
	Hence $\mathcal{H}^p_{(2)}(M)\perp \im(d^{p-1})$ and we have proved $\mathcal{H}^p_{2}(M)\subseteq H^p_{(2)}(H^{l-*}\Omega^*(M); d^p)$
	\par To prove the other side, let $\mu\in \ker(d^p)$. Then $\mu=\omega+\eta\in \mathcal{H}^p_{(2)}(M)\oplus (\mathcal{H}^p_{(2)}(M))^{\perp_{\ker d^p}}\subseteq H^{l-p}\Omega(M)$. Now for any $\nu\in \mathcal{H}^p_{(2)}(M)$ we get:
	\begin{equation*}
	\brac{(1+\Delta_p)^{\frac{l-p}{2}}\eta, \nu}_{L^2}=\brac{(1+\Delta_p)^{\frac{l-p}{2}}\eta, (1+\Delta_p)^{\frac{l-p}{2}}\nu}_{L^2}=\brac{\eta, \nu}_{l-p}=0
	\end{equation*} 
	Recall the decomposition in \Cref{1.18}. We see $(1+\Delta_p)^{\frac{l-p}{2}}\eta\in \bar{\im d^{p-1}}$. But now observe $(1+\Delta_p)^{k/2}$ defines an $G$-equivariant isometric isomorphism $H^l\Omega^p(M)\cong H^{l-k}\Omega^p(M)$.This can be observed by applying functional calculus to its (unique) self-adjoint extension. Consequently, since $d^{p-1}\Delta_p=\Delta_pd^{p-1}$, we have from \nameref{normalspec} that $d^{p-1}$ commutes with $(1+\Delta_p)^{(p-l)/2}$ as well. Hence $\eta\in \bar{\im(d^{p-1})}$ and the other side is proved.
\end{proof}
On the other hand if we fix an equivariant smooth triangulation $K$ of $M$. Throughout the discussion we shall not distinguish the simplex $\sigma$ with its geometric realization $|\sigma|$. We have a cocompact $G$-CW complex. recall in \Cref{ell2fct2} we have identified $C^*_{(2)}(K)$ with $\ell^2C^*(K)$. Choose $\omega\in H^k\Omega^p(M)$, we see $\omega\in C^1(M)$ by Sobolev embedding theorem. Hence we can integrate $\omega$ over any oriented $p$-simplex of $K$, whence obtaining an element in $\ell^2C^p(K)$. Hence we can choose a large enough $l>0$, and define the following map via integration:
\begin{equation}
A^*:H^{l-*}\Omega^*(M)\to C^*_{(2)}(K)
\end{equation}
Next we define the right inverse of $A^*$ as follows. Let $\{U_\sigma\}_{\sigma\in S_0(K)}$ be the open covering given by open stars of $0$-simplices. Note $g\cdot U_\sigma=U_{g\sigma}$. Next choose a $G$-invariant partition of unity $\{e_\sigma\}$ subordinate to $U_\sigma$, that is $e_\sigma\in C^\infty(M, [0,1])$ such that $e_{g\sigma}\circ L_g=e_\sigma$, and $\sum_{\sigma\in S_0(K)}^{}e_\sigma=1$. 
\par Now given a $p$-simplex $\tau$ with vertices $\sigma_0, \cdot,\sigma_p$. For the associated characteristic function $\chi_{\tau}$, we define a $p$-form with support in the star of $\tau$ by:
\begin{equation*}
W(\chi_{\tau}):=p!\sum_{i=0}^{p}(-1)^ie_{\sigma_i}d^0e_{\sigma_0}\wedge \cdots \wedge d^0e_{\sigma_0{i-1}}\wedge d^0e_{\sigma_0{i+1}}\wedge \cdots \wedge d^0e_{\sigma_p}
\end{equation*}
By default we define $W(e_{\sigma})=e_{\sigma}$. This now gives an cochain map of Hilbert $\vna(G)$-cochain complexes $W^*:C^*_{(2)}(K)\to H^{l-*}\Omega^*(M)$. The reader is readily to check $A^*\circ W^*=\id$, and both maps does not depend on the choice of orientations of the simplices. In particular, we see at $L^2$-cohomology level,
\begin{equation*}
H^*_{(2)}(A^*):H^*_{(2)}(H^{l-*}\Omega^*(M))\to H^*_{(2)}(C^*_{(2)}(K))
\end{equation*}
is surjective. 
\par In order to prove $H^*_{(2)}(A^*)$ is injective, we cannot construct a homotopy between $W^*\circ A^*$ and $\id$ as we did in the compact case, since the construction on the quotients is not local, whence can not be lifted to $K$ and $M$ directly. To remedy this we define a new $\tilde{W}^*$, which we replace the smooth partition of unity $e_\sigma$ by the barycentric coordinate function $e_\sigma$, we see the barycentric coordinate function $e_\sigma$'s are only non-smooth on the $\dim(M)-1$-skeleta. Hence the $p$-forms are defined in the sense of distributions. We claim it is a continuous map and image are square integrable $p$-forms:
\begin{Lemma}\label{2.79}
Given an equivariant smooth triangulation $K$, the map 
\begin{equation*}
\tilde{W}^p_K:C^p_{(2)}(K)\to L^2\Omega^p(M)
\end{equation*}
is a bounded from below.
\end{Lemma}
\begin{proof}
We first prove the map is well-defined i.e., $\tilde{W}^*_K(C^p_{(2)}(M))\subseteq L^2\Omega^p(M)$. Observe $\tilde{W}(\chi_{\tau})$ is continuous in a distributional sense. To see so, given $\tau$ a $p$-dimensional face on both $\dim(M)$-simplices $\sigma_0$ and $\sigma_1$, then for inclusion $i_k:\tau\to \sigma_k$ for $k=0,1$, we have $i^*_0\tilde{W}(u)|_{\sigma_0}=i^*_1\tilde{W}(u)|_{\sigma_1}$. Hence we have the continuity. From the definition we see $\norm{\chi_{\tau}}_{L^2}$ are uniformly bounded for all $p$-simplices $\tau$, hence $\tilde{W}^p(u)\in L^2\Omega^p(M)$ and the map is well-defined.
 \par Next recall $G$ acts freely and cocompactly. Hence for any $p$-simplex $\sigma$, we can choose $D>0$ and $S>0$, such that:
 \begin{align*}
 \int_{\sigma}^{}\norm{\tilde{W}(\chi_{\sigma})}^2&\mass{\vol_\sigma}\geq 2\cdot D\\
\quad |\{\tau\in S_*(K)\mid \tau\in st(\sigma)\}|\leq S&\qquad  |\{\tau\in S_*(K)\mid \sigma\in st(\tau)\}|\leq S 
 \end{align*}
 where $st(\sigma)$ is the closed star of $\tau$. For $p$-simplex $\sigma$ we choose a neighbourhood $U(\sigma)$ of $\int \sigma$ open in $K$ such that $\bar{U_{\sigma}}$ as a face and we choose such neighbourhoods equivariantly, i.e.:
 \begin{equation*}
  g\cdot U(\sigma)=U(g\sigma) \qquad U(\sigma)\cap U(\tau)=\emptyset \quad \text{for }\sigma\neq \tau
 \end{equation*}
Since $W(\chi_\tau)$ is supported in $st(\tau)$, we have $W(\chi_{\tau})(x)=0$ for $x\in U(\sigma)\backslash st(\tau)$.By possibly shrinking $U(\sigma)$ to a smaller neighbourhood we may find a number $\delta>0$, such that up to a small error we can identify $U(\sigma)$ with $\int(\sigma)\times (-\frac{\delta}{D}, \frac{\delta}{D})$, and we then have:
\begin{equation}\label{110}
\int_{U(\sigma)}\norm{\tilde{W}^p(\chi_\tau)}_x^2\mass{\vol_x}\begin{cases}
\geq \delta \quad &\text{if }\tau=\sigma\\
\leq \frac{\delta}{4S-2)S}\quad  &\text{if }\tau\neq \sigma
\end{cases}
\end{equation}
So now given $u=\sum_{\sigma}^{}u_\sigma\cdot \chi_{\sigma}$, we see:
\begin{equation*}
\sum_{\sigma}^{}|u_\sigma|^2\leq \frac{1}{\delta}\int_{U(\sigma)}\norm{u_\sigma\cdot \tilde{W}(\chi_\tau)}_x^2\mass{\vol}\leq\frac{1}{\delta}\sum_{\sigma}^{}\int_{U(\sigma)}\Big(\norm{\sum_{\tau}u_\tau \tilde{W}(\chi_{\tau})}_x^2+\norm{\sum_{\tau\neq \sigma}^{}u_\tau \tilde{W}(\chi_{\tau})}_x^2\Big)\mass{\vol}
\end{equation*}
So now we study two summands separately by the bounds we get above. First by $U(\tau)$ are pairwise disjoint, we have:
\begin{equation*}
\sum_{\sigma}^{}\int_{U(\sigma)}\norm{\sum_{\tau}u_\tau \tilde{W}(\chi_{\tau})}_x^2\mass{\vol}\leq \int_M\norm{\sum_{\tau}u_\tau \tilde{W}(\chi_{\tau})}^2_x\mass{\vol}\leq \norm{\tilde{W}(u)}_{L^2}
\end{equation*}
On the other hand, we note by the bound in \ref{110} and $\norm{\sum_{i=1}^ra_i}\leq (2r-1)\sum_{i=1}^r\norm{a_i}^2$, that:
\begin{equation*}
\begin{split}
\int_{U(\sigma)}\norm{\sum_{\substack{\tau\neq \sigma}}u_\tau \tilde{W}(\chi_{\tau})}_x^2&=\int_{U(\sigma)}\norm{\sum_{\substack{\tau\neq \sigma\\ \sigma\in st(\tau)}}u_\tau \tilde{W}(\chi_{\tau})}_x^2\mass{\vol}\\
&\leq (2S-1)\cdot\int_{U(\sigma)}\sum_{\substack{\tau\neq \sigma\\ \sigma\in st(\tau)}}\norm{u_\tau \tilde{W}(\chi_{\tau})}_x^2\mass{\vol}\\
&\leq (2S-1)\cdot \sum_{\substack{\tau\neq \sigma\\ \sigma\in st(\tau)}}|u_\tau|^2\int_{U(\sigma)}\norm{\tilde{W}(\chi_{\tau})}_x^2\mass{\vol}\\
&\leq (2S-1)S\cdot \sum_{\tau}|u_\tau|^2\cdot\frac{\delta}{(4S-2)S}\leq \frac{\delta}{2}\cdot\norm{u}^2_{L^2}
\end{split}
\end{equation*}
Summing up, we have $\norm{u}_{L^2}\leq \frac{2}{\delta}\norm{\tilde{W}(u)}_{L^2}$, hence the map is bounded from below.
\end{proof}
Now the advantage of using barycentric coordinate functions is that we can choose perform barycentric divisions on $K$, and use such to construct a sequence of $\{W^p_K\circ A^p_K\}_K$ which approximates $\id$ in operator norm. Recall \textbf{mesh} and \text{fullness} of a triangulation $K$ are defined as:
\begin{align*}
\mathrm{mesh}(K):=\sup\{d(p,q)\mid \text{$p,q$ vertices of 1-simplex}\}\\
\mathrm{full}(K):=\inf\{\frac{\vol(\sigma)}{\dim(M)\mathrm{mesh}(K)}\mid \sigma\in S_{\dim(M)}(K)\}
\end{align*}
which measured the `density' and `convexity' of triangulation respectively. Then:
\begin{Lemma}\text{\cite[Lemma~3.9]{dodziuk1977}}\label{2.76}
	Fix $\theta>0$, $k>\frac{\dim M}{2}+1$ and an equivariant smooth triangulation $K$. Then there is a constant $C>0$ such that for any equivariant barycentric subdivision $K'$ of $K$ with $\mathrm{full}(K')\geq \theta$, we have:
	\begin{equation}
	\forall \omega\in H^k\Omega^p(M) \qquad \norm{\omega-\tilde{W}^p_{K'}\circ A^p_{K'}(\omega)}_0\leq C\cdot \mathrm{mesh}(K')^{\frac{\dim(M)}{2}+1}\cdot \norm{\omega}_k
	\end{equation}
\end{Lemma}
\begin{proof}[Sketch of Proof]
First choose local trivialization with respect to $\{U_\sigma\}_{\sigma\in K}$, and then choose open subsets $\{V_\tau\}_{\tau\in K'}$ (again equivariantly), such that $V_\tau\subseteq \tau$, and each point in $M$ is covered by maximally $m$-many $V_\tau$. We can get an upper bound for all $x\in \tau\in S_N(K')$:
\begin{equation}
|\omega-\tilde{W}_{K'}\circ A_{K'}(\omega)|\leq C\cdot \mathrm{diam}\tau (\norm{\omega|_{V_\tau}}_k+\norm{\omega|_{V_\tau}}_0)
\end{equation}
which we can choose $C$ independent of $\omega, \tau$ and $K'$ in view of \cite[Proposition~2.4]{dodziuk1976}. Consequently:
\begin{equation*}
\begin{split}
\norm{\omega-\tilde{W}_{K'}\circ A_{K'}\omega}_0^2&\leq \sum_{\tau\in K'}^{}\int_\tau|\omega-\tilde{W}_{K'}\circ A_{K'}(\omega)|^2\mass{\vol}\\
&\leq C^2\cdot \dim(M)^2\cdot\mathrm{mesh}(K')^{2+\dim(M)}\sum_{\tau}^{}(\norm{\omega|_{V_\tau}}_k+\norm{\omega|_{V_\tau}}_0)^2\\
&\leq 2\cdot C'\cdot \mathrm{mesh}(K')^{2+\dim(M)}\sum_{\tau}^{}((\norm{\omega|_{V_\tau}}_k)^2+(\norm{\omega|_{V_\tau}}_0)^2)\\
&\leq C''\cdot m\cdot \mathrm{mesh}(K')^{2+\dim(M)}\cdot \norm{\omega}_k^2
\end{split}
\end{equation*}
Since $m$ is universally chosen, the lemma is proved.
\end{proof}
We may choose the subdivision $K(\epsilon)$ with mesh small enough such that $\norm{\omega-\tilde{W}^p_{K(\epsilon)}\circ A^p_{K(\epsilon)}(\omega)}_0<\epsilon/2$. Now choose a representative $\omega\in \mathcal{H}^p_{(2)}(M)$ with $[\omega]\in H^*_{(2)}(A^*_K)$, we want to prove $[\omega]\in H^p_{(2)}(H^{l-*}\Omega^*(M))$ vanishes. 
\par First observe $C^*_{(2)}(K)\cong C^*_{(2)}(K')$ is compatible with differential, hence $H^*_{(2)}(A^*_{K(\epsilon)})([\omega])=0$. Now find $u\in C^{p-1}_{(2)}(K(\epsilon))$ such that 
\begin{equation*}
\norm{A^p_{K(\epsilon)}(\omega)-c^{p-1}_{K(\epsilon)}(u)}<\frac{\epsilon}{2\cdot \norm{W^p_{K(\epsilon)}}}
\end{equation*}
we have:
\begin{equation*}
\begin{split}
\norm{\omega-\tilde{W}^p_{K(\epsilon)}\circ c^{p-1}_{K(\epsilon)}(u)}_0&\leq \norm{\omega-\tilde{W}^p_{K(\epsilon)}\circ A^p_{K(\epsilon)}(\omega)}_0+\norm{\tilde{W}^p_{K(\epsilon)}\circ A^p_{K(\epsilon)}(\omega)-W^p_{K(\epsilon)}\circ c^{p-1}_{K(\epsilon)}(u)}_0<\epsilon
\end{split}
\end{equation*}
Last one checks that $d^{p-1}_{\max}\circ W^{p-1}_{K(\epsilon)}=W^p_{K(\epsilon)}\circ c^{p-1}_{K(\epsilon)}$, hence for any $\epsilon>0$, we can choose $v\in \dom d^p_{\max}$ such that $\norm{\omega-d^{p-1}_{\max}v}_0<\epsilon$. Consequently $\omega$ is trivial, and we have proved bijectivity of $H^*_{(2)}(A^*)$ and consequently \nameref{l2hdr} is proved.

\bigskip

The proof of \Cref{2.68} follows exactly the same procedure. We first prove $F_p(K)\preceq F_p(M)$:
\begin{Lemma}
	Given any equivariant smooth triangulation $K$ we have 
	\begin{equation*}
	F_p(K)\preceq F_p(H^{l-*}\Omega^*(M))\preceq F_p(M)
	\end{equation*}
\end{Lemma}
\begin{proof}
	Note surjectivity of $A^*$ has directly implied the first part. It suffices to prove $F_p(H^{l-*}\Omega^*(M))\preceq F_p(M)$. By the isometric isomorphism $(1+\Delta_p)^{k/2}$, it suffices to 
	\begin{equation}
	F(d^{p\perp}_{H^1}:(\im d^p_{H^2})^\perp \subseteq H^{1}\Omega^p(M)\to L^2\Omega^{p+1}(M))\preceq F(d^{p\perp}_{\min})
	\end{equation}
	Now arguing like \Cref{1.75} we see for $\omega\in \Omega^{p-1}_c(M)$, and $\eta\in \im(d^{p-1}_{\min})^\perp$, we have $\brac{d^{p-1}(\omega), \eta}_{L^2}=0$. Meanwhile, for $\omega\in H^1\Omega^p(M)$, one has:
	\begin{equation}
	\norm{\omega}_1^2=\brac{(1+\Delta)_p\omega, \omega}=\norm{\omega}^2_0+\norm{d^p(\omega)}^2_0+\norm{\delta^p(\omega)}_0^2
	\end{equation}
	Together we have, by restricting the inclusion $H^1\Omega^p(M)\to L^2\Omega(M)$ to $\im(d^{p-1}_{H^2})^\perp$ one gets an injective morphism $j:\im(d^{p-1}_{H^2})^\perp\to \im(d^{p-1}_{\min})^\perp$.
	\par For $0\leq \lambda<1$, recall definition $\mathcal{L}(d^{p\perp}_{H^1}, \lambda)$ in \Cref{2.1}. For $\omega\in L$,we have $\delta^p(\omega)=0$, and we have:
	\begin{equation*}
	\norm{d^p(\omega)}^2_0\leq \lambda^2\cdot \norm{\omega}^2_1\leq \lambda^2(\norm{\omega}^2_0+\norm{d^p(\omega)}^2_0
	\end{equation*}
	Hence $E^{(d^{p\perp}_{\min})^*d^{p\perp}_{\min}}_{\lambda^2/(1-\lambda^2)}\circ j$ is injective when restricted to $L$. Consequently, by \Cref{2.3}
	\begin{equation*}
		\dim_{\vna(G)}(L)\leq \dim_{\vna(G)}(\im E^{(d^{p\perp}_{\min})^*d^{p\perp}_{\min}}_{\lambda^2/(1-\lambda^2)})=F_p(M)(\frac{\lambda}{\sqrt{1-\lambda^2}})
	\end{equation*}
	Since this holds for all $L\in \mathcal{L}(d^{p\perp}_{H^1}, \lambda)$, we have $F_p(H^{l-*}\Omega^*(M)(\lambda)\leq F_p(M)(\frac{\lambda}{\sqrt{1-\lambda^2}})$ and hence proves the lemma.
\end{proof}
To finish the proof of \Cref{2.68} it suffices to prove $F_p(M)\preceq F_p(K)$. By homotopy equivalence it suffices to show the claim for one fixed smooth triangulation $K$. Also fix $\epsilon>0$. For $\omega\in \im(E^{(d^{p\perp}_{\min})^*d^{p\perp}_{\min}}_{\epsilon^2})\subseteq \im(d^{p-1}_{\min})^\perp$, note  $(\Delta_p)_{\min}|_{\im(d^{p-1}_{\min})^\perp}=(\delta^{p+1}d^p)^\perp_{\min}=(d^{p\perp}_{\min})^*d^{p\perp}_{\min}$. Moreover, $E^{(d^{p\perp}_{\min})^*d^{p\perp}_{\min}}_{\epsilon^2}(\omega)=\omega$, so by \Cref{2.3} we see:
\begin{equation}
\norm{\omega}^2_k=\brac{\omega, (1+\Delta_p)^k(\omega)}_{L^2}=\brac{\omega, (1+(d^{p\perp}_{\min})^*d^{p\perp}_{\min})^k(\omega)}_{L^2}\leq (1+\epsilon^{2k})\cdot \norm{\omega}^2_0
\end{equation}
Hence we have a bounded $G$-equivariant operator $i_\epsilon:\im(E^{(d^{p\perp}_{\min})^*d^{p\perp}_{\min}}_{\epsilon^2})\inj H^{l-p}\Omega^p(M)$. Now the following diagram commutes:
\begin{equation}
\begin{tikzcd}[row sep=small]
\im(E^{(d^{p\perp}_{\min})^*d^{p\perp}_{\min}}_{\epsilon^2})\rar[hook]{i_{\epsilon}} \dar{d^{p\perp}_{\min}} &H^{l-p}\Omega^p(M)\rar{A^p} \dar{d^p} &C^p_{(2)}(K)\dar{c^p} \rar[two heads]{\mathrm{pr}}&\im(c^{p-1})^\perp\arrow[dl, swap,"c^p"]\\
\im(E^{(d^{(p+1)\perp}_{\min})^*d^{(p+1)\perp}_{\min}}_{\epsilon^2})\rar[hook,swap]{i_\epsilon} &H^{l-p-1}\Omega^{p+1}(M)\rar[swap]{A^{p+1}} &C^{p+1}_{(2)}(K) \rar[swap]{W^{p+1}} &L^2\Omega^{p+1}(M)
\end{tikzcd}
\end{equation}
We first claim the upper composite map is bounded from below. It is a fact that we can find an equivariant subdicision whose fullness is bounded below with mesh arbitrarily small. Hence using \Cref{2.76} we can choose a constant $C_0<1$ such that for any $p$-form $\eta\in H^k\Omega^p(M)$:
\begin{equation*}
\norm{\eta-W^p\circ A^p(\eta)}_0\leq C_0\cdot \norm{\eta}_0
\end{equation*}
Denote $\mathrm{pr}':L^2\Omega^p(M)\twoheadrightarrow \im(d^{p-1}_{\min})^\perp$. Then we have $W^p(\im(c^{p-1}))\subseteq \im(d^{p-1}_{\min})$, hence $\mathrm{pr}'\circ W^p\circ \mathrm{pr}=\mathrm{pr}'\circ W^p$. For $\omega\in \im(E^{(d^{p\perp}_{\min})^*d^{p\perp}_{\min}}_{\epsilon^2})$, we have $\mathrm{pr}'(\omega)=\omega$, and:
\begin{align*}
\norm{\omega}_0&\leq \norm{\mathrm{pr}'\circ W^p\circ A^p\circ i_\epsilon(\omega)}_0+\norm{\omega-\mathrm{pr}'\circ W^p\circ A^p\circ i_{\epsilon}(\omega)}_0\\
&\leq \norm{W^p}\cdot \norm{\mathrm{pr}\circ A^p\circ i_{\epsilon}(\omega)}_0+\norm{\omega-W^p\circ A^p\circ i_\epsilon(\omega)}_0\qquad &\text{from $\mathrm{pr}'(\omega)=\omega$}\\
&\leq \norm{W^p}\cdot \norm{\mathrm{pr}\circ A^p\circ i_{\epsilon}(\omega)}_0+C_0\cdot \norm{\omega}_0\qquad &\text{from \Cref{2.76}}
\end{align*}
Hence we have proved the claim. 
\par Next we prove for all $\lambda\leq \epsilon$, and $\omega\in \im(E^{(d^{p\perp}_{\min})^*d^{p\perp}_{\min}}_{\lambda^2})$:
\begin{equation}
\norm{c^p\circ \mathrm{pr}\circ A^p\circ i_{\epsilon}(\omega)}_0\leq C_3\cdot \lambda\cdot \norm{\mathrm{pr}\circ A^p\circ i_{\epsilon}(\omega)}_0
\end{equation}
We see this is straightforward by applying the claim above, \Cref{2.79} and the above commutative diagram. Hence we conclude $\bar{\mathrm{pr}\circ A^p\circ i_{\epsilon}(\im(E^{(d^{p\perp}_{\min})^*d^{p\perp}_{\min}}_{\lambda^2}))}\in \mathcal{L}(c^p, C\lambda)$. But $\mathrm{pr}\circ A^p\circ i_{\epsilon}$ is bounded from below hence is injective, and we have proved $F_p(M)\preceq F_p(K)$ and consequently \Cref{2.68} by observing:
\begin{equation}
F_p(M)(\lambda)=\dim_{\vna(G)}(\bar{\mathrm{pr}\circ A^p\circ i_{\epsilon}(\im(E^{(d^{p\perp}_{\min})^*d^{p\perp}_{\min}}_{\lambda^2}))})\leq F_p(K)(C\cdot \lambda)
\end{equation}








\chapter{Computation of $L^2$-invariants}\label{chapter3}
In this chapter we will compute various $L^2$-invariants on various examples. The examples are arranged in a crescendo of difficulty:
\par As an hors d'\oe uvre we first discuss the flat torus, in which we observe how the analytic approach and the topological approaches unify;
\par The main course of this chapter is devoted to the computation of $L^2$-invariants of symmetric spaces. Before stating the main theorem \Cref{olbrich1.1} we have used two sections to familiarize the readers with pertinent term and theorems.
\section{$L^2$-Invariants of Flat Torus and $\integer^n$-CW complexes}
We begin the discussion with the easiest nontrivial case, namely the flat torus. Consider the flat torus to be the quotient of $\real^n$, under the action:
\begin{equation}
\integer^n \longrightarrow \mathrm{Isom}(\real^n) \qquad (a_1, \cdots, a_n)\mapsto \Big((x_1, \cdots, x_n)\mapsto (x_1+a_1, \cdots, x_n+a_n)\Big)
\end{equation}
Clearly this is free proper action, with the group acts isometrically, so we can $T^n$ furnished with the unique Riemannian metric such that  the quotient map $\pi: \real^n\to T^n$ is Riemannian covering. Consequently, the curvature tensor on $T^n$ vanishes altogether with $\int_{T^n}\mass{\vol}=\int_{[0,1]^n}\mass{x}=1$.
\par Next we inspect the Laplace operators and heat kernel. The Laplacian on functions $\Delta_0f=\sum_i\diff{^2}{x_i^2}f$. On the other hand, we see $L^2(\real^n, \Lambda^pT^*\real^n)$ can be identified with $L^2(\real^n)\otimes \Lambda^pT^*\real^n$, where the Laplacian $\Delta_p$ acts on $p$-form $\omega=f dx_{i_1}\wedge \cdots \wedge dx_{i_p}$ by $\Delta_p\omega=\sum_{j=1}^n\diff{^2f}{x_j} dx_{i_1}\wedge \cdots \wedge dx_{i_p}$, hence it suffices to investigate the heat kernel on functions. 
\par Now by appealing to Fourier transformation, we yield the heat kernel on functions and consequently on forms given by:
\begin{equation}
K_p(t, x,y)=\frac{1}{(4\pi t)^{n/2}}\exp(-\frac{\norm{x-y}^2}{4t})\otimes \id_{\real^{n \choose p}}
\end{equation}
with ${n \choose p}$ the dimension of $\Lambda^p T_x^*\real^n=\Lambda^p \real^n$. Now $\tr_\complex K_p(t,x, x)={n \choose p}\cdot \frac{1}{(4\pi t)^{n/2}}$, so consequently, for all $0\leq p\leq n$
\begin{equation}
b_p^{(2)}(\real^n, \vna(\integer^n))=\lim_{t\to \infty}\frac{1}{(4\pi t)^{n/2}}=0
\end{equation}
Next we inspect the Novikov-Shubin invariant. To compute this we use \Cref{3.136}. First observe there is no gap in spectrum around $0$, since for any $\lambda\in \real+$, we have: $\Delta e^{\sqrt{\lambda}x_i}=\lambda\cdot e^{\sqrt{\lambda}x_i}$, hence the spectrum of lambda contains $\real^+$ and in particular, $\alpha^\Delta_p(\tilde{T^n})\neq \infty^+$. Moreover, 
\begin{equation}
\theta_p(\tilde{T^n})(t):=\int_{[0,1]^n}\tr_{\complex}K_p(t,x,y)\mass{\vol}=\int_{[0,1]^n}{n \choose p}\frac{1}{(4\pi t)^{n/2}}\mass{\vol}={n \choose p}\frac{1}{(4\pi t)^{n/2}}
\end{equation}
Hence by taking $C_p=\frac{{n\choose p}}{(4\pi)^{n/2}}$, we have:
\begin{equation*}
\alpha^\Delta_p(\tilde{T^n})=\liminf_{t\to \infty}\frac{-\ln(\theta_p(\tilde{T^n})(t)-b_p^{(2)}(\tilde{T^n}))}{\ln(t)}=\liminf_{t\to \infty}\frac{\frac{n}{2}\ln(C_p\cdot t)}{\ln(t)}=\frac{n}{2}
\end{equation*}
We leave the computation of $\alpha_p(\tilde{T^n})$ to the last, and compute the analytic $L^2$-torsion first. Observe for any $0\leq p\leq n$, we have:
\begin{equation}
\int_{\epsilon}^{\infty}t^{-1}\cdot\Big(\theta_p(\tilde{T^n})(t)-b_p^{(2)}(\tilde{T^n})\Big)\mass{t}=\int_{\epsilon}^{\infty} C_p\cdot t^{-1-\frac{n}{2}}<\infty
\end{equation}
hence $\real^n$ is indeed of analytic determinant class. Also recall \cite[Chapter~6, Theorem~1.6 \& 1.7]{stein2003} that $\frac{1}{\Gamma(s)}$ is an entire function with simple zeros at $s\in \integer_{\leq 0}$. Moreover, 
\begin{equation}\label{stein6.1.7}
\frac{1}{\Gamma(s)}=e^{\gamma s}\prod_{n=1}^{\infty}(1+\frac{s}{n})e^{-s/n}
\end{equation}
Then some computation gives us $1/\Gamma(0)=0$ and $\frac{d}{ds}\frac{1}{\Gamma(s)}|_{s=0}=1$. Now use \ref{3.131} we choose $\epsilon=1$ and have
\begin{equation}
\begin{split}
\rho_{\mathrm{an}}(\tilde{T^n})&=\frac{1}{2}\sum_{p \geq 0}(-1)^p\cdot p\cdot\Bigg(\frac{d}{ds}\frac{1}{\Gamma(s)}\int^1_0C_p\cdot t^{s-1-\frac{n}{2}}\mass{t}\Big|_{s=0}+\int_{1}^{\infty}C_p\cdot t^{-1-\frac{n}{2}}\mass{t}\Bigg)\\
&=\frac{1}{2}\sum_{p \geq 0}(-1)^p\cdot p\cdot\Big(\frac{d}{ds}\frac{C_p}{\Gamma(s)\cdot (s-\frac{n}{2})}\Big|_{s=0}+\frac{2\cdot C_p}{n}\Big)\\
&=\frac{1}{2}\sum_{p \geq 0}(-1)^p\cdot p\cdot\Big(\frac{-C_p\cdot2}{n}+\frac{2\cdot C_p}{n}\Big)=0
\end{split}
\end{equation}
so we see the $L^2$-torsion vanishes for $\tilde{T^n}$ for all $n$. 
\par Lastly we want to prove $F_p(\tilde{T^n})=n$ for all $0\leq p\leq n$. We do this first by observe the following general phenomenon that $\alpha_p^{\Delta}(M)=1/2\cdot \min\{\alpha_p(M), \alpha_{p+1}(M)\}$, which with a careful calibration of domain, is a easy consequence of \Cref{2.11}. But now observe on $\real^n$ the higher differential is essentially the same as the differential on function, whence we have the same Novikov-Shubin invariants for all $p$. Hence we have $\alpha_p(\tilde{T^n})=n$ for all $0\leq p\leq n$.

\bigskip

Next we consider the topological approach. We shall begin with consider the underlying ring $\vna(\integer^n)$ self. From Fourier analysis it is readily to derive an isomorphism between $\ell^2(\integer)$ and $L^2(T)$ which sends every $f\in L^2(T^n)$ to $\hat{f}:\integer^n \to \complex$, such that:
\begin{equation}
\hat{f}(n):=(2\pi)^{-n}\int_{T^n}f(x)e^{-i\brac{k,x}}\mass{x}
\end{equation}
and conversely sending $\hat{g}\in \ell^2(\integer^n)$ to $g=\sum_{k\in \integer^n}\hat{f}(k)e^{i\brac{k, -}}$. Moreover we see this map is $\integer^n$-equivariant, if we take $T^n\in \complex^n$ the unit torus, and $(k_1, \cdots, k_n)\in \integer^n$ acts as $(z_1, \cdots, z_n)\mapsto (z_1^{k_1}, \cdots, z_n^{k_n})$. Consequently, we see $\vna(\integer^n)=\mathcal{B}(L^2(T^n))^{\integer^n}$. Invoking \Cref{conix.2.6} we have a representation  of $L^\infty(T^n)$ on $L^2(T^n)$:
\begin{equation}
L^\infty(T^n) \to \mathcal{B}(L^2(T))^{\integer^n}
\end{equation}
which sends $f\in L^\infty(T)$ to the $\integer^n$-equivariant operator $M_f:L^2(T^n)\to L^2(T^n)$ via multiplication by $f$. Under this isomorphism we have the von Neumann dimension can be realized as:
\begin{equation}
\begin{split}
\dim_{\vna(\integer^n)}(\im M_f)=\tr_{L^2(T^n)}(M_f)=\brac{f\cdot \frac{\id_{T^n}}{\sqrt{\vol(T^n)}}, \frac{\id_{T^n}}{\sqrt{\vol(T^n)}}}=\frac{1}{\vol(T^n)}\int_{T^n}f\mass{\vol}=\frac{\norm{f}_{1}}{\vol(T^n)}
\end{split}
\end{equation}
In particular, we observe $\{E^{M_{|f|}}_\lambda\}_{\lambda\geq 0}$ which is a family of projections on $L^2(T^n)$, can be realized using \Cref{conix.2.6} as:
\begin{equation}
E^{|M_f|}_\lambda= \chi_S: L^2(T^n)\to L^2(T^n) \qquad g\mapsto \chi_S\cdot g
\end{equation}
for the set $S$ corresponds to the Borel subset $\spec(M_{|f|})\cap (-\infty, \lambda]=\mathrm{essran}(|f|)\cap (-\infty, \lambda]$. Hence in our case,
\begin{equation*}
S=\{z\in T^n\mid |f(z)|\leq\lambda\}
\end{equation*}
The reader can readily verify that this indeed defines a spectral measure which satisfy the conditions stated in \nameref{normalspec}.
\par Now we move to consider $L^2$-Betti number and Novikov-Shubin invariants for general $\integer^n$-CW complexes of finite type.
\begin{Lemma}\label{1.34}
	Let $C_*$ be a free $\complex[\integer^n]$-chain complex of finite type with some basis. Denote $\complex[\integer^n]^{(0)}$ be the quotient field of $\complex [\integer^n]$. Then:
	\begin{equation}
	b_p^{(2)}(C_*^{(2)})=\dim_{\complex[\integer^n]^{(0)}}\Big(\complex[\integer^n]^{(0)}\otimes_{\complex[\integer^n]}H_p(C_*)\Big)
	\end{equation}
\end{Lemma}
\begin{proof}
	Since the Betti number only captures local data, we may assume without loss of generality that $C_*$ is finite dimensional. Abbreviate $\complex[\integer^n]^{(0)}\otimes_{\complex[\integer^n]}C_*$ as $C_*^{(0)}$, we first treat the case where $C_*^{(0)}$ has trivial homology. Since $\complex[\integer^n]^{(0)}$ is a field, we have $C_*^{(0)}$ is contractible and can then choose a $\complex[\integer^n]^{(0)}$-chain contraction $\delta_*$ for it. Now we claim there exists a $u\in \complex[\integer^n]$ such that for all $p$, $M_u\circ \delta_p=\gamma_p^{(0)}$ for some $\complex[\integer^n]$-chain map $\gamma_*:C_*\to C_{*+1}$, and $M_u$ the multiplication by $u$. To see so, we choose a $\complex[\integer^n]^{(0)}$-basis of $C^{(0)}_*$ and take $\delta_*$ as a finite-dimensional $\complex[\integer^n]^{(0)}$-valued matrix. Then take $u$ to be the product of all entries of this matrix, we have $M_u\circ \delta_*$ to be a $\complex[\integer^n]$-valued matrix, and hence can be realized as a $\complex[\integer^n]$-chain map $\gamma_*$.
\par Now $\gamma_*$ defines a $\complex[\integer^n]$-chain homotopy $M_u\simeq 0:C_*\to C_*$ since $M_u$ commutes with any chain map. This indeed induces a chain homotopy of Hilbert $\vna(G)$-chain complexes. Now by appealing to the previous discussion on structure of $\vna(\integer^n)$, we see $M_u^{(2)}$ is an injective map between $\ell^2(\integer^n)^k$ and hence all $L^2$-homology $H^{(2)}(C^{(2)}_*)$ vanishes, whence $b_p^{(2)}=0$.
\par Now we treat the general case by construct some acyclic mapping one. Denote $b_p$ to be the $\complex[\integer^n]$-dimension of  $\Big(\complex[\integer^n]^{(0)}\otimes_{\complex[\integer^n]}H_p(C_*)\Big)$.  Now since $\complex[\integer^n]^{(0)}$ is a field, we have $\complex[\integer^n]^{(0)}\otimes_{\complex[\integer^n]}H_p(C_*)$ is a $\complex[\integer^n]^{(0)}$-free module, hence by choosing appropriate basis, we have a $\complex[\integer^n]^{(0)}$-isomorphism between $\oplus^{b_p}_{i=1} \complex[\integer^n]^{(0)}\to \complex[\integer^n]^{(0)}\otimes_{\complex[\integer^n]}H_p(C_*)$. Again by the same trick as the first case, we can compose it with a suitable element such that the composite map is induced by some $\complex[\integer^n]$-map:
\begin{equation}
i_p:\oplus^{b_p}_{i=1}\complex[\integer^n]\to H_p(C_*)
\end{equation}
Now Take $D_*$ be a $\complex[\integer^n]$-chain complex with $D_p=\oplus_{i=1}^{b_p}\complex[\integer^n]$ for all $p$ and the differential are all trivial. Then we have $H(C_*)=H(D_*)$. Now consider the chain map $j_*: D_*\to C_*$ with $j_p=i_p$ for all $p$. Consider the following exact sequence of $\complex[\integer^n]$-chain complexes:
\begin{equation*}
0\longrightarrow C_*\longrightarrow \cone_*(j_*)\longrightarrow \Sigma D_*\longrightarrow 0
\end{equation*}
where $\Sigma C_*$ the suspension of $C_*$, is hence a free $\complex[\integer^n]$-module. Thus this sequence splits, and hence we may extend it to a exact sequence to $C^{(2)}_*$ or $C^{(0)}_*$. Now we see from the exact sequence that $\cone_*(j_*)$ is acyclic, and then the first case applies. Now \Cref{1.21} shows $b_p^{(2)}=b^{(2)}(C_*^{(2)})=b^{(2)}(D_*^{(2)})=b_p$.
\end{proof}
Now we proceed to prove Novikov-Shubin invariants for $\integer$-CW complexes, Since we will need the underlying ring to be a principal ideal domain, which is only enjoyed by $\complex(\integer)$ but not for $\complex(\integer^n)$ in general, hence a complete analog of \Cref{1.34} is not possible. Nonetheless, this can be remedied by studying some product formula of spectral density functions and Novikov-Shubin invariants.
\begin{Lemma}\label{2.31}
	Let $G$ and $H$ be discrete groups. Let $f:U\to U$ and $g:V\to V$ be positive maps of Hilbert $\vna(G)$-modules and Hilbert $\vna(H)$-modules respectively. Then $f\otimes \id_U+\id_V\otimes g$ defines a positive map of Hilbert $\vna(G\times H)$-modules with:
	\begin{equation}
	F(f\otimes \id_V+\id_U\otimes g)\simeq F(f)\cdot F(g)
	\end{equation}
\end{Lemma}
\begin{proof}
	It is clear from definition of von Neumann dimension that $\dim_{\vna(G\times H)}(U\otimes V)=\dim_{\vna(G)}(U)\cdot \dim_{\vna(H)}(V)$. Now for $x\in \im(E^f_{\lambda/2})=F(f)(\lambda/2)$, and $y\in (E^g_{\lambda/2})=F(g)(\lambda/2)$, we have from \Cref{2.3}
	\begin{equation*}
	\begin{split}
	\norm{(f\otimes \id_V+\id_U\otimes g)(x\otimes y)}&\leq \norm{f(x)}\cdot \norm{y}+\norm{x}\cdot \norm{g(y)}\\
	&\leq \frac{\lambda}{2}\norm{x}\norm{y}+\frac{\lambda}{2}\norm{x}\norm{y}=\lambda\norm{x\otimes y}
	\end{split}
	\end{equation*}
	Hence we have again by \Cref{2.3}, that $F(f)(\lambda/2)\cdot F(g)(\lambda/2)\leq F(f\otimes \id_V+\id_U\otimes g)(\lambda)$. 
	\par On the other hand, $f\otimes \id_V\leq f\otimes \id_V+\id_U\otimes $ implies $F(f\otimes\id_V)\geq F(f\otimes \id_V+\id_U\otimes)$ and consequently $\im(E_\lambda^{f\otimes \id_V+\id_U\otimes })\subseteq \im(E_\lambda^{f\otimes \id_V})$. By a symmetric argument, we have:
	\begin{equation*}
	\im(E_\lambda^{f\otimes \id_V+\id_U\otimes })\subseteq \im(E_\lambda^{f\otimes \id_V})\otimes \im(E_\lambda^{\id_U\otimes g})=\im(E_\lambda^f)\otimes \im(E_\lambda^g)
	\end{equation*}
	Summing up the result, we have $F(f)(\lambda)\cdot F(g)(\lambda)\geq F(f\otimes \id+\id\otimes g)(\lambda)$.
\end{proof} 
Now the $\liminf$ does not comply with additivity of two functions, so the Novikov-Shubin invariants of two functions does not necessarily adds up. This has motivates the following definition:
\begin{Def}
	A Fredholm spectral density function has the \textbf{limit property} if $F(\lambda)=F(0)$ for some $\lambda>0$, or if
	\begin{equation}
	\lim_{\lambda\to 0+}\frac{\ln(F(\lambda)-F(0))}{\ln(t)}
	\end{equation}
	exists. We say a Fredholm Hilbert chain complex $C_*$ has the limit property if $F_p(C_*)$ has the limit property for all $p\in \integer$, and a $G$-CW complex $X$ of the finite type has the limit property if the asscociated cellular $L^2$-chain complex has the limit property.
\end{Def}
Define now another auxiliary function $\delta:[0, \infty]\to \{1\}\cup \{\infty^+\}$ such that $\delta(0):=\infty^+ $
and $\delta(r)=1$ elsewhere.
\begin{Lemma}
	Let $F,G$ are two density functions which are Fredholm and has the limit property, then so is $F\cdot G$ with:
	\begin{equation}
	\alpha(F\cdot G)=\min\{\alpha(F)+\alpha(G), \delta(F(0))\cdot \alpha(G), \alpha(F)\cdot \delta(F(0))\}
	\end{equation}
\end{Lemma}
\begin{proof}
	Since $\alpha(F)=\alpha(F-F(0))$, so it suffices to treat the case $F(0)=G(0)=0$. Now the result is direct from $\ln(F\cdot G)=\ln(F)\cdot (G)$ and the limit exists, so the result readily follows from \Cref{2.11}.
\end{proof}
Now the following product formula follows from the previous lemmas as well as a careful calibration of domain. We omit the proof due to the cumbersome notations. For details of proof refer to \cite[Lemma~2.35]{luck2013}:
\begin{Theo}\label{2.44}
	Let $G, H$ be discrete groups. Let $C_*$ and $D_*$ be a Hilbert $\vna(G)$-chain complex and a Hilbert $\vna(H)$-chain complex respectively. Suppose $C_p$ and $D_p$ vanishes for $p<0$, and are both Fredholm with limit property. Then $C_*\otimes D_*$ is a Fredholm Hilbert $\vna(G\times H)$-chain complex with limit property and:
	\begin{equation}
	\alpha^\Delta_n(C_*\otimes D_*)=\min_{i=0,\cdots, n}\{\alpha^\Delta_i(C_*)+\alpha^\Delta_{n-i}(D_*), \delta(F_i(C_*))\cdot \alpha^\Delta_{n-i}(D_*), \delta(F_i(D_*))\cdot \alpha^\Delta_{n-i}(C_*)\}
	\end{equation}
	and
	\begin{multline}
	\alpha_n(C_*\otimes D_*)=\min_{i=0,\cdots, n}\{\alpha_{i+1}(C_*)+\alpha_{n-i}(D_*),
	\alpha_{i}(C_*)+\alpha_{n-i}(D_*),\\
	  \delta(F_i(C_*))\cdot \alpha_{n-i}(D_*), \delta(F_i(D_*))\cdot \alpha_{n-i}(C_*)\}
	\end{multline}
\end{Theo} 
From this we can easily derive  analogical formulae for products of $G$-CW complexes for varying $G$'s. In particular, it suffices, in view of this theorem, to compute the Novikov-Shubin invariants of $\vna(\integer)$-chain complexes.
\begin{Lemma}\label{2.58}
	Let $C_*$ be a free $\complex[\integer]$-chain complex of finite type. Since $\complex[\integer]$ is a principal ideal domain, we by the structure theorem can write:
	\begin{equation}
	H_p(C_*)=\complex[\integer]^{n_p}\oplus \Bigg( \bigoplus_{i_p=1}^{s_p} \complex[\integer]/((z-a_{p, i_p})^{r_{p, i_p}})\Bigg)
	\end{equation}
	fir $a_{p, i_p}\in \complex$, and $n_p, s_p, r_{p, i_p}\in \integer$ with $n_p, s_p\geq 0$ and $r_{p, i_p}\geq 1$, and $z$ a fixed generator of $\integer$.  Then $\ell^2(\integer)\otimes_{\complex[\integer]}C_*$ has the limit property. 
	Take $S=\{i_p=1,\cdots, s_p\mid \norm{a_{p, i_p}}=1 \}$. Then we have:
	\begin{equation}
	\alpha_{p+1}(C_*^{(2)})=\begin{cases}
	\min\{\frac{1}{r_{p, i_p}}\mid i_p\in S \} &\qquad \text{if }s_p\geq 1, S\neq \emptyset\\
	\infty^+ &\qquad \text{if otherwise}
	\end{cases}
	\end{equation}
\end{Lemma}
\begin{proof}
	The key of this lemma is to reconstruct a chain complex with nicer differential map at each degree, which induces the same homology as $C_*$.
	\par Let $P(n_p)_*$ be the free $\complex[\integer]$-chain complex concentrated in dimension $0$ with module $\complex[\integer^n]^{n_p}$. Let $Q(a_{p, i_p}, r_{p, i_p})_*$  be another free $\complex[\integer]$-chain complex concentrated in dimension $0$ and $1$ as follows:
\begin{equation*}
\begin{tikzcd}[column sep=huge]
\cdots \rar &0 \rar &\complex[\integer]\rar{M_{(z-a_{p, i_p})^{r_p, i_p}}} &\complex[\integer] \rar &0 \rar &\cdots
\end{tikzcd}
\end{equation*}
Now $H_*(D_*)$ concentrated at dimension $0$ with $\complex[\integer]/((z-a_{p, i_p})^{r_p, i_p})$. Now assembling components together, we can easily construct the following $\complex[\integer]$-map:
\begin{equation*}
f_*: \bigoplus_{p\geq 0} \Sigma^p\Bigg(P(n_p)_*\oplus \bigoplus_{i_p=1}^{s_p}Q(a_{p, i_p}, r_{p, i_p})_* \Bigg) \longrightarrow C_*
\end{equation*}
which induces an isomorphism on homology. Now since both sides are free $\complex[\integer]$-modules, we have $f$ to be a $\complex[\integer]$-chain equivalence. Now by \Cref{2.11} it suffices to prove the statement for $Q$-component, i.e., for $a\in \complex, r\in \integer, rg\geq 1$, we have:
\begin{equation}
\alpha_1(Q(a, r)_*)=\begin{cases}
\frac{1}{r} \quad &\text{if }\norm{a}=1\\
\infty^+ \quad &\text{if } \norm{a}\neq 1
\end{cases}
\end{equation}
Next from the discussion on von-Neumann algebra of $\integer^n$ we see immediately from the differential being injective that:
$F_1(Q(a,r)_*)(\lambda)=F(M_{(z-a)^r})(\lambda)=F(M_{|(z-a)^r|})(\lambda)=\vol\{z
\in S^1\mid |(z-a)^r|\leq \lambda \}$. Now clearly $F_1=0$ for $0\leq \lambda<|1-|a||^r$, and $\alpha_1=\infty^+$ for $|a|\neq 1$. Otherwise, we have:
\begin{equation}
\vol\{z \in S^1\mid |(z-a)^r|\leq \lambda \}=\vol\{\cos(\phi)+i\sin(\phi)\mid |2-2\cos(\phi)|^{r/2}\leq \lambda\}
\end{equation}
Moreover, we see $\lim_{\phi\to 0}\frac{2-2\cos(\phi)}{\phi^2}=1$, that $Q(a,r)$ has the limit property, with $\alpha_1(Q(a,r)_*)=\frac{1}{r}$. Hence the lemma is proved.
\end{proof}
Lastly we see the $L^2$-torsion of a finite free $\integer^n$-CW complex that is of det-$L^2$-acyclic. Again we follow the same mechanism as in Novikov-Shubin invariants, by procuring some product formula of $L^2$-torsion of det-$L^2$-acyclic free $G$-CW complexes for varying $G$s. Then from $L^2$-torsion of $\integer$-CW complexes we get results for spaces that are products of such $\integer$-CW complexes, so in particular has included $\tilde{T^n}$, since all its $L^2$-Betti numbers vanish.
\begin{Theo}
	\textnormal{\cite[Theorem~3.35(6)]{luck2013}} Let $f:C_*\to C_*'$ be a chain map of dim-finite Hilbert $vna(G)$-chain complexes and  $g_*: D_*\to D_*'$ a chain map of dim-finite Hilbert $vna(H)$-chain complexes. Denote $\chi^{(2)}(C_*)\in \real$ the $L^2$-Euler characteristic. Then:\begin{enumerate}[label=\Roman*]
		\item If $D_*$ is further det-$L^2$-acyclic, then so is $C_*\otimes D_*$ as dim-finite Hilbert $\vna(G\times H)$ chain complex, with:
		\begin{equation}
		\rho^{(2)}(C_*\otimes D_*)=\chi^{(2)}(C_*)\cdot \rho^{(2)}(D_*)
		\end{equation}
		\item If $C_*$ and $D_*$ are of determinant class, then so is $C_*\otimes D_*$ and:
		\begin{equation}
		\rho^{(2)}(C_*\otimes D_*)=\chi^{(2)}(C_*)\cdot  \rho^{(2)}(D_*)+\chi^{(2)}(D_*)\cdot \rho^{(2)}(C_*)
		\end{equation}
		\item If $f_*$ and $g_*$ are weak homology equivalences of determinant class, then so is $f_*\otimes g_*$  and:
		\begin{equation}
		\rho^{(2)}(\cone_*(f_*\otimes g_*))=\chi^{(2)}(C_*)\cdot \rho^{(2)}(\cone_*(g_*))+\chi^{(2)}(D_*)\cdot \rho^{(2)}(\cone_*(f_*))
		\end{equation}
	\end{enumerate}
\end{Theo}
Now the following theorem follows readily from this and the fact that $\chi^{(2)}(X)=\chi(G\backslash X)$ for finite free $G$-CW complexes:
\begin{Theo}[Product Formula for $L^2$-torsion]\label{3.93(4)}
Let $X$ be a finite free $G$-CW complex and $Y$ a finite free $H$-CW complex. Suppose $X$ is det-$L^2$-acyclic, then the finite free $G\times H$-CW complex $X\times Y$ is det-$L^2$-acyclic and:
\begin{equation}
\rho^{(2)}(X\times Y, \vna(G\times H))=\chi(H\backslash Y)\cdot \rho^{(2)}(X, \vna(G))
\end{equation}
\end{Theo}
So now we consider a det-$L^2$-acyclic $\complex[\integer]$-chain complex $C_*$. First note any acyclic chain complexes of free modules over a principal ideal domain is contractible, by a easy induction from below and structure theorem of finitely generated module over principal ideal domain. Hence we have such $C_*$ is contractible. Now \Cref{3.93(1)} implies $\rho^{(2)}(C_*)=\phi^\integer(\tau(f))$ depending on the $G$-chain contraction $f$ we choose. However, since the Whitehead group $\Wh(\integer)$ vanishes (in fact, \cite{bass1964} shows Whitehead group vanishes for any $\integer^n$). Consequently, $\rho^{(2)}(C_*)=0$. So in view of \Cref{3.93(4)} we proved that for any finite free $G$-CW complex $D$, then $D\times S^1$ has vanishing $L^2$-torsion.



\section{Survey on Lie Algebra and Plancherel Formula}
From this section onwards we set off to compute the $L^2$-invariants of symmetric spaces, together with a discrete group $\Gamma$ acts on it properly free. This will provide an arsenal which may then be used to yield some partial results of some outstanding conjectures.
\par The trilogy begins with a crude survey on basic definitions and facts of representation of semisimple Lie groups. This is a vast topic with many ramifications, and has insofar witnessed many important applications to $L^2$-cohomology. It is way beyond the efforts of this article to even enumerate all these, and we will hence satisfy with ourselves at those aspects which related to our main discussion. For details of many important results we quote herewith the reader is referred to the classical text \cite{knapp2016}.
\par The second part is continuous cohomoloy of Lie groups, which is a natural outgrowth of the underlying representation theory. It has a potential to be extended to general reductive groups such as $p$-adic groups, but we are content with linear connected reductive groups. Interested reader can also refer to \cite{borel2013} for a thorough discussion.
\par Throughout the rest of the chapter we are working with \textbf{linear connected reductive group}, which is a closed connected group of real or complex matrices that is stable under conjugate transpose. In this section we collect all th relative terms and theorems in representation of Lie algebra, cumulating to the point of a Fourier inversion formula on such group, due to Harish-Chandra \cite{harish1976}.  
\par First recall an \textbf{analytic group} is a topological group with smooth structure compatible with group multiplication and inversion. A \textbf{Lie group} is a locally connected topological group with a countable base such that the identity component is an analytic group. An abstract \textbf{Lie algebra} is a vector spaces with Lie brackets, The Lie algebra of an analytic group $G$ is the tangent space at identity. Throughout this chapter we use old German fraktur letters $\Lie{g}, \Lie{h}, \cdots$ to denote Lie algebras. 
\par Given a real semisimple Lie algebra $\Lie{g}$, we define the \textbf{Killing form} $B$ for $\Lie{g}$ to be:
\begin{equation}
B(X, Y)=\tr(\ad_X\ad_Y)
\end{equation}
with $\ad_X$ the adjoint representation of Lie algebra on $\Lie{g}$ with $\ad_X(Z)=[X, Z]$ for $X, Z\in \Lie{g}$. We define a \textbf{Cartan involution} $\theta$ on $\Lie{g}$ to be an automorphism of $\Lie{g}$ such that $\theta^2=1$, and such that $-B(X, \theta Y)$ defines a positive definite symmetric form on $\Lie{g}$. Throughout our discussion we take $\Theta$ be inverse conjugate transpose on $G$ and take $\theta=d\Theta|_e$ and call both Cartan involution.
\par A $\theta$-stable \textbf{Cartan subalgebra} of $\Lie{g}$ is a subalgebra $\Lie{h}$ that is maximal among abelian $\theta$-stable subalgebras of $\Lie{g}$. A $\Theta$-stable \textbf{Cartan subgroup} $H$ of $G$ is defined as $Z_G(\Lie{h})$ for some $\theta$-stable Cartan subalgebra $\Lie{h}$. It is a fact that all $\theta$-stable Cartan subalgebras of $\Lie{g}$ have same dimension\footnote{This can be seen from that $\theta$-stable Cartan subalgebra are compatible with compact dual. Now all Cartan subgroup of a compact Lie group are conjugate with each other hence have same dimension. }, hence we can define the \textbf{(complex) rank} $\rk_\complex(G)$ of $G$ to be the dimension of $\theta$-stable Cartan subalgebra. As a surprising result, the rank have completely decide if the $L^2$-invariants of symmetric space of noncompact type vanish, as we will discuss in \Cref{olbrich1.1}.
\par With respective to $\theta$ one admits for $\Lie{g}$ a \textbf{Cartan decomposition}, given by:
\begin{equation}
\Lie{g}=\Lie{k}\oplus \Lie{p}
\end{equation}
with $\Lie{k}$ and $\Lie{p}$ are respectively $\pm 1$-eigenspaces of $\theta$. Since $\theta$ is an automorphism, we have:
\begin{equation}
[\Lie{k}, \Lie{k}]\subseteq \Lie{k},\qquad  [\Lie{k}, \Lie{p}]\subseteq \Lie{p},\qquad  [\Lie{p}, \Lie{p}]\subseteq \Lie{k}
\end{equation}
So in particular $\Lie{k}$ is a Lie subalgebra of $\Lie{g}$. 
\par Define \textbf{trace form} $B_0$ on $\Lie{g}$ by $B_0(X, Y):=\tr(XY)$. Note $B_0|_\Lie{p}$ is positive definite. Then $\brac{X, Y}=-\Re B_0(X,\theta(Y))$ defines an inner product on $\Lie{g}$, and we see $\Lie{p}$ and $\Lie{k}$ are orthogonal with respect to $\brac{-,-}$.
\par With respect to Cartan decomposition we can construct a compact dual of $G$, wherein we can apply all the nice theories of compact Lie groups. We can assume for most of time that $\Lie{k}\cap i\Lie{p}=0$.\footnote{For $G$ to be real linear this is trivial. for cases when $G$ such as $SL(2, \complex)$, we need to interpret $\Lie{g}^\complex$ and $G^\complex$ in the way that we regard $G\subseteq GL(n,\complex)\subseteq GL(2n, \real)$. The exact way we force $\Lie{k}\cap i\Lie{p}=0$ will not be important.} Then the \textbf{compact dual}, which we denote as $G^d$ is the analytic group of matrices with Lie algebra $\Lie{k}\oplus \Lie{p}$. It is a fact \cite[Proposition~5.3]{knapp2016} that $G^d$ is compact if $G$ is linear connected semisimple. 
\par Following this we fix a $\theta$-stable Cartan subalgebra $\Lie{h}$ and then form a Cartan subalgebra of the compact dual $\Lie{g}^d=\Lie{k}\oplus \Lie{p}$ to be $\Lie{h}^d=\Lie{(h\cap k)}\oplus i\Lie{(h\cap p)}$. Observe $\Lie{h}^\complex=\Lie{h}^\complex$, so then we can apply the root space decomposition in compact case to this by discussing $\Delta:=\Delta(\Lie{h}^\complex:\Lie{g}^\complex)$ (See \cite[Chapter~IV]{knapp2016} for discussion in compact Lie group case. It is also similar to the case of restricted root as mentioned later.) Again as in compact case, we can define an ordering for $\Delta$ and define a notion of positivity on it. Moreover, we can define a subset of \textbf{simple root} in positive roots, i.e., all those positive roots that cannot be decomposed into sum of two positive roots.
\par Now one can define an \textbf{algebraic Weyl group} $W(\Lie{h}^\complex:\Lie{g}^\complex) $ to be the generated by all root reflection $\{s_\alpha\}_{\alpha\in \Delta}$ with $s_\alpha$ the reflection of $\Delta(\Lie{h}^\complex:\Lie{g}^\complex)$ with respect to hyperplane $\alpha^\perp$. Furthermore, define \textbf{analytic Weyl group} $W(H:G):=N_K(H)/Z_K(H)$. Then we have $W(H:G)\subseteq W(\Lie{h}^\complex:\Lie{g}^\complex)$, with equality holds if $G$ is compact. 
\par Now consider the adjoint representation of Lie algebra. Consider $\Lie{a}$ to be a maximal abelian subspace of $\Lie{p}$. With respect to $\brac{-,-}$ we see each $\ad_{\Lie{p}}$ are symmetric matrices. So $\ad_{\Lie{a}}$ give a commuting subfamily of symmetric matrices are simultaneously diagonalizable with real eigenvalues. Then this defines a linear functional $\lambda$ on $\Lie{a}$, we write:
\begin{equation}
\Lie{g}_{\lambda}=\{X\in \Lie{g}\mid \forall H\in \Lie{a},{} [H, X]=\lambda(H)X \}
\end{equation}
If $\lambda$ is not trivial on $H$, and $\Lie{g}_\lambda\neq 0$, then we say $\lambda$ is a \textbf{restricted root} of $\Lie{g}$ with $\Lie{g}_\lambda$ the restricted root space. We denote the set of restricted roots with $\Delta(\Lie{a}: \Lie{g})$, or $\Sigma$, when there is no ambiguity on $\Lie{a}$. We also denote $Z_{\Lie{k}}(\Lie{a})$ to be all the elements in $\Lie{k}$ that commutes with all elements in $\Lie{a}$. Now:
\begin{Prop}\label{knapp5.9}
	\textnormal{\cite[Proposition~5.9]{knapp2016}} Restricted roots and their root space decomposition have the following properties:
	\begin{enumerate}
		\item $\Lie{g}=\Lie{g}_0\oplus \bigoplus_{\lambda\in \Sigma}\Lie{g}_\lambda$;
		\item $[\Lie{g}_\lambda, \Lie{g}_\mu]\subseteq \Lie{g}_{\lambda+\mu}$;
		\item $\theta \Lie{g}_\lambda=\Lie{g}_\lambda$ and hence $\Sigma$ is a symmetric subset of $\Lie{a}^*$;
		\item If $\lambda\neq \mu$, then $\Lie{g}_\lambda$ is orthogonal to $\Lie{g}_\mu$ with respect to $\brac{-,-}$;
		\item $\Lie{g}_0=\Lie{a}\oplus \Lie{m}$, where $\Lie{m}=Z_{\Lie{k}}(\Lie{a})$, the sum is orthogonal with respect to $\brac{-,-}$.
	\end{enumerate}
\end{Prop}
Again we may define positivity on $\Sigma$. Denote all the positive restricted roots to be $\Sigma^+$, we define $\Lie{n}=\sum_{\lambda\in \Sigma^+}\Lie{g}_{\lambda}$. Now from Cartan decomposition, which we can derive \textbf{Iwasawa decomposition}:
\begin{Theo}[Iwasawa decomposition]
	\textnormal{\cite[Theorem~5.12]{knapp2016}} For $G$ a linear connected semisimple group, then we have a direct sum decomposition $\Lie{g}=\Lie{k}\oplus \Lie{a}\oplus \Lie{n}$. Let $A$ and $N$ be the analytic subgroups with Lie algebras $\Lie{a}$ and $\Lie{n}$. Then $A, N$ and $AN$ are simply connected closed subgroups of $G$  and the multiplication map:
	\begin{equation*}
	K\times A\times N\to G, \qquad (k,a,n)\mapsto kan
	\end{equation*}
	is a diffeomorphism onto.
\end{Theo}
Iwasawa decomposition now allows us to define \textbf{real rank} of $G$ to be the dimension of $\Lie{a}$. This rank determines the existence of discrete series representation, which then decides the discrete spectrum of $\Delta$, as later discussions reveal.
\par Following the way of Iwasawa decomposition we can form a scheme of decomposing closed subgroups of $G$. If we write Iwasawa decomposition as $G=KA_\Lie{p}N_{\Lie{p}}$ and take compact subgroup $M_{\Lie{p}}=Z_K(A_{\Lie{p}})$. Note $M_\Lie{p}$ normalizes each space $\Lie{g}_\lambda$ and hence $M_{\Lie{p}}A_{\Lie{p}}N_{\Lie{p}}$ is a closed subgroup of $G$. We define $M_\Lie{p}A_\Lie{p}N_\Lie{p}$ and its conjugates to be \textbf{minimal parabolic subgroup} of $G$. 
\par A \textbf{parabolic subgroup} of $G$ is a closed subgroup containing some conjugate of $M_\Lie{p}A_\Lie{p}N_\Lie{p}$. Now for each parabolic subgroup there is a corresponding decomposition $S=MAN$ known as \textbf{Langlands decomposition}, which is uniquely defined by the following properties:
\begin{enumerate}
	\item At Lie algebraic level we have $\Lie{s}=\Lie{m}\oplus \Lie{a}\oplus \Lie{n}$;
	\item $\Lie{m}, \Lie{a}, \Lie{n}$ are mutually orthogonal with respect to $\brac{-,-}$ on $\Lie{g}$;
	\item $\Lie{m}\oplus \Lie{a}=\Lie{s}\cap \theta\Lie{s}=Z_{\Lie{g}}(\Lie{a})$;
	\item $\Lie{a}=\Lie{p}\cap Z_{\Lie{m}\oplus \Lie{a}}$.
\end{enumerate}
Now let $A, N, M_0$ be the analytic subgroup of $G$ corresponding to $\Lie{a}, \Lie{n}, \Lie{m}$ respectively, and denote $M=Z_K(\Lie{a})M_0$, then one can show that similar to Iwasawa decomposition that multiplication defines a diffeomorphism $M\times A\times N$ onto $S$, and $M_0$ is a linear connected reductive group with compact centre. Now the parabolic subgroups are classified by the following result:
\begin{Prop}\label{knapp5.23}
	\textnormal{\cite[Proposition~5.23]{knapp2016}} Fix $M_\Lie{p}A_\Lie{p}N_\Lie{p}$ and let $\Sigma^+$ be positive roots of $(\Lie{g}, \Lie{a_p})$ determined by $\Lie{n_p}$. Let $\Pi$ be the set of \textbf{simple root} in $\Sigma^+$, the same way as we defined semisimple case. Then there is a one-to-one correspondence between:
	\begin{equation}
	\Big\{S=MAN\mid S\supseteq  M_\Lie{p}A_\Lie{p}N_\Lie{p}\textnormal{  parabolic}\Big\}\longleftrightarrow \Big\{\Pi_S\subseteq \Pi\Big\}
	\end{equation}
	with the correspondence being $\lambda\in \Pi_S$ if and only if $\Lie{g}_\lambda\subseteq \Lie{m}$. Furthermore, no two of these parabolic are conjugate within $G$.
\end{Prop}
\begin{Rmk}\label{cuspidal parabolic}
	Note there is a natural construction of parabolic subgroups from $\theta$-stable Cartan subalgebras. Choose one such subalgebra $\Lie{h}$ and take $\Lie{m,a,n}$ as follows:
	\begin{enumerate}
		\item $\Lie{a}:=\Lie{h}\cap \Lie{p}$;
		\item $\Lie{m}$ the orthogonal complement of $\Lie{a}$ in $Z_{\Lie{g}}(\Lie{a})$;
		\item Perform a restricted root decomposition relative to $\ad \Lie{a}$ as in \Cref{knapp5.9} and similarly define $\Lie{n}$ to be the sum of positive root spaces. 
	\end{enumerate} 
Similar to before, we obtain a parabolic subgroup $S=MAN$. This parabolic subgroup is characterized by the fact that $\Lie{m}$ has a Cartan subalgebra $\Lie{h}\cap \Lie{k}$. We call such \textbf{cuspidal parabolic subgroup}. By \cite[Theorem~5.22(b)]{knapp2016}, there is a unique $\theta$-stable Cartan subalgebra $\Lie{h}$ (up to conjugacy) such that the Cartan subalgebra $\Lie{h\cap k}$ takes maximal dimension. We call the cuspidal parabolic subgroup constructed from such $\Lie{h}$ the \textbf{fundamental parabolic subgroup}.
\end{Rmk}
\begin{Rmk}\label{hierachy of parabolic subgroup}
	Together with \Cref{knapp5.23} we have if $\dim\Lie{h}_i\cap \Lie{p}\leq \dim\Lie{h}_j\cap \Lie{p}$, then we have the following ``hierarchy'', which can be made to subsets by conjugacy within $G$:
	\begin{equation}
	\dim \Lie{a}_i\leq \dim \Lie{a}_j \qquad \dim \Lie{m}_i\geq \dim \Lie{m}_j \qquad \dim \Lie{n}_i\geq \dim \Lie{n}_j
	\end{equation}
	So in particular, the fundamental parabolic subgroup $P=MAN$ takes minimal $\dim \Lie{a}=\dim\Lie{h}-\dim \Lie{h\cap k}$, where $\Lie{h\cap k}$ in this case is the Cartan subalgebra of $\Lie{k}$. Summing up, we have for any cuspidal subgroup $P_i=M_iA_iN_i$, we have $\dim \Lie{a_i}\geq \rk_\complex(G)-\rk_\complex\complex(K)$, with the equality taken in case of fundamental parabolic subgroup.
\end{Rmk}
As the last step before stating the main result, we need to define a special type of irreducible unitary representations on $M$ then derive an induced representation (of the same type) on $G$. The stunning result is that the character of these representations govern the a all Schwartz functions on $G/K$, as \nameref{plancherel} later reveals.
\begin{Def}\label{knapp9.6}
	\cite[Propsition~9.6]{knapp2016} Given an irreducible unitary representation $\pi$ of $G$ on $V$, the following conditions are equivalent:
	\begin{enumerate}
		\item $\pi$ is equivalent with a direct summand of the \textbf{right regular representation} $R$ of $G$ on $L^2(G)$, where $R(g)f(x)=f(xg)$;
		\item Given any $u,v\in V$, \textbf{matrix coefficients} $\pi_{u,v}:G\to \complex$ defined by $\pi_{u,v}(g)=\brac{\pi(u),v}_V$ is in $L^2(G)$.
	\end{enumerate}
When these conditions are satisfied, we say $\pi$ is in the \textbf{discrete series} of $G$.
\end{Def}
One should note when $G$ is compact, then every irreducible representation can be made unitary by averaging the inner product on $V$ and can be shown to be discrete series. Consequently, we know $M_\Lie{p}$ in minimal parabolic subgroup, which is always compact, have discrete series. 
\par More generally, we consider the case when $\rk_\complex G=\rk_\complex K$, then we can choose $\Lie{b\subseteq k\subseteq g}$ be a Cartan subalgebra. As a direct consequence,  we can define root systems for both $\Lie{g}^\complex$ and $\Lie{k}^\complex$ with respect $\Lie{b}^\complex$, as we did in restricted root case. Denote:
\begin{equation}
\Delta=\Delta(\Lie{b}^\complex: \Lie{g}^\complex)\qquad \Delta_K=\Delta(\Lie{b}^\complex:\Lie{k}^\complex)
\end{equation}
with respective algebraic Weyl group $W_G$ and $W_K$. Relative to any choice of positive system $\Delta^+$ we shall take $\Delta_K^+=\Delta^+\cap \Delta_K$, and define:
\begin{equation}
\delta_G:=\frac{1}{2}\sum_{\alpha\in \Delta^+}\alpha, \qquad \delta_K:=\frac{1}{2}\sum_{\alpha\in \Delta^+_K}\alpha
\end{equation}
Lastly we state for a general $\theta$-stable Cartan subalgebra $\Lie{h}$, a linear functionl $\lambda\in (\Lie{h}^\complex)^*$ is \textbf{analytically integral} if for all $H\in \Lie{h}_0:=\Lie{(h\cap k)\oplus i(h\cap p)}$ with $\exp H=1$, then $\lambda(H)\in 2\pi i\integer$. Note this implies that $\lambda$ is real-valued on $i\Lie{h}_0$.
\par Now we are ready to state the result of Harish-Chandra \cite{harish1966} which classified all discrete series representations:
\begin{Theo}\label{discrete series}
	\textnormal{\cite[Theorem~9.20 ,12.20 \& 12.21]{knapp2016}} A linear connected semisimple group $G$ has discrete series representations if and only if $\rk_\complex(G)=\rk_\complex(K)$. In such case, for $\lambda\in (i\Lie{b})^*$ \textbf{nonsingular}, i.e., for all $\alpha\in \Delta$, $\brac{\lambda, \alpha}\neq 0$. take $\Delta^+$ respectively to be:
	\begin{equation}
	\Delta^+=\{\alpha\in \Delta\mid \brac{\lambda, \alpha}>0\}
	\end{equation}
	If furthermore $\lambda+\delta_G$ is analytically integral, then there exists a discrete series representation $\pi_\lambda$, with two such constructed representations equivalent if and only if their parameters $\lambda$ are conjugate under $W_K$. Moreover, this has classified all discrete series up to equivalence.
\end{Theo}
\begin{Rmk}
	We call such $\lambda$ \textbf{Harish-Chandra parameter} of the discrete series $\pi_\lambda$. One can also write out properties of such discrete series (e.g. infinitesimal character, highest weight, etc.) Readers can refer to \cite[Theorem~9.22ff]{knapp2016} for more details.
\end{Rmk}
As one last step to-wards the statement of Plancherel formula, we want to see how a discrete series of $M$\footnote{one should be warned that we conceal much details about the discrete series of $M$ here, as $M$ is often not connected. To remedy this, we construct discrete series of $M_0$ and $M_0Z_M$, and extend them to $M$ by induced representation. To see more details, one can inspect \cite[Chapter~XII \S 8]{knapp2016}.} can induce a general representation of $G$ as follows:
\begin{Def}\label{principal series}
Given an irreducible unitary representation $\sigma$ of $M$ on a space $V^\sigma$ and $\nu\in (\Lie{a}^\complex)^*$, then we can form the \textbf{induced representation} of $\sigma\otimes \exp \nu\otimes 1$ of $S=MAN$ to be the representation of $G$, which we denote as $(\pi_{\sigma, \nu}, H^{\sigma, \nu})$, with the representation space defined as:
\begin{equation}
H^{\sigma, \nu}:=\Big\{F:G\to V^\sigma\mid F(xman)=e^{-(\nu+\rho_\Lie{a})\log a}\sigma(m)^{-1}F(x), F|_K\in L^2(K, V^\sigma)\Big\}
\end{equation}
with norm of $\norm{F}^2:=\int_K|F(K)|^2\mass{k}$, and $G$ in the induced representation acts on $F$ by $g\cdot F(x)=F(g^{-1}x)$. Here $\rho_\Lie{a}:=\frac{1}{2}\sum_{\alpha\in \Delta^+(\Lie{a}:\Lie{g})}\alpha$ the half sum of restricted positive roots, with positivity decided by $N$.
\end{Def} 
\par Recall $\mathcal{C}(G)$ the \textbf{Schwartz space} containing all smooth functions which are rapidly decaying under all left and right invariant derivatives by $U(\Lie{g}^\complex)$. (see \cite[Chapter~XII.\S 4]{knapp2016} for definition) and denote $\mathcal{C}(G)_{K\times K}$ the subspace of bi-$K$-finite-Schwartz functions.(c.f. \Cref{borel205}) 
\par Now we are ready to state the main result of this section:
\begin{Theo}[Plancherel formula]\label{plancherel}
	\textnormal{\cite[Theorem~13.11]{knapp2016}} Let $G$ be a linear connected reductive group, and let $H_1,\cdots, H_s$ be a complete set of non-conjugate $\Theta$-stable Cartan subgroups. Then there exist computable real analytic functions $p^{H_j}(\lambda, i\nu):i\Lie{b}_j^*\times i\Lie{a}_j^*\to [0, \infty)$ such that for all $f\in \mathcal{C}(G)_{K\times K}$: 
	\begin{equation}
	f(g)=\sum^{s}_{j=1}\Big\{\sum_{\lambda\in \hat{M}_d}\int_{\Lie{a}_j^*}\tr(\pi_{\lambda, i\nu}(f)\pi_{\lambda, i\nu}(g^{-1})) p^{H_j}(\lambda, i\nu)\mass{\nu}\Big\}
	\end{equation}
	where for each $j$, $\lambda$ runs over infinitesimal characters of all equivalence class of discrete series of $M_{j}$ constructed from $H_j$, and $\Lie{b}_j$ the subalgebra of $\Lie{m}_{j}$ and $\Lie{b}_j\oplus \Lie{a}_j=\Lie{h}_j$. Moreover, let $H_1$ be the maximally compact Cartan subgroup, then for a nonzero constant $c$, such that $p^{H_1}(\lambda, i\nu)$ takes the form:
	\begin{equation}
	p^{H_1}(\lambda, i\nu)=C(-1)^{\frac{\dim \Lie{n}_1}{2}}\prod_{\alpha\in \Delta^+}\frac{\brac{\lambda+i\nu, \alpha}}{\brac{\rho_{\Lie{g}},\alpha}}
	\end{equation}
	with $C$ a non-zero constant depending only on normalization of Haar measure.
\end{Theo}
\begin{Rmk}\label{normalization}
	There are several things to note in Plancherel formula:
	\begin{enumerate}
		\item The \textbf{Plancherel measures} $p^{H_j}(\lambda, i\nu)\mass{\nu}$ depends on the normalization of Haar measure $\mass{g}$. In our case we use the following: Let $\mass{x}$ be the Riemannian volume form of $X=G/K$ and $dk$ be the Haar measure of $K$ such that $\int_K\mass{k}=1$. Then we have:
		\begin{equation}
		\int_G f(g)\mass{g}=\int_X\int_K f(gk)\mass{k}\mass{x}
		\end{equation}
		We also normalize trace form $B_0$ on $\Lie{g}$ such that its restriction to $\Lie{p}\cong T_{eK}X$ coincide with the Riemannian metric of $X$. Then set $\mass{\nu}$ to be the Lebesgue measure corresponding to the induced form on $\Lie{a}^*$. By these choices $p^{H_j}(\lambda, -)$ is uniquely determined.
		\item The determination of the Plancherel density for fundamental parabolic subgroup has several variation. In particular, \cite{knapp2016} used an averaged version of characters, which is differed from the version of our use by a constant. Moreover, the constant $C_X$ was not explicitly there, which arise as we differentiate a function $F^T_f$. In general, this takes some effort to compute. In \cite{harish1975} and \cite{harish1976} both are explicitly computed, which is only different from our version by a constant arise from difference in normalization of measures $\mass{g}$ and $\mass{\nu}$. We shall explain this in times of need.
		\item The \textbf{Plancherel density} $p^{H_j}(\lambda, i\nu)$ is an elementary function, i.e., it is a function of one variable which is a composition of arithmetic operations. In particular, it is a function of polynomial growth.
		\item We note here the second sum runs over all Harish-Chandra parameters of $M$ associated to $H_j$, which we can, in view of \Cref{discrete series}, replace by all equivalence class of discrete series. Moreover, for each $H_j$ we can construct as before a cuspidal parabolic subgroup, so the first sum can also be summed over conjugacy classes of cuspidal parabolic subgroups of $G$.
		\item Note $\mathcal{C}(G)_{K\times K}$ is dense in $L^2(G)$. When $i\nu\in i\Lie{a}^*$ purely imaginary, we have $\pi_{\sigma, i\nu}$ unitary, and then for $f\in \mathcal{C}(G)_{K\times K}$:
		\begin{equation}
		\pi_{\sigma, i\nu}(f):=\int_G f(g)\pi_{\sigma, i\nu}\mass{g}
		\end{equation}
		defines a trace-class operator on $H^{\sigma, \nu}$. This in fact holds for every admissible representation with $K$-types bounded.(See for instance \cite[Theorem~10.2]{knapp2016})	
		\item Note in \ref{olbrich13} $\dim \Lie{n}_1$ of fundamental parabolic subgroup $P_1=MAN$ has even dimension \cite[Chapter~III, Lemma 4.2(i)]{borel2013}. 
	\end{enumerate}
\end{Rmk}
To conclude this section, we mention lastly admissible representations of a Lie group $G$ and its interaction with centre of universal enveloping algebra, which will be of crucial importance to our ensuing discussions on relative Lie cohomology. We begin with universal enveloping algebra.
\begin{Def}
	Let $\Lie{g}$ be a finite-dimensional Lie algebra over $\complex$ and let $T(\Lie{g}):=\sum_{r=0}^\infty \bigotimes^r\Lie{g}$ the \textbf{tensor algebra} of $\Lie{g}$, and the \textbf{universal enveloping algebra} $U(\Lie{g})$ is the quotient of $T(\Lie{g})$ by two sided ideal generated by:
	\begin{equation*}
	\{(X\otimes Y-Y\otimes X-[X, Y])\mid X, Y\in \Lie{g}\}
	\end{equation*}
	It satisfied the universal property that if $\phi: \Lie{g}\to A$ a homomorphism between associative algebras with identity such that $\phi([X, Y])=\phi(X)\phi(Y)-\phi(Y)\phi(X)$, then it admits a unique lift to an algebra homomorphism $\bar{\phi}:U(\Lie{g})\to A$ via the canonical inclusion $\Lie{g}\to U(\Lie{g})$. 
\end{Def}
Let $G_\real$ be an analytic group with Lie algebra $\Lie{g}_\real$. Now we can identify $U((\Lie{g}_\real)^\complex)$ with the space of left-invariant differential operator $D(G_\real)$ via the following algebra isomorphism: Given $X\in \Lie{g}_\real$, then assign to it a left-invariant vector field $\tilde{X}$ via:
\begin{equation}
\tilde{X}f(x)=\frac{d}{dt}f(x\cdot(\exp tX))\big|_{t=0}
\end{equation}
One can also take it as an first-order differential operator. Now extend the map to $U((\Lie{g}_\real)^\complex)$ via universal property. Conversely, given a differential operator $D\in D(G_\real)$, we realize it as an element in $U(\Lie{g})$ use the equation above. 
\begin{Def}\label{casimir}
	Denote $Z(\Lie{g}^\complex)$ the centre of $U(\Lie{g}^\complex)$. By \cite[Proposition~3.8]{knapp2016} we can identify the centre with all $G$-invariant differential operator, i.e., 
	\begin{equation}
	Z(\Lie{g}^\complex)=\{D\in U(\Lie{g}^\complex)\mid \forall g\in G, \Ad_gD=D  \}
	\end{equation}
	Recall $B(X,Y)=\tr(\ad X \ad Y))$ the Killing form of $\Lie{g}$. Choose a basis $X_1, \dots, X_n$ of $\Lie{g}$, then $g=[g_{ij}]:=[B(X_i, X_j)]$ defines a $n\times n$-invertible matrix with inverse $[g^{ij}]$. Put $X^j=\sum_i g^{ij}X_i$. We now define the \textbf{Casimir element} $\Omega$ in $\Lie{g}^\complex$ by $\Omega:=\sum_{i,j}g_{ij}X^iX^j$. One can show $\Ad_g\Omega=\Omega$ for all $g\in G$, hence we have $\Omega\in Z(\Lie{g}^\complex)$.
\end{Def}
\begin{Def}\label{borel205}
	Let $G$ be linear connected reductive with compact subgroup $K$ and $\pi$ is a representation of $G$ on a Hilbert space $V$. We call $v\in V$ to be \textbf{$K$-finite} if $\pi(K)v$ spans a finite-dimensional space. When $K$ acts by unitary operators, then $\pi|_K$ splits into orthogonal sum of irreducible representations by Peter-Weyl Theorem \cite[Theorem~1.12]{knapp2016}:
	\begin{equation}\label{knapp(8.5)}
	\pi|_K=\bigoplus_{\tau \in \hat{K}}n_\tau \tau
	\end{equation} 
	with $\hat{K}$ denotes the equivalence class of irreducible representations of $K$, and $n_\tau\in \nat\cup \{\infty \}$ is the \textbf{multiplicity}. Consequently, $K$-finite vector is dense in $V$. We call all those $\tau$ with positive multiplicity in $\pi$ the \textbf{$K$-types} of $\pi$.
\end{Def}
One can prove for irreducible unitary representation on $V$, the multiplicities $n_\tau\leq \dim\tau$ for every $\tau$ in $K$-type of $\pi$ (c.f. \cite[Theorem~8.1]{knapp2016}). This then motivates the following definition:
\begin{Def}
	A representation of linear connected reductive group $G$ on Hilbert space $V$ is \textbf{admissible} if:
	\begin{enumerate}
		\item $\pi(K)$ operates by unitary operators;
		\item Each $K$-type of $\pi$ has finite multiplicities.
	\end{enumerate}
\end{Def}
If $K$ is furthermore maximally compact, then every $K$-finite vector in an admissible representation is a $C^\infty$-vector, and moreover is the space of $K$-finite vectors is stable under $\pi(\Lie{g})$ (see \cite[Proposition~8.5]{knapp2016}). Consequently we have admissible representation induces a representation of $\Lie{g}$ on $K$-finite vectors. we can define any two admissible representations $\pi$ and $\pi'$ of $G$ on $V$ and $V'$ respectively are \textbf{infinitesimally equivalent} if there is a linear isomorphism $L:V\to V'$ such that $\pi(\Lie{g})L=L\pi'(\Lie{g})$. 
\par Analogically can define \textbf{$\Lie{k}$-finite} vectors in a representation of $\Lie{g}$ on $V$ to be those vectors such that $U(\Lie{k}^\complex)\cdot v$ is finite dimensional; and \textbf{admissible representations of $\Lie{g}$} to be such where $\Lie{k}$ acts as skew-adjoint operators and each $\Lie{k}$-type has finite multiplicities. Note $K$ acts on $V$ as unitary operators implies that $\Lie{k}$ acts as skew-adjoint operators. 
\par The significance of admissible representations is twofold: First it allows us to define two type of characters, from which we are enough to distinguish infinitesimally inequivalent representations; Second its character can be expressed as a locally integrable function of $G$ which is analytic on the regular part. we focus on the first property here. As it turns out in \Cref{borel1.4.2}, the characters has determined if the cohomology vanish.
\begin{Theo}\label{knapp8.7}
	\textnormal{\cite[Theorem~8.7 \& 8.9; Corollary 8.10 \& 8.14]{knapp2016}} Let $\pi$ be an admissible representation of a linear connected reductive group $G$ on $V$. Let $V_0$ be the subspace of $K$-finite vectors in $V$, then:
	\begin{enumerate}
		\item For any $u\in V_0, v\in V$, we have the function $\pi_{u,v}:g\mapsto \brac{\pi(g)u,v}$ a analytic function on $G$;
		\item Any $\Lie{g}$-invariant subspace of $K$-finite vectors is $G$-invariant;
		\item There is a one-to-one correspondence between:
		\begin{equation*}
		\{\textnormal{closed $G$-invariant subspaces of $U$ of $V$}\}\longleftrightarrow \{\textnormal{$\Lie{g}$-invariant subspaces of $U_0$ of $V_0$}\}
		\end{equation*}
		The correspondence being $U_0=U\cap V_0$ and $U=\bar{U_0}$. In particular, $\pi(G)$ has no nontrivial closed invariant subspaces in $V$ if and only if $\pi(\Lie{g})$ has no nontrivial invariant subspaces in $V_0$. In such case we call $\pi$ \textbf{irreducible admissible}.
		\item If $\pi$ is irreducible admissible, then each member of $Z(\Lie{g}^\complex)$ acts as scalar operators on $V_0$. 
	\end{enumerate}
\end{Theo}
More generally  when $G$ is an admissible representation $\pi$ for which $\pi(Z)$ acts as scalars on $K$-finite matrices (in particular when $\pi$ is irreducible), we can define a character $\chi:Z(\Lie{g}^\complex)\to \complex$ given by $\pi(Z)=\chi(Z)\cdot \id$. In such case we say $\pi$ has \textbf{infinitesimal character} $\chi$.
\par While this character garners information of differential operators on $G$, the other type of character gives information on $L^2(G)$. Recall $\pi(f):=\int_G \pi(g)f(g)\mass{g}$ gives a representation of $C^\infty_c(G)$ on $V$.
\begin{Def}
	Given an admissible representation $(\pi,V)$ of $G$ on has a \textbf{global character} $\Theta$ if for all $ f\in C^\infty_c(G)$, $\pi(f)$ is of trace class when considered as a endomorphism of $V$ and if $\Theta: f\mapsto \tr\pi(f)$ is a distribution.
\end{Def} 
Now admissible representations works acts an ideal receptacle for global characters, as the following reveals:
\begin{Theo}
	\textnormal{\cite[Theorem~10.2]{knapp2016}} An admissible representation $\pi$ of a linear connected reductive group $G$ has a global character if the multiplicities of each $K$-type is universally bounded by their dimension. that is, there is a $C>0$, such that $n_\tau<C\dim \tau$ for all $\tau\in \hat{K}$ in the decomposition \ref{knapp(8.5)}. In particular, every irreducible admissible representation has a global character.
\end{Theo}


\section{Continuous Cohomology and Vanishing Results}
In this section we brought cohomological meaning to representations. Given $X=G/K$ with Riemannian volume and Haar measure chosen at \Cref{normalization}, we want to break $L^2\Omega^p(X)$ into irreducible unitary representations of $G$. Now note in view of Cartan decomposition, there is an isomorphism of homogeneous vector bundles $\Lambda^pT^*X\cong G\times_K \Lambda^p\Lie{p}^*$, which indeed gives an isomorphism of $G$-representations:
\begin{equation}
L^2\Omega^p(X)=L^2(X, \Lambda^pT^*X)\cong [L^2(G)\otimes \Lambda^p\Lie{p}^*]^K
\end{equation}
Now \nameref{plancherel} have already allow us to understand $L^2(G)$ by trace of the representation on it. What we need is to decipher $K$-invariant space $[V_\pi\otimes \Lambda^p\Lie{p}^*]^K$ occurring in Plancherel decomposition. To understand them we need to first frame them into a cohomological setting, and then we can use the homology theory tools to yield many vanishing results. These altogether will give the answer to $L^2$-Betti numbers directly. For Novikov-Shubin invariants and $L^2$-torsions, this however requires a closer inspection of the respective representations.
\par We shall first begin laying the cohomological frameworks.Throughout this section the group will again be a linear connected reductive group, mostly semisimple. The proofs can largely be retrieved from \cite{borel2013}.
\par The aforementioned properties of admissible representation are so crucial to our ensuing discussions that we axiomatize them as following:
\begin{Def}
	Let $G$ be a linear connected reductive group with $K$ be one of its maximal compact subgroup. A \textbf{$(\Lie{g}, K)$-module} (resp. a \textbf{$(\Lie{g}, \Lie{k})$-module}) is a real or complex vector space $V$ which is $\Lie{g}$-module and a semisimple $K$-module (resp. semisimple $\Lie{k}$-module) such that:
	\begin{enumerate}
		\item Every $v\in V$ is $K$-finite (resp. $\Lie{k}$-finite) module;
		\item For all $k\in K$, $X\in U(\Lie{g})$ and $v\in V$, we have $\pi(k)\cdot (\pi(X))\cdot v=\pi(\Ad_k(X))\pi(k)(v)$;
		\item If $F$ is a $K$-stable finite-dimensional subspace of $V$, then the representation of $K$ on $F$ is differentiable, has $\pi|_{\Lie{k}}$ as its differential.
	\end{enumerate}
\end{Def} 
Note the second and the third are solely reserved for the $(\Lie{g}, K)$-case to make sure the compatibility of two representations. For $(\Lie{g}, \Lie{k})$-modules we restrict the module structure of $\Lie{g}$ to $\Lie{k}$. In particular, from the above discussions we see for admissible representations $V$ is both a $(\Lie{g}, \Lie{k})$-module and a $(\Lie{g}, K)$-module.
\par Now we are ready to define the cochain complex. Fix a representation $(\pi, V)$ of $\Lie{g}$, where the space $V$ over a field $F$ is often infinite-dimensional. Denote $C^q=C^q(\Lie{g}; V)=\Hom(\Lambda^q\Lie{g}, V)$, with differential $d:C^q\to C^{q+1}$:
\begin{multline}\label{borel1.1.1}
df(X_0,\cdots , X_q)=\sum_i(-1)^i\cdot f(X_0, \cdots, \hat{X_i}, \cdots, X_q)\\
+\sum_{i<j}(-1)^{i+j}\cdot f([X_i, X_j], X_0, \cdots , \hat{X_i}, \cdots, \hat{X_j}, \cdots, X_q)
\end{multline}
Define furthermore the endomorphisms $i_X, \theta_X$ respectively by:
\begin{equation}
\begin{split}
i_Xf(X_1, \cdots, X_{q-1})&:=f(X, X_1, \cdots, X_{q-1})\\
\theta_Xf(X_1, \cdots, X_q)&:=\sum_if(X_1, \cdots, [X_i,X], \cdots,X_q)+X\cdot f(X_1, \cdots, X_q)
\end{split}
\end{equation}
Invoke from differential geometry that $d, i_X, \theta_X$ here corresponds to the covariant derivative $\nabla$, the contraction $i_X$, and the Lie derivative $\mathcal{L}_X$ on $(p,0)$-tensor respectively. Consequently from Cartan's magic formula one has $\theta_X=d\circ i_X+i_x\circ d$. Further define:
\begin{equation}
C^q(\Lie{g}, \Lie{k};V) :=\{\phi\in C^q(\Lie{g};V)\mid \forall x\in \Lie{k}, i_X\phi=\theta_X\phi=0 \}
\end{equation}
One can then readily check this is a submodule stable under $d$. Consequently one can define \textbf{relative cohomology groups} $H^q(\Lie{g}, \Lie{k}; V)$. One can also identify the relative cochain complex with $C^q(\Lie{g}, \Lie{k};V)=\Hom_{\Lie{k}}(\Lambda^q(\Lie{g}/\Lie{k}), V)$, where $\Lie{k}$ acts on $\Lambda^q(\Lie{g}/\Lie{k})$ via adjoint representation induced on $\Lie{g/k}$, i.e, for $X\in \Lie{k}$, and $X_i\in \Lie{g/k}$,
\begin{equation}
X\cdot f(X_1,\cdots, X_q)=\sum_i f(X_1, \cdots, [X,X_i],\cdots,X_q)
\end{equation} 
It is straightforward to check this is a $(\Lie{g}, \Lie{k})$-module.
\par Analogously we can define $C^q(\Lie{g}, K;V):=\Hom_K(\Lambda^q(\Lie{g}/\Lie{k};V))$ with $K$ acts again via adjoint representation. Take $K_0$ to be the identity component of $K$, then derivative at $e$ and exponential map gives a isomorphism $\Hom_{K_0}(\Lambda^q(\Lie{g}/\Lie{k});V)\cong \Hom_{\Lie{k}}(\Lambda^q(\Lie{g}/\Lie{k});V)$ where $K/K_0$ acts naturally on the left. Consequently, we see $C^q(\Lie{g},\Lie{k};V)^{K/K_0}=C^q(\Lie{g},K;V)$ since the action clearly commutes with $d$, we have $H^q(\Lie{g}, K;V)=H^q(\Lie{g}, \Lie{k};V)^{K/K_0}$.
\begin{Rmk}
	$(\Lie{g}, \Lie{k})$-cohomology can be related to differential form in the following sense: If we take $F=\real$ and let $K$ be a closed connected subgroup of $G$, with respective Lie algebras $\Lie{g}$ and $\Lie{k}$. Now if $V$ is a smooth $G$-module, then an element of $C^q(\Lie{g}, \Lie{k};V)=\Hom_{\Lie{k}}(\Lambda^q(\Lie{g}/\Lie{k}), V)$, defines an $q$-form on $G/K$ at $eK$. Since $G$ acts transitively on $G/K$, this then induces a $G$-invariant $q$-form $\omega\in \Omega^q(G/K; V)$. Conversely, $\omega$ evaluates at $e$ gives an element in $C^q(\Lie{g}, \Lie{k};V)$. Hence we have a graded isomorphism from $G$-invariant differential forms $\Omega^*(G/K;V)^G$ onto $C^*(\Lie{g}, \Lie{k};V)$. In particular, when $G$ is compact, we have then $H^*(\Lie{g}, \Lie{k};V)=H^*_{\mathrm{dR}}(G/K;V)$ since all forms in such case can be made $G$-invariant by averaging the inner product over $G$.
\end{Rmk}
We can more generally define abelian categories of $(\Lie{g}, \Lie{k})$-modules and $(\Lie{g}, K)$-modules with enough projectives and injectives. Consequently one can define $\Ext$-functors $\Ext^q_{\Lie{g},\Lie{k}}(U,V)$ and $\Ext_{\Lie{g}, K}^q(U,V)$ as the derived functors of $\Hom_\Lie{g}(U,V)=\Hom_{\Lie{g}, \Lie{k}}(U,V)$ and of $\Hom_{\Lie{g}, K}(U,V)$ respectively. For details see \cite[Chapter~1, \S 2]{borel2013}.
\par Now by appealing to general theory as in \cite[Chapter~IX, Corollary~4.4]{cartan2016}, we see:
\begin{equation}\label{cartan identity}
\Ext^q_{\Lie{g},\Lie{k}}(F, \Hom_F(U,V))=\Ext^q_{\Lie{g},\Lie{k}}(U,V)
\end{equation}
for any $(\Lie{g}, \Lie{k})$-modules $U, V$. Now we need to identify the left-hand side with $H^q(\Lie{g}, \Lie{k};V)$:
\begin{Lemma}\label{borel1.2.5}
	When $F$ is a field of characteristic zero and $\Lie{k}$ if reductive in $\Lie{g}$, then for any $(\Lie{g}, \Lie{k})$-module:
\begin{equation}
H^q(\Lie{g}, \Lie{k};V)=\Ext^q_{\Lie{g},\Lie{k}}(F, \Hom_F(F,V))=\Ext^q_{\Lie{g},\Lie{k}}(F,V)
\end{equation}
\end{Lemma}
\begin{proof}
	By taking $X_q:=\Lie{g}\otimes_{\Lie{k}}\Lambda^q(\Lie{g}/\Lie{k})$, we define an projective resolution of $F$:
	\begin{equation}
	\begin{tikzcd}
	\cdots \rar &X_q\rar{\partial_q} &\cdots \rar{\partial_1}&X_0\rar{\epsilon} &F \rar &0
	\end{tikzcd}
	\end{equation}
	with $\epsilon: X_0\to F$ the augmentation and $\partial_q:X_q\to X_{q-1}$ the same way as \ref{borel1.1.1}. Note each $X_q$ are projective since it follows from \cite[Chapter~I, \S 2.4]{borel2013} that for each $(\Lie{g}, \Lie{k})$-module $V$, the induced module $\Lie{g}\otimes_\Lie{k}V$ is projective.\footnote{Reader is to note here the $\Lie{k}$-module structure of $\Lie{g}$ is induced by adjoint representation.Moreover, as a byproduct of the proof, one see the $\Lie{k}$-module structure given by right multiplication on $\Lie{g}$-component coincide with the induced $\Lie{k}$-module structure on tensor product via adjoint representation on $\Lie{g}$-component and intrinsic $\Lie{k}$-module structure on $U$ itself.} Consequently we see the above chain is exact, and then defines a projective resolution of $F$. Now $\Hom_{\Lie{g}}(X_q, V)=\Hom_{\Lie{k}}(\Lambda^q(\Lie{g}/\Lie{k}), V)$ by Frobenius Reciprocity, and from the definition of $C^q(\Lie{g}, \Lie{k};V)$ we see the claim of the lemma readily follows.
\end{proof}
The above construction can be analogously extended to $\Ext$-functor in the category of $(\Lie{g},K)$-module. From the definition we can easily derive that $\Hom_{\Lie{g},K}(U,V)=\Hom_{\Lie{g},K^0}(U,V)^{K/K_0}$. Consequently apply this to the projective resolution, and we have:
\begin{equation}
 \Ext^q_{\Lie{g}, K}(U,V)=(\Ext^q_{\Lie{g},K^0}(U,V))^{K/K_0}=\Ext^q_{\Lie{g},\Lie{k}}(U,V)^{K/K_0}
\end{equation}
The last thing we need before stating our first vanishing theorem is the dual structure $(\Lie{g},K)$ and $(\Lie{g},\Lie{k})$-structure on the dual vector space $V'$. Note $V'$ does not remain to be a $(\Lie{g}, K)$-module mostly, so we need a suitable subspace.
\begin{Def}
	Given $(\pi,V)$ now a $(\Lie{g},K)$-module, we then define \textbf{contragredient $(\Lie{g},K)$-module to $V$} be the space of $K$-finite vectors in $V'$, with the \textbf{contragredient representation} $\tilde{\pi}$ be $\tilde{\pi}(x):=\pi^t(-x)$ of $x\in \Lie{g}$, where $\pi^t$ the transpose of $\pi$. Similarly for $(\Lie{g,k})$-module $W$ we define \textbf{contragredient $(\Lie{g},\Lie{k})$-module to $W$} be the space in $V'$ spanned by all $\Lie{k}$-finite dimensional subspaces, with contragredient representation.
\end{Def}
\begin{Theo}\label{borel1.4.2}
	\textnormal{\cite[Theorem~4.1\& Theorem~5.3]{borel2013}} Let $U, V$ be two $(\Lie{g}, \Lie{k})$-modules (resp. $(\Lie{g}, K)$-modules) with infinitesimal characters $\chi_U, \chi_V$ respectively. If $\chi_U\neq \chi_V$, then the $\Ext$-groups $\Ext^q(U,V)$ and $\Ext^q_{\Lie{g}, K}(U,V)$) vanish for all $q$'s respectively. In particular, if $U$ is finite dimensional and $\chi_{\tilde{U}}\neq \chi_V$, then $H^q(\Lie{g}, \Lie{k}; U\otimes V)$ and $H^q(\Lie{g}, K; U\otimes V)$ vanish for all $q$'s respectively.
\end{Theo}
\begin{proof}[Sketch of Proof]
	We skip proof of the first part. Roughly speaking, when two representations have different infinitesimal characters, then one can find an element $z\in Z(\Lie{g}^\complex)$ that `separates' two functional such that $\chi_V(z)=0$, and $\chi_U(z)=1$, then one proves that $Z(\Lie{g}^\complex)$ act on $\Ext^q(U,V)$ via left multiplication is independent of its action on $U,V$ respectively. Consequently we have the desired result.
	\par Assuming the first part, the second part follows readily by observing $\tilde{U}=U'$ due to finite dimensionality, therefore from \Cref{borel1.2.5} and \ref{cartan identity}
	\begin{equation}
	H^q(\Lie{g},\Lie{k},U\otimes_F V)=H^q(\Lie{g},\Lie{k},\Hom_F(U',V))=\Ext^q_{\Lie{g},\Lie{k}}(U',V)
	\end{equation}	
\end{proof}
Now we are to discuss a special case of vanishing theorem where $\Omega$ plays an essential role. Let $V=H\otimes E$, with $(\rho,E)$ is a finite-dimensional continuous representation of $G$, and $(\sigma,H)$ is a unitary $(\Lie{g},\Lie{k})$-module, that is $H$ is a pre-Hilbert space on which $\Lie{g}$ acts as formally skew-adjoint operators. Then $\tau=\sigma\otimes \rho$  is the tensor product of representation on $H\otimes E$ given by $\tau(x)=\sigma(x)\otimes \id_E+\id_H\otimes \rho(x)$. Take $K$ with Lie algebra $\Lie{k}$ to be a maximal compact subgroup of $G$, then by Cartan decomposition we have 
\begin{equation}\label{borel2.1.4(1)}
C^q(\Lie{g}, \Lie{k};V)=\Hom_{\Lie{k}}(\Lambda^q\Lie{p},V)=(\Lambda^q\Lie{p}^*\otimes V)^\Lie{k}
\end{equation}
Meanwhile, since $[\Lie{p}, \Lie{p}]\subseteq \Lie{k}$, we have the second term in \ref{borel1.1.1} vanishes, and to stress the representation $\pi$, we write
\begin{equation}\label{borel2.1.5(5)}
df(X_0,\cdots , X_q)=\sum_i(-1)^i\pi(X_i)\cdot (f(X_0, \cdots, \hat{X_i}, \cdots, X_q))
\end{equation}
Now endow $E$ with an \textbf{admissible scalar product}, i.e., one which $\Lie{k}$ acts via skew-adjoint operators and $\Lie{p}$ acts as self-adjoint operators. To see one can always do such with finite-dimensional representations, first lift the representation to $\Lie{g}^\complex$ on $V$ using the intrinsic complex structure on $V$. Then restrict to $\Lie{k}\oplus i\Lie{p}$, which is the Lie algebra associated to the compact dual. Now applying the averaging argument to $G^d$ to get a unitary representation on $V$, then we left the reader to check it is the desired scalar product. Consequently we can give $V$ the tensor product representation$\brac{,}_H\otimes \brac{,}_E$, and recall the scalar product on $\Lambda^q\Lie{p}^*$ is induced by $B(-,-)$, which is positive definite on $\Lie{p}$. (c.f. \Cref{casimir})
\par Now we can accordingly define an adjoint representation $\tau^*$ of $\Lie{g}$ with respect to $\brac{-,-}_V$, then we see $\tau^*(\Lie{k})$ acts as skew-adjoint operators, and:
\begin{equation}
\tau^*(X)=\id_H\otimes \rho(X) - \sigma(X)\otimes \id_E \qquad \text{if $X\in \Lie{p}$}
\end{equation}
Now we may define a dual differential $\delta:C^q(\Lie{g}, \Lie{k};V)\to C^{q-1}(\Lie{g,k};V)$ by:
\begin{equation}\label{borel2.2.3(1)}
\delta\eta(X_0, \cdots X_q)=\sum_{j=1}^m \tau^*(Y_j)(\eta(Y_j,X_0,\cdots, X_q))
\end{equation}
where $Y_j$ is a orthogonal basis of $\Lie{p}$ with respect to the real part of trace form. We leave the reader to check $\delta(C^q)\subseteq C^{q-1}$ and moreover for each $\eta\in C^q, \mu\in C^{q-1}$, one has $\brac{\delta\eta, \mu}=\brac{\eta, d\mu}$.
\par Naturally we now define the Laplacian of the chain complex $C^*(\Lie{g}, \Lie{k};V)$: $\Delta=d\delta+\delta d$, then the following lemma shows the Laplacian in such case is essentially the Casimir element as defined in \Cref{casimir}:
\begin{Theo}[Kuga's Lemma]\label{Kuga}
Let $\tau=\sigma\otimes \rho$ and view $V$ as a $(\Lie{g}, \Lie{k})$-module under $\tau$, with corresponding Laplacian $\Delta_\tau$. Then:
\begin{equation}
\Delta_{\tau}\eta=(\rho(\Omega)-\sigma(\Omega))\cdot \eta
\end{equation}
where $\Omega$ is the Casimir element.
\end{Theo}
\begin{proof}
	See \cite[Chapter~II, Theorem~2.5]{borel2013}. The proof is elementary.
\end{proof}
Now we have the vanishing of cohomology totally determined by Casimir element in case of irreducible admissible representations:
\begin{Prop}\label{olbrich3.1}
	Assume that $\sigma(\Omega)=s\cdot\id_H$ and $\rho(\Omega)=r\cdot \id_E$, (This is in particular the case when $H$ is an irreducible admissible representation and $E$ is furnished with $\Lie{g}$-invariant metric), then:
	\begin{enumerate}
		\item If $r\neq s$, then $H^q(\Lie{g}, \Lie{k}; H\otimes E)=0$ for all $q$'s;
		\item If $r=s$, then all cochains are closed, harmonic, and we have:
		\begin{equation*}
		H^q(\Lie{g}, \Lie{k};H\otimes E)=C^q(\Lie{g}, \Lie{k}; H\otimes E) \qquad \text{for all $q$'s}
		\end{equation*}
		\item If $(\rho,E)$ is further irreducible, then $H^*(\Lie{g}, \Lie{k};E)=0$ if $\rho$ is nontrivial, and $H^q(\Lie{g}, \Lie{k};E)=C^q(\Lie{g,k};E)$ for all $q$'s if $\rho$ is trivial representation.
		\item If $H_K$ is the space of $K$-finite vectors in an irreducible admissible representation $H$ of $G$. Then $H^q(\Lie{g,k};H_K)=\Hom_{\Lie{k}}(\Lambda^q\Lie{p},H_K)$ if $\sigma(\Omega)=0$; and $H^*(\Lie{g,k};H_K)$ vanish if otherwise.
	\end{enumerate}
\end{Prop}
\begin{proof}
First by \nameref{Kuga} we have $\Delta_\tau=(r-s)\cdot \id_{H\otimes E}$ on $C^*(\Lie{g}, \Lie{k};H\otimes E)$. Hence if $r\neq s$, then for $\eta\in C^q$ such that $d\eta=0$, then $\Delta \eta=d\delta \eta$. Then $\eta=(r-s)^{-1}\cdot d\delta \eta$, hence is a coboundary, and hence we have the first part.
\par When on the other hand $r=s$, then $\Delta=0$, then argue as in \Cref{1.18} one has $d\eta=\delta\eta=0$ for all $\eta\in C^q$ for all $q$. Hence the second part.
\par Now if $\rho$ is irreducible, then we have by Schur's Lemma, $\rho(\Omega)=r\cdot \id$, also note that by \cite[Proposition~5.28(c)]{knapp2013} we have $r=0$ if and only if $\rho$ is trivial representation. Hence the third statement readily follows by taking $\sigma$ to be trivial representation. The last statement is straightforward.
\end{proof}
From now on we want to focus on the $(\Lie{g},K)$-cohomology when $V$ is a discrete series representation or induced representations $(\pi_{\lambda, i\nu}, H^{\lambda, i\nu})$ as occurring in \nameref{plancherel}. To prepare this, we first need to better understand the infinitesimal characters, and second to discuss its interplay with discrete series and induced series respectively.
\par As observed by Harish-Chandra \cite{harish1951}, the infinitesimal character is completely determined by a functional on some Cartan subalgebra $\Lie{h}^\complex$. In fact he constructed an explicit homomorphism: 
\begin{Def}\label{knapp(8.24)}
	We define the \textbf{Harish-Chandra homomorphism} to be an homomorphism $\gamma:Z(\Lie{g}^\complex)\to U(\Lie{h}^\complex)$ such that for all $Z\in Z(\Lie{g}^\complex)$ and $\Lambda\in (\Lie{h}^\complex)^*$:
	\begin{equation}
	\gamma(Z)(\Lambda)=\gamma_{\Delta^+}'(Z)(\Lambda-\delta_G)
	\end{equation}
	where $\gamma_{\Delta^+}'$ is the projection fo $Z(\Lie{g}^\complex)$ onto $U(\Lie{h}^\complex)\cap Z(\Lie{g}^\complex)$, and $\delta_G=\frac{1}{2}\sum_{\alpha\in\Delta^+}^{}\alpha$ the half sum of positive roots.
\end{Def}
\begin{Theo}\label{knapp8.18}
	\textnormal{\cite[Theorem~8.18]{knapp2016}} $\gamma$ is an algebra isomorphism of $Z(\Lie{g}^\complex)$ onto the subalgebra $(U(\Lie{h}^\complex))^W$ of $U(\Lie{h}^\complex)$ containing all those elements fixed by algebraic Weyl group $W:=W(\Lie{h}^\complex:\Lie{g}^\complex)$. Moreover, the homomorphism is solely dependent on choice of Cartan subalgbera $\Lie{h}^\complex$ and is independent of choice of positive system $\Delta^+$.
\end{Theo}
So now for any $\gamma\in (\Lie{h}^\complex)^*$, we can define $\chi_{\Lambda}$ as a linear functional on $Z(\Lie{g}^\complex)$ using Harish-Chandra homomorphism as:
\begin{equation}
\chi_{\Lambda}(z)=\Lambda(\gamma(z)) \qquad \text{for $z\in Z(\Lie{g}^\complex)$}
\end{equation}
by extending $\Lambda$ to a linear functional on $U(\Lie{h}^\complex)$ via universal property. Now it turns out all linear functional on $Z(\Lie{g}^\complex)$ are characterized in this way:
\begin{Theo}
	\textnormal{\cite[Proposition~8.20 \& 8.21]{knapp2016}} Every homomorphism from $Z(\Lie{g}^\complex)$ to $\complex$ is of the form $\chi_{\Lambda}$ for some $\Lambda\in (\Lie{h}^\complex)^*$. Moreover, if $\chi_{\Lambda_1}=\chi_{\Lambda_2}$, then $\Lambda_1=w\Lambda_2$ for some $w\in W(\Lie{h}^\complex: \Lie{g}^\complex)$.
\end{Theo}
This in turn gives us more information about discrete series in \Cref{discrete series} and principal series, namely those series induced from parabolic subgroup in the way as in \Cref{principal series}, as listed below:
\begin{Theo}\label{discrete series2}
	\textnormal{\cite[Theorem~9.20 \& Corollary 12.22]{knapp2016}} Let $G$ and the Harish-Chandra parameter $\lambda$ as in \Cref{discrete series}, then for $\lambda\in (i\Lie{b})^*$ nonsingular and $\lambda+\delta_G$ analytically integral, we have the associated discrete representation $\pi_\lambda$ has the following properties:
	\begin{enumerate}
		\item $\pi_\lambda$ has infinitesimal character $\chi_\lambda$;
		\item $\pi_\lambda|_K$ contains with multiplicity one the $K$-type with highest weight $\Lambda=\lambda+\delta_G-2\delta_K$;
		\item If $\Lambda'$ is the highest weight of a $K$-type in $\pi_\lambda|_K$, then $\Lambda'$ is of the form: $\Lambda'=\Lambda+\sum_{\alpha\in \Delta^+}n_\alpha\alpha$ for integers $n_\alpha\geq 0$.
		\item Any given $K$-type $\mu$ occurs in only finitely many discrete series;
		\item The trivial $K$-type appears in no discrete series unless $G$ is compact.
	\end{enumerate}
\end{Theo}
\begin{Rmk}\label{infichar of cpt grp}
	In particular, when $G$ is itself compact, we see the infinitesimal character of an irreducible representation $V$ is $\Lambda+\delta_G$ with $\Lambda$ the highest weight of $V$. This can either seen from the theorem above or as a product of Theorem of the Highest Weight (c.f.\cite[Theorem~5.5]{knapp2013}).
\end{Rmk}
\begin{Prop}\label{induced infi character}
	\textnormal{\cite[Proposition~8.22]{knapp2016}} Let $MAN$ be a parabolic subgroup of $G$, with $\Lie{t}$ a $\theta$-stable Cartan subalgebra of $\Lie{m}$, and let $\sigma$ be an irreducible unitary representation $V$ of $M$ with infinitesimal character $\lambda_{\sigma}$ with respect to $\Lie{t}^\complex$. Now if $\nu\in (\Lie{a}^\complex)^*$, then the induced representation $\pi_{\sigma, \nu}$ in \Cref{principal series} has infinitesimal character $\lambda_\sigma+\nu\in ((\Lie{a}\oplus \Lie{t})^\complex)^*$.
\end{Prop}
These additional information, together with \Cref{borel1.4.2}, then gives us our third vanishing theorem:
\begin{Theo}\label{borel2.5.3}
	\textnormal{\cite[Chapter~II, Theorem~5.3]{borel2013}} Given $(\pi_\lambda,V_\lambda)$ be a discrete series representation with Harish-Chandra parameter $\lambda$. Let $H$ be the $(\Lie{g},K)$-module of $K$-finite vectors in $V$, and $(\sigma,E)$ be an irreducible finite-dimensional representation of $G$. Then:
	\begin{enumerate}
		\item If the highest weight of $(\sigma, E)$ relative to $\Delta^+$ is not $\lambda-\delta_G$, then $\Ext^i_{\Lie{g}, \Lie{k}}(E,H)=0$ for all $i$;
		\item If the highest weight of $(\sigma,E)$ is $\lambda-\delta_G$, then $\dim \Hom_{\Lie{k}}(\Lambda^i\Lie{p}\otimes E, H)=1$ if $i=\frac{\dim(G/K)}{2}$; and it vanishes if otherwise.
	\end{enumerate}
\end{Theo}
\begin{Rmk}
	As a side remark, we note an irreducible finite-dimensional representation $V$ of $\Lie{g}$, which is of course admissible, has infinitesimal character $\lambda=\Lambda+\delta_G$, with $\delta_G$ the half sum and $\Lambda$ the highest weight of $V$. This is a byproduct of Harish-Chandra map(c.f. \cite[Proposition~5.42ff]{knapp2013})
\end{Rmk}
\begin{proof}
	The first part is a direct consequence of \Cref{borel1.4.2} and \Cref{discrete series2}. Since $\chi(\pi_\lambda)=\lambda$ and $\chi_\sigma=\Lambda_\sigma+\delta_G$, where $\Lambda_\sigma$ is the highest weight of $(\sigma,E)$.
	\par As for the second part, first note:
	\begin{equation}
	\Ext^i_{\Lie{g,k}}(E,H)=H^i(\Lie{g,k};E^*\otimes H)=\Hom_{\Lie{k}}(\Lambda^i\Lie{p}\otimes E, H)
	\end{equation}
	To compute the dimension of right hand side, we first observe the weights of $\Lambda^i\Lie{p}^\complex\otimes E$. \par Let $\Delta_n:=\Delta-\Delta_K$ the set of \textbf{noncompact roots} and define $\Delta^+_n=\Delta_n\cap \Delta^+$. By considering the adjoint representation of $\Lie{g}$ on $\Lambda^i\Lie{p}^\complex$, since $\ad_{\Lie{b}}$ preserves Cartan decomposition in our case, we see the weights are sums of noncompact roots. Consequently by rewriting as difference of sums of positive roots and take $\delta_n=\delta_G-\delta_K$, we see the weights of $\Lambda^i\Lie{p}^\complex$ are of the form $2\delta_n-Q$ with $Q$ the sum of elements in $\Delta_n$, the highest weight module being $\Lambda^q\Lie{p}^\complex$ of multiplicity one by basic properties of root space decompositions. (c.f.\cite[Proposition~4.1]{knapp2016}) For $E$ we have a similar result. By Theorem of Highest Weight (c.f.\cite[Theorem~5.5]{knapp2013}) we see the weights of $E$ are $\lambda-\delta_G-Q'$ for some $Q'$ the sum of positive roots, with the highest weight $\lambda-\delta_G$ of multiplicity one. 
	\par Consequently, we see only the weights of $\Lambda^i\Lie{p}^\complex \otimes E$ are $2\delta_n+\lambda-\delta_G-Q''=\lambda+\delta_G-2\delta_K-Q''$, with $Q''=0$ if $i=q$, and in such case the height weight $\lambda+\delta_G-2\delta_K$ of multiplicity one. Now from \Cref{discrete series2} and the fact $K$-homomorphism preserves $K$-type, that 
	\begin{equation}
	\dim\Hom_K(\Lambda^q\Lie{p}^\complex\otimes E,V_\lambda)=1
	\end{equation}
	 where $V_\lambda$ is the discrete series of $G$ with infinitesimal character $\lambda$, and have unique $K$-type of highest weight $\lambda+\delta_G-2\delta_K$.
\end{proof}
Lastly we want to calculate the $(\Lie{g,k})$-cohomology for induced representation $\pi_{\lambda, i\nu}$ as appeared in \nameref{plancherel}. This part is rather technical, which involves a use of Hochschild-Serre spectral sequence \cite[Theorem~6.5]{borel2013} to cuspidal parabolic subgroup $\Lie{p}$. For the simplicity of discussion we omit the details here and satisfied with a full statement without proof. For details the reader is referred to \cite[Chapter~III, Theorem~3.3 \&Theorem 5.1]{borel2013}.
\par We again begin with a $\Theta$-stable Cartan subalgebra $\Lie{h}$ of $\Lie{g}$, we consider $\Lie{t}:=\Lie{h}\cap \Lie{k}$ and $\Lie{a}:=\Lie{h}\cap\Lie{p}$ the Cartan subalgebra of $\Lie{k}$ and $\Lie{p}$ respectively. Recall now \Cref{induced infi character} shows an induced representation $\sigma\otimes \exp \nu\otimes 1$ from parabolic subgroup $P=MAN$ can be written as an element $\lambda_\sigma+\nu$ of $(\Lie{h}^\complex)^*$, with $\lambda_\sigma$ the infinitesimal character of $\sigma$. Take now $W_\Lie{g}:=W(\Lie{h}^\complex:\Lie{g}^\complex)$, and we define 
\begin{equation}\label{def of Xi}
\Xi:=\{\lambda_{\sigma}\in (\Lie{b}^\complex)^*\mid -\delta_G\in W_\Lie{g}\cdot (\lambda_{\sigma}+\nu)\}
\end{equation}
where $W_\Lie{g}\cdot$ denotes the adjoint action of Weyl group on $(\Lie{h}^\complex)^*$. With slight abuse of notation, we can identify $\Xi$ as a subset of all discrete series of $M$.
\begin{Theo}\label{borel3.5.1}
	\textnormal{\cite[Chapter~III,Theorem~5.1]{borel2013}} Let $(\pi_{\lambda, i\nu},H^{\lambda, i\nu})$ be the induced representation in \nameref{plancherel}. Then $H^*(\Lie{g}, K;H^{\lambda,i\nu}_K)\neq\{0\}$ only if $P$ is fundamental, $\nu=0$, and $\lambda\in \Xi$, and:
	\begin{equation}
	\dim H^p(\Lie{g}, K;H^{\lambda,0}_K)=\begin{cases}
	{m \choose p-\frac{n-m}{2}} \qquad &\text{if $p\in [\frac{n-m}{2}, \frac{n+m}{2}]$}\\
	0 \qquad &\text{if otherwise.}
	\end{cases}
	\end{equation}
\end{Theo}
\pagebreak


\section{$L^2$-invariants of Symmetric Spaces}
From now on let $X=G/K$ be a Riemannian symmetric space of the noncompact type. Here $G$ is a real linear connected semisimple Lie group without compact factors. Let $\Gamma$ be a group acts freely, properly and cocompactly  on $X$, then $\Gamma\backslash X$ is a compact locally symmetric space. Moreover, we can identify $\Gamma$ with a torsion-free \textbf{lattice} of $G$, i.e., a discrete subgroup that acts cocompactly. Then we have the following vanishing theorem:
\begin{Theo}\label{olbrich1.1}
	\textnormal{\cite{olbrich2002}} Let $n=\dim X$ and $m(X):=\rk_\complex(G)-\rk_\complex(K)$ be the \textbf{fundamental rank} of $G$. Then:
	\begin{enumerate}[label*={\Roman*}]
		\item $b_p^{(2)}(X; \vna(\Gamma))\neq 0$ if and only if $m(X)=0$ and $p=\frac{n}{2}$. In this case,
		\begin{equation}
		b^{(2)}_{\frac{n}{2}}(X)=(-1)^{\frac{n}{2}}\chi(X)=\frac{\vol(\Gamma\backslash X)}{\vol(X^d)}\chi(X^d)
		\end{equation}
		\item $\alpha_p(X; \vna(\Gamma))\neq \infty^+$ if and only if $m>0$ and $p\in [\frac{n-m}{2}+1, \frac{n+m}{2}]$. Within this range, 
		\begin{equation}
		\alpha_p(X; \vna(\Gamma))=m
		\end{equation}
		\item $\rho^{(2)}(X)\neq 0$ if and only if $m(X)=1$.
		\item Suppose further $m(X)=1$, then $X=X_0\times X_1$, where $X_0$ is a symmetric space of non-compact type with $m(X_0)=0$, and $X_1=X_{p,q}:=SO(p,q)^0/SO(p)\times SO(q)$ for $p,q$ odd or $X_1=SL(3, \real)/SO(3)$. In such case, the $L^2$-torsion of $X$ is correlated to the volume of $\Gamma\backslash X$ by:
		\begin{equation}
		\rho^{(2)}(X)=\vol(\Gamma\backslash X)\cdot T^{(2)}(X)
		\end{equation}
		where $T^{(2)}(X)$ is a constant given by:
		\begin{enumerate}
			\item Let $X^d_0$ the compact dual of $X_0$, then:
			\begin{equation}
			T^{(2)}(X):=(-1)^{\dim(X_0)/2}\cdot \frac{\chi(X^d_0)}{\vol(X_0^d)}\cdot T^{(2)}(X_1)
			\end{equation}
			\item $T^{(2)}(X_{p,q})=(-1)^{\frac{pq-1}{2}}\chi(X^d_{p-1, q-1})\frac{\pi C_{p+q-1}}{\vol(X^d_{p,q})}$, where $C_{p+q-1}$ is a constant defined as:
			\begin{equation}
			C_{d}:=\sum_{j=0}^{n-1}(-1)^{n+j+1}\frac{n!}{(2n)!\pi^n}{2n \choose j}\cdot \sum_{k=0}^{n}K^n_{k,j}\cdot \frac{(-1)^{k+1}}{2k+1}\cdot (n-j)^{2k+1}
			\end{equation}
			with $K_{k,j}^n$ the integer coefficients at degree $2k$ of the polynomial $P^n_j(x):=\frac{\prod_{i=0}^{n}(x^2+i^2)}{x^2+(n-j)^2}$.
			\item If $X_1=SL(3, \real)/SO(3)$, then: 
			\begin{equation}
			T^{(2)}(SL(3, \real)/SO(3))=\frac{\pi}{2\cdot\vol(X^d)}
			\end{equation}
			If the invariant metric on $X$ is induced from twice the trace form of the standard representation of $\Lie{sl}(3, \real)$, then $\vol(X^d)=4\pi^3$, and hence $T^{(2)}(X)=\frac{1}{8\pi^2}$.
		\end{enumerate}
	\end{enumerate}
\end{Theo}


\pagebreak

We begin our proof of our theorem by first reinterpreting \nameref{plancherel} using representation data. Denote $\hat{G}$ as the equivalence classes of irreducible unitary representations, we first want to recall the abstract Plancherel formula, the idea of which dates back to von Neumann:
\begin{Theo}\label{abstract Plancherel}
	\textnormal{\cite[Theorem~18.8.1]{dixmier1982}} Let $G$ be a linear connected semisimple group.\footnote{In the original text it assumes a larger class of groups, i.e., unimodular postliminal separable locally compact group, which in particular the class of our concern.} Let $\lambda$ be the left-regular representations of $G$. Then there exists a positive measure $\mu$ on $\hat{G}$ and an isomorphism $W$:
	\begin{equation}\label{absplancherel}
	W: L^2(G)\cong \int_{\hat{G}}H_\pi\otimes H_\pi^*\mass{\mu(\pi)} \qquad \lambda\cong \int_{\hat{G}}\pi\otimes \id_{H^*_\pi}\mass{\mu(\pi)}
	\end{equation}
	where $H^*$ is the dual Hilbert space of $H$. There is a complete analog for right regular representations.
\end{Theo}
Now we recall \nameref{Kuga}, which in our case $\tau=\lambda$ is the left-regular representation. Hence under the identification $L^2\Omega^p(X)\cong[L^2(G)\otimes \Lambda^p\Lie{p}^*]^K$, we see $\Delta_p=-[\lambda(\Omega)\otimes \id_{\Lambda^p\Lie{p}^*}]^K$, with right hand side the induced map at $K$-invariant level. This together with \ref{absplancherel} gives:
\begin{equation}\label{casimir=laplacian}
 \Delta_p=-\int_{\hat{G}}^{}\pi(\Omega)\otimes \id_{[H^*_\pi\otimes \Lambda^p\Lie{p}^*]^K}\mass{\mu(\pi)}
\end{equation}
Now we turn to the explicit formula as given in \nameref{plancherel}, note it in particular says $\mu$ is supported in a subset of $\hat{G}$ which is parametrized by principal series induced from parabolic subgroups. Moreover, by \Cref{knapp8.7} we see $\Omega$ acts on each element of $\hat{G}$ as scalars, whence we can take $\Omega$ as a scalar function on $\hat{G}$. 
\par More explicitly write $X=X_\Lie{a}+X_\Lie{m}+X_\Lie{n}\in \Lie{a}\oplus \Lie{m}\oplus\Lie{n}$ with respect to the Langlands decomposition, and also $\Lie{n}$ acts trivially, we have:
\begin{equation}
\begin{split}
\pi_{\lambda, i\nu}(X)\circ f(1)=\frac{d}{dt}\circ f(\exp(tX)^{-1})\big|_{t=0}&=\frac{d}{dt}(e^{i\nu+\rho_\Lie{a}}\otimes \sigma\otimes 1)(\exp tX)\circ f(1)\big|_{t=0}\\
&=((i\nu+\rho_{\Lie{a}})(X_\Lie{a})+\sigma(X_\Lie{m}))f(1)
\end{split}
\end{equation}
Moreover, we note with regard to Killing form $B(-,-)$ we can choose a basis of $\Lie{m}$ and consequently construct the respective Casimir element $\Omega_M$ for $M$. Consequently we have:
\begin{equation}\label{olbrich6}
\pi_{\lambda, i\nu}(\Omega)=-\brac{i\nu, i\nu}-\brac{\rho_{\Lie{a}}, \rho_{\Lie{a}}}+\sigma(\Omega_M)=-\norm{\nu}^2-\norm{\rho_{\Lie{a}}}^2+\pi_{\lambda}(\Omega_M)
\end{equation}
where $\pi_{\lambda}$ is the discrete series of $M$ with infinitesimal $\lambda$. This together with \ref{casimir=laplacian}, gives the following lemma:
\begin{Lemma}\label{olbrich2.3}
	Using the notations and settings of \nameref{plancherel}, and take $P_j=M_jA_jN_j$ the Langlands decomposition of cuspidal parabolic subgroup $P_j$ constructed from the $\theta$-stable Cartan subalgebra $H_j$ (c.f. \Cref{cuspidal parabolic}), then we have:
	\begin{equation}
	\tr_{\vna(\Gamma)}e^{-t\Delta_p}=\vol(\Gamma\backslash X) \sum_{j=1}^s\sum_{\lambda\in \hat{M}_d}\int_{\Lie{a}^*}e^{-t(\norm{\nu}+\norm{\rho_{\Lie{a}}}-\pi_\lambda(\Omega_M))}\dim[H^{\lambda, i\nu}\otimes \Lambda^p\Lie{p}^*]^Kp_\xi(i\nu)\mass{\nu}
	\end{equation}
\end{Lemma}
\begin{proof}
	First note the heat kernel $e^{-t\Delta_p}(x,y)$ is $G$-invariant, since $G$-acts isometrically on $X=G/K$, hence we have $e^{-t\Delta_p}(xK, yK):=h_t^p(y^{-1}x)$ is a smooth function on $G$. We know from general theory that it is a Schwartz function, and it is $K$-bi-invariant,  i.e., $h_t^p\in [\mathcal{C}(G)\otimes \mathrm{End}(\Lambda^p)]^{K\times K}$. Moreover, for $f\in L^2\Omega^p(X)$, we have:
	\begin{equation}
	e^{-t\Delta_p}f(g_0)=\int_G h_t^p(g^{-1}g_0)f(g)\mass{g}
	\end{equation}
	Consequently we have $\tr e^{-t\Delta_p}(x,x)=\tr h_t^p(e)$ for any $x\in X$. On the other hand, by \ref{casimir=laplacian} we have:
	\begin{equation}
	e^{-\Delta_p}=\int_{\hat{G}}^{}e^{-t\pi(\Omega)}\otimes \id_{[H^*_\pi\otimes \Lambda^p\Lie{p}^*]^K}\mass{\mu(\pi)}
	\end{equation}
	Now from \ref{olbrich6} we derive the formula.
\end{proof}
\begin{proof}[\textbf{Proof of \Cref{olbrich1.1}.I}]
	We begin our proof on $L^2$-Betti numbers, which only depend on discrete spectrum. First observe that the $L^2$-eigenspaces of Laplacian coincide with all the twisted discrete series representation, i.e.:
	\begin{equation}
	L^2(X, \Lambda^pT^*X)_d\cong \bigoplus_{\pi\in \hat{G}_d}H_\pi \otimes [H_\pi^*\otimes \Lambda^p\Lie{p}^*]^K
	\end{equation}
	To see this, we first note $\Omega$ acts on discrete series and principal series as scalars. Moreover, as a by-product of Plancherel formula, we note each discrete series take non-zero Plancherel measure, whence all discrete series are eigenspaces of Laplacian. On the other hand, by \ref{olbrich6} we see for fixed eigenvalue $\lambda$, we have only finite values of $\nu$, the principal series $H^{\sigma, i\nu}$ of which the Casimir operator $\Omega$ takes value $\lambda$. Now \Cref{normalization} says $\mass{\nu}$ is Lebesgue measure, whence all such spaces are of measure zero. Summing up, we have the desired result. 
	\par Now by 4. of \Cref{olbrich3.1} it suffices to consider all those discrete series $H_\pi$ on which $\Omega$ acts on $H_\pi$ as trivially. This in particular says $0$ is the only discrete spectrum. Now the non-vanishing result of such discrete series is given by \Cref{borel2.5.3}, which we take $E=\complex$ the trivial representation of $G$, which has highest weight $0$. Arguing backwards, we have:
	\begin{equation}\label{olbrich9}
	\dim[H_\pi^*\otimes \Lambda^p\Lie{p}^*]^K=\begin{cases}
	1 \qquad &\text{if $p=\frac{\dim G-\dim K}{2}$ and $\chi_\pi=\delta_G$}\\
	0 \qquad & \text{if otherwise}	
	\end{cases}
	\end{equation}
	In the case when $\hat{G}_d\neq \emptyset$, such discrete series representation always exists, since one can trivially check $\delta_G$ is non-singular and $2\delta_G$ is analytically integral, that is the $L^2$-kernel at degree $p:=\frac{\dim X}{2}$ always exist. Summing up, we have:
	\begin{equation}\label{olbrich(9)}
	b^{(2)}_{n/2}(X;\vna(\Gamma))=(-1)^{\frac{\dim X}{2}}\chi^{(2)}(X)
	\end{equation}
	where $\chi^{(2)}(X):=\sum_{i=0}^n(-1)^{i}b_i^{(2)}(X)$ the $L^2$-Euler characteristic. Now we claim $L^2$-Euler characteristic of a free finite $\Gamma$-CW complex coincide with the cellular $L^2$-Euler characteristic of $\Gamma\backslash X$. This follows from the fact von Neumann dimension is additive with respect to exact sequences, and we could hence express the cellular Euler characteristic in terms of the alternating sum of number of $p$-cells, and note $\dim_{\vna(\Gamma)}(C_p(X))=\dim_\complex (C_p(\Gamma\backslash X; \complex))$. Consequently, we have:
	\begin{equation}
	b^{(2)}_{\frac{n}{2}}(X)=(-1)^{\frac{\dim X}{2}}\chi(\Gamma \backslash X)=\frac{\vol(Y)}{\vol(X^d)}\chi(X^d)
	\end{equation}
	where the last equality follows form \nameref{hirzebruch}. Hence we have finished the proof of part I.
\end{proof}
As a byproduct of this proof, we have yielded some discrete spectral information, which we summarized here:
\begin{Cor}
	The discrete spectrum of $\Delta_p$ on $L^2(X, \Lambda^pT^*X)$ is empty unless $m(X)=0$ and $p=\dim X/2$. In this case $0$ is the only eigenvalue of $\Delta_p$. \qed
\end{Cor}
\begin{Rmk}
	Recall \Cref{F=dim} that $\mass{F^\Delta_p}$ is a Borel measure, hence by decomposition of measure and the fact that $\Delta_p$ is essentially self-adjoint, so there is no singular spectrum: 
	\begin{equation}
	\mass{F^\Delta_p}=\mass{F_p^{\Delta_{\mathrm{disc}}}}+\mass{F_p^{\Delta_\mathrm{cont}}}
	\end{equation}
	where $\Delta_{\mathrm{disc}}$ and $\Delta_\mathrm{cont}$ are respectively the discrete and continuous spectrum of the Laplacian. Then the above corollary gives in particular:
	\begin{equation}
	\mass{F^\Delta_p}=\chi_0 F^\Delta_p(0)+\mass{F_p^{\Delta_\mathrm{cont}}}
	\end{equation}
	with $\chi_0$ the Dirac measure at $0$. 
\end{Rmk}
We now prove the part on Novikov-Shubin invariants. By \Cref{unique closed extension} we have $(\delta^p_{\min})^\perp=((d^{p-1}_{\min})^\perp)^*$, this together with \Cref{2.3} and \Cref{2.11} gives $\lambda\geq 0$:
\begin{equation*}
F_p(X)(\lambda):=F((d^{p\perp}_{\min}))(\lambda)=F((d^{p\perp}_{\min})^*d^{p\perp}_{\min}))(\lambda^2)=F((\delta^{p+1}d^p)^\perp_{\min})(\lambda^2)
\end{equation*}
Moreover, we recall $F(f)$ and $F(f)-b^{(2)}(f)$ give the same Novikov-Shubin invariant. For this sake we suffices to study just the asymptotic behaviour of $F_{p}(X)(\lambda)-b^{(2)}_{p}(X)$. 
\par The next is to observe $(\Delta_p)_{\min}|_{\im(d^{p-1}_{\min})^\perp}=(\delta^{p+1}d^p)_{\min}^\perp$. Now since $(\ker d^{p})^\perp\subseteq (\im d^{p-1})^\perp$, we conclude from above discussions that:
\begin{equation}
\alpha_{p+1}(X)=\alpha(F(d^{p\perp}_{\min}))=2\cdot \alpha\Big(F(\delta^{p+1}d^p)_{\min}^\perp-b^{(2)}_p(X)\Big)=2\cdot \alpha\Big(F(\Delta_p)_{\min}|_{(\ker d^p_{\min})^\perp}\Big) 
\end{equation}
Denote $\Delta_p^\perp=(\Delta_p)_{\min}|_{(\ker d^p_{\min})^\perp}$. We see $\Delta_p^\perp$ is again an self-adjoint operator. Using the same argument of \Cref{3.138} and \Cref{3.136} we define analogously $\theta_p^\perp(t):=\int_{0}^\infty e^{-t\lambda}\mass{F(\Delta_p^\perp)(\lambda)}$ and see $\theta_p^\perp(t)=\tr_{\vna(\Gamma)}e^{-t\Delta_p^\perp}$. Moreover, if $\alpha_{p+1}\neq\infty^+$, i.e, when there is no gap on the spectrum of $\Delta^\perp_p$ around $0$, we have:
\begin{equation}\label{NS=trace}
-\frac{1}{2}\cdot\alpha_{p+1}(X)=\lim_{t\to \infty}\frac{\theta_p^\perp(t)-F(\Delta^\perp_p)(0)}{\ln(t)}=\lim_{t\to \infty}\frac{\tr_{\vna(\Gamma)}e^{-t\Delta_p^\perp}-F(\Delta^\perp_p)(0)}{\ln(t)}
\end{equation}
So now in view of \Cref{abstract Plancherel} it suffices to consider what is the restriction of domain on the representation side. By mimicking the argument of \Cref{olbrich2.3} we respectively associate with $e^{-t\Delta_p^\perp}$ a function $h_t^{p\perp}\in [\mathcal{C}(G)\otimes \mathrm{End}(\Lambda^p\Lie{p}^*)]^{K\times K}$.
\par Recall the isomorphism $L^2(X, \Lambda^paT^*X)\cong [L^2(G)\otimes \Lambda^p\Lie{p}^*]^K$ and consider the left regular representation of $G$ on $L^2(G)$. This allows us to define on the chain complex of $L^2$ $p$-forms $d$ and $\delta$ as in \ref{borel2.1.5(5)} and \ref{borel2.2.3(1)}.\footnote{One should caution that the differential and co-differential are defined formally, since $L^2(G)^K$ is itself not $K$-locally finite, hence $\{L^2(X, \Lambda^pT^*X)\}_p$ does not give a chain of $(\Lie{g}, K)$-modules.} Altogether we have a commutative diagram (at a formal level):
\begin{equation}
\begin{tikzcd}[column sep=huge]
(\ker d^p_{\min})^\perp \arrow[r,"\Delta_p^\perp"] \arrow[d, hook] & (\ker d^p_{\min})^\perp \arrow[d, hook]\\
(L^2(G)\otimes \Lambda^p \mathfrak{p}^*)^K \arrow[r, swap, "(\lambda(G)\otimes \id_{\Lambda^p\Lie{p}^*})^K"] & (L^2(G)\otimes \Lambda^{p}\mathfrak{p}^*)^K 
\end{tikzcd}
\end{equation}
Consequently we have $\Delta_p^\perp=P_p\circ [\lambda(G)\otimes \id_{\Lambda^p\Lie{p}^*}]^K$ with $P_p:L^2\Omega^p(X)\twoheadrightarrow \ker(d^{p}_{\min})^\perp$ the orthogonal projection. This together with the above commutative diagram and \ref{casimir=laplacian} gives:
\begin{equation}
\Delta_p^\perp=\int_{\hat{G}}\pi(\Omega)\otimes P_{p,\pi}\mass{\mu(\pi)}
\end{equation}
where $P_{p, \pi}:[H_\pi^*\otimes \Lambda^{p}\Lie{p}^*]^K\to \big(\ker d^{p}_\pi:[H_\pi^*\otimes \Lambda^{p}\Lie{p}^*]^K\to [H_\pi^*\otimes \Lambda^{p+1}\Lie{p}^*]^K\big)^\perp$ the orthogonal projection. Consequently, we argue like \Cref{olbrich2.3} and get:
\begin{equation}
\tr_{\vna(\Gamma)}e^{-t\Delta_p^\perp}=\vol(Y)\sum_{P}\sum_{\lambda\in \hat{M}_d}\int_{\Lie{a}^*}^{}e^{-t(\norm{\nu}^2+\norm{\rho_{\Lie{a}}}^2-\pi_\lambda(\Omega_M))}\dim[\im P_{p, (\lambda, i\nu)}]p_{\lambda}(i\nu)\mass{\nu}
\end{equation}
Note when $m=0$, we have the compact cuspidal parabolic group $P=G$. In such case we deduce from \ref{olbrich9} all $d^p_{\lambda}=0$, whence we can refine the formula to be: 
\begin{equation}\label{olbrich11}
\tr_{\vna(\Gamma)}e^{-t\Delta_p^\perp}=\vol(\Gamma\backslash X)\sum_{P\neq G}\sum_{\lambda\in \hat{M}_d} \int_{\Lie{a}^*}^{}e^{-t(\norm{\nu}^2+\norm{\rho_{\Lie{a}}}^2-\pi_\lambda(\Omega_M))}\dim[\im P_{p, (\lambda, i\nu)}]p_{\lambda}(i\nu)\mass{\nu}
\end{equation}
\begin{proof}[\textbf{Proof of \Cref{olbrich1.1}.II}] First note $d^p_{\lambda, i\nu}|_{(\ker d^{p-1}_{\lambda, i\nu})^\perp}$ is injective, so:
	\begin{equation}
	\dim(\im P_{p, (\lambda,i\nu)})=\dim (\im d^p_{\lambda, i\nu})
	\end{equation} 
In view of \ref{olbrich11} it suffices to consider the case when $d^p_{\lambda, i\nu}$ is nontrivial. For fixed $\lambda$,there are two cases: 
\par The first case is if for all $\nu\in \Lie{a}^*$, we have $d^p_{\lambda, i\nu}\neq 0$ and $H^p(\Lie{g}, K, H^{\lambda, i\nu}_K)=H^p([H^{\lambda, i\nu}\otimes \Lambda^\star\Lie{p}^*]^K, d_{\lambda, i\nu})= 0$. In such case we have by 4. of \Cref{olbrich3.1} and \ref{olbrich6} that:
\begin{equation}
\inf_{\nu\in \Lie{a}^*}\pi_{\lambda, i\nu}(\Omega)=\inf_{\nu\in \Lie{a}^*}(\norm{\nu}^2+\norm{\rho_{\Lie{a}}}^2-\pi_\lambda(\Omega_M))\neq 0 \qquad \text{for all $\nu\in \Lie{a}^*$}
\end{equation}
	Now since $\norm{\nu}^2+\norm{\rho_{\Lie{a}}}^2-\pi_\lambda(\Omega_M)$ is a quadratic polynomial of $\norm{\nu}$ which obviously takes value greater than $0$, we see it is closed map and consequently $\inf_{\nu\in \Lie{a}^*}\norm{\nu}^2+\norm{\rho_{\Lie{a}}}^2-\pi_\lambda(\Omega_M)>0$. In view of \ref{olbrich11} this part is upper bounded when $t$ tends to infinity, hence not contributed to our computation of $\alpha_{p+1}(X)$. In particular, we see if for all proper parabolic subgroup $P$ and for all respective $\lambda$ the cohomology vanishes, then we have a gap in spectrum and $\Delta^\perp_p$, which means $\alpha_{p+1}(X)=\infty^+$.
\par So it suffices to consider the second case, that is those $\lambda$ for which $H^p(\Lie{g}, K, H^{\lambda, i\nu}_K)\neq 0$ for some $\nu\in \Lie{a}^*$. By \Cref{borel3.5.1} this only happens to the case when $P$ is fundamental and $\lambda\in \Xi$. In this case \Cref{olbrich3.1} shows $\pi_{\lambda, 0}(\Omega)=0$. Moreover, inspecting the definition of induced representation one see $H^{\lambda, i\nu}$ and $H^{\lambda, 0}$ are differed by scaling, and have same action of $K$ that is governed by $K\cap M$. Hence they have the same $K$-finite dimension, that is:
\begin{equation}
\dim[H^{\lambda,i\nu}\otimes \Lambda^p\Lie{p}^*]^K=\dim[H^{\lambda,0}\otimes \Lambda^p\Lie{p}^*]^K=\dim H^p(\Lie{g}, K, H^{\lambda, 0}_K)
\end{equation}
Moreover, the right hand side is computed explicitly in \Cref{borel3.5.1}. But on the other hand, we see left hand side of the equation have vanishing cohomology, hence $\dim[H^{\lambda,i\nu}\otimes \Lambda^p\Lie{p}^*]^K=\dim (\im d^p_{\lambda, i\nu})+\dim (\im d^{p-1}_{\lambda, i\nu})$ for $\nu\neq 0$. Hence computing inductively, we have:
\begin{equation}
\dim(\im P_{p, (\lambda, i\nu)})=\begin{cases}
{m-1 \choose p-\frac{n-m}{2}} \qquad &\text{if $p\in [\frac{n-m}{2}, \frac{n+m}{2}-1]$}\\
0 \qquad &\text{if otherwise}
\end{cases}
\end{equation}
Subbing this result in \ref{olbrich11} and summarize all the discussion, we suffices to compute the power of the leading term of:
\begin{equation*}
\vol(\Gamma\backslash X){m-1 \choose p-\frac{n-m}{2}}\sum_{\lambda\in \Xi}^{}\int_{\Lie{a}^*}^{}e^{-t\norm{\nu}^2}p_{\lambda}(i\nu)\mass{\nu}
\end{equation*}
Now it suffices to ascertain the Plancherel measure $p_{\lambda}(i\nu)$ for $\xi\in \Xi$. Since $P$ is fundamental, we recall here the second part of \nameref{plancherel} that:
\begin{equation}
p_\lambda(i\nu)=C(-1)^{\frac{\dim \Lie{n}_1}{2}}\prod_{\alpha\in \Delta^+}\frac{\brac{\lambda+i\nu, \alpha}}{\brac{\rho_{\Lie{g}},\alpha}}
\end{equation}
is a polynomial of degree $\dim \Lie{n}$. Hence $p_{\lambda}(i\nu)$ is nonnegative for $\nu\in \Lie{a}^*$. Now by the definition of $\Xi$ in \ref{def of Xi} we see $\lambda\in W_\Lie{g}\cdot (-\delta_G)\cap \Lie{b}^*$, hence we have in particular $\prod_{\alpha\in \Delta^+}\brac{\lambda, \alpha}=\prod_{\alpha\in \Delta^+}\brac{\rho_{\Lie{g}}, \alpha}$ when $\lambda\in \Xi$. In such case $p_{\lambda}(0)=\pm C$. Since $p_\lambda$ is nonnegative on $i\Lie{a}^*$, we have $p_\lambda(0)>0$. 
\par To finish our proof, now we suffices to break $p_\lambda=\sum_{k=0}^{\dim\Lie{n}/2}p_{\lambda, 2k}$ into homogeneous polynomials. Now since the exponential part only depends on the radius $\norm{\nu}$, which we use spherical coordinate, and note $\nu\in \real^{\dim \Lie {a_p}}$. When $P$ is fundamental, $\dim\Lie{a_p}=m$. So we have the following change of coordinate by taking $R=\norm{\nu}$:
\begin{equation*}
\begin{split} \int_{\Lie{a}^*}^{}e^{-t\norm{\nu}^2}p_{\lambda}(i\nu)\mass{\nu}&=\sum_{k=0}^{\dim\Lie{n}/2}\int_{\norm{\nu}=1}^{}p_{\lambda, 2k}(\nu)\mass{\nu}\int_{0}^{\infty}e^{-t\cdot R^2}R^{2k}(R^{m-1}\mass{R})\\
&=\sum_{k=0}^{\dim\Lie{n}/2}C_{\lambda, k}\cdot t^{-\frac{m+2k}{2}}
\end{split}
\end{equation*}
Moreover, we see for all $\lambda\in \Xi$, $C_{\lambda, 0}>0$, since $\int_{\norm{\nu}=1}^{}p_{\lambda}(0)\mass{\nu}>0$. Hence we see the leading term of $\tr_{\vna(\Gamma)}e^{-t\Delta^\perp_p}$ is a non-zero multiple of $t^{\frac{m}{2}}$ as $t\to \infty$ and $p\in [\frac{n-m}{2}, \frac{n+m}{2}-1]$. So now bring this back to \ref{olbrich11} and together with \ref{NS=trace} we have the desired result that $\alpha_{p+1}(X)=m$ in this range.
\end{proof}

Now we started proving the last of three $L^2$-invariants, namely the $L^2$-torsion. First recall \nameref{poincare} that the $L^2$-torsion of an even-dimensional manifold vanish. Consequently, we suffices to consider the case $n$ is odd. Since $n$ and $m$ have the same parity, we have $m$ is odd and is in particular greater than $0$. Hence $\Delta_p$ have no discrete spectrum by I of \Cref{olbrich1.1}. Retaining the notation of \ref{3.131}, we have for $T^{(2)}:=\frac{\rho^{(2)}(X)}{\vol(\Gamma\backslash X)}$:
\begin{equation}\label{olrbichtorsion}
T^{(2)}(X)=\frac{d}{ds}\Big|_{s=0}\Big(\frac{1}{\Gamma(s)}\int_{0}^{\epsilon}k_X(t)\cdot t^{s-1}\mass{t}\Big)+\int_{\epsilon}^{\infty}k_X(t)\cdot t^{-1}\mass{t}
\end{equation}
where:
\begin{equation}
k_X(t):=\frac{1}{2\vol(\Gamma\backslash X)}\sum_{p =0}^n(-1)^p\cdot p\cdot \theta_p^\perp(t)=\frac{1}{2\vol(\Gamma\backslash X)}\sum_{p =0}^n(-1)^p\cdot p\cdot \theta_p(t) 
\end{equation}
is the main part we would like to investigate. To start with, we want to use a slightly different version of \ref{olbrich2.3} by stressing the parabolic subgroup. Note $K_M:=K\cap M$ is the maximal compact subgroup of $M$. Now we recall the induced representation $H^{\sigma, \nu}$. The restriction of $G$ to $K$ is gives a dense subspace:
\begin{equation}
\{F\in C(K, V^\sigma)\mid F(km)=\sigma(m)^{-1}F(k) \quad \text{for }k\in K, m\in K_M \}
\end{equation}
Note this restriction is one-to-one since $G=KMAN$. Hence we can also view $H^{\sigma,\nu}=\ind^K_{K_M}\sigma$ an induced representation of $K$ from $K_M$. Consequently, by Frobenius Reciprocity we have:
\begin{equation}
[H^{\lambda, i\nu}\otimes \Lambda^p\Lie{p}^*]^K\cong [V^\lambda\otimes \Lambda^p\Lie{p}^*]^{K_M}
\end{equation}
Taking this into \Cref{olbrich2.3} we have a new formula of $\tr_{\vna(\Gamma)}e^{-t\Delta_p}$:
\begin{equation}\label{olbrich8}
\tr_{\vna(\Gamma)}e^{-t\Delta_p}=\vol(\Gamma\backslash X) \sum_{j=1}^s\sum_{\lambda\in \hat{M}_d}\int_{\Lie{a}^*}e^{-t(\norm{\nu}+\norm{\rho_{\Lie{a}}}-\pi_\lambda(\Omega_M))}\dim[V^\lambda\otimes \Lambda^p\Lie{p}^*]^{K_M}p_\lambda(i\nu)\mass{\nu}
\end{equation}
Moreover, from 2. and 4. of \Cref{discrete series2} we see there are only finitely many pairs $(P, \lambda)$ for which $[H^{\lambda, i\nu}\otimes \Lambda^p\Lie{p}^*]^K\cong [V^\lambda\otimes \Lambda^p\Lie{p}^*]^{K_M}\neq \{0\}$.

\bigskip

We begin our proof with $m\neq 1$ implies the vanishing of $\rho^{(2)}(X)$. In view of the preceding discussions it suffice to concern with the case $m\geq 2$. 
\par Given a parabolic subgroup $P=MAN$,  we choose a unit vector $Y\in \Lie{a}$, and denote its orthogonal complement in $\Lie{p}$ to be $\Lie{p}_Y$. Recall $\Lie{m}$ normalizing $\Lie{a}$, hence we have $\Lie{p}^*=\real\cdot Y^*\oplus \Lie{p}^*_Y $ is a decomposition of $K_M$-module. Therefore, in the representation ring of $K_M$, we have since $\Lambda^l Y^*=0$ for any $l\geq 2$, that:
\begin{equation}\label{Moscovici2.1}
\begin{split}
\sum^{n}_{p=1}(-1)^p\cdot p\cdot \Lambda^p\Lie{p}^* &= \sum_{p=1}^n(-1)^p\cdot p\cdot [\Lambda^p\Lie{p}^*_Y\oplus \Lambda^{p-1}\Lie{p}^*_Y]\\
&=\sum^n_{p=1}(-1)^p\cdot p\cdot \Lambda^p\Lie{p}^*_Y+\sum^{n-1}_{p=0}(-1)^{p+1}\cdot (p+1) \Lambda^{p}\Lie{p}^*_Y\\
&=\sum^{n}_{l=0}(-1)^{l+1}\Lambda^l\Lie{p}_Y^*=\Lambda^\odd\Lie{p}_Y^*-\Lambda^\even\Lie{p}^*_Y
\end{split}
\end{equation}
In the case when $\dim \Lie{a}\geq 2$, then we can find another $H\in \Lie{a}\cap \Lie{p}_Y$. We consider the Clifford multiplication of $H$, that is: $c\ell(H):=\wedge H+ \llcorner H:\Lambda^\even \Lie{p}^*\leftrightarrow \Lambda^\odd\Lie{p}^*$. Since $M$ centralizes $H$, we have $c\ell(H)$ is homomorphism of $K_M$-modules. It is also an isomorphism by checking its action on bases. Hence $\Lambda^\even\Lie{p}^*$ and $\Lambda^\odd \Lie{p}^*$ are isomorphic $K_M$-modules. We then conclude from this and \ref{Moscovici2.1} for such $P$, $\sum(-1)^pp\Lambda^p\Lie{p}^*=0$. Moreover, from \Cref{hierachy of parabolic subgroup} we see all cuspidal parabolic groups $P$ has $\dim \Lie{a}\geq m$. Hence when $m>2$, we have $k_X(t)\equiv 0$ and consequently the $L^2$-torsion vanish.

\bigskip

From now on let $m=1$. From the discussions above we see the fundamental parabolic group $P=MAN$ is the only that of our concern, which has $\dim \Lie{a}=1$. In this case we have, as $K_M^0:=K\cap M_0$-modules $\Lie{p}_Y\cong \Lie{p}_\Lie{m}\oplus \Lie{n}$, where $\Lie{p}_\Lie{m}=\Lie{p}\cap \Lie{m}$. Consequently, subbing this into \ref{Moscovici2.1} we have as $K_M^0$-modules:
\begin{equation}\label{Moscovici2.3}
\sum^{n}_{p=0}(-1)^p\cdot p\cdot \Lambda^p\Lie{p}^*=\sum_{p=0}^{n}(-1)^{p+1}\Lambda^p(\Lie{p}_\Lie{m}^*\oplus \Lie{n}^*)=\sum_{l=0}^{\dim\Lie{n}}(-1)^{l+1}(\Lambda^{\even}\Lie{p}_\Lie{m}^*-\Lambda^\odd\Lie{p}_\Lie{m}^*)\otimes \Lambda^l\Lie{n}
\end{equation}
Since $P$ is fundamental, we can conclude that the discrete series of $M$ are exactly the representations induced by $M_0$.\footnote{We conceal much details here. By \cite[Proposition~12.32]{knapp2016} we have $\hat{M}_d$ are all induced representations of $\hat{M^\sharp}_d$, with $M^\sharp:=M_0Z_M=M_0F(B^-)$ (see \cite[Lemma 12.30(1)]{knapp2016}. When $P$ is fundamental, the respective Cartan group is compact, hence has no real roots, whence $M^\sharp=M_0$, hence our claim is true.} Consequently by Frobenius reciprocity, we have $[V^\lambda\otimes \Lambda^p\Lie{p}^*]^{K_M}\cong [V^\lambda_0\otimes \Lambda^p\Lie{p}^*]^{K_M^0}$, where $V^\lambda_0\in (\hat{M_0})_d$ is the respective discrete series representation of $M_0$ such that $V^\lambda=\ind^{M}_{M_0}(V^\lambda)$. 
\par Consequently \ref{Moscovici2.3} and \ref{olbrich8} altogether gives a concrete formula of $k_X(t)$:
\begin{equation}\label{olbrich15}
\begin{split}
k_X(t)&=\frac{1}{2}\sum^{\dim \Lie{n}}_{l=0}\sum_{\lambda\in \hat{M}_d}(-1)^{l+1}\dim [V^\lambda\otimes (\Lambda^{\even}\Lie{p}_\Lie{m}^*-\Lambda^\odd\Lie{p}_\Lie{m}^*)\otimes \Lambda^l\Lie{n}]^{K_M}Q_{\Lie{a}, \lambda}\\
&=\frac{1}{2}\sum^{\dim \Lie{n}}_{l=0}\sum_{\lambda\in \hat{M}_d}(-1)^{l+1}\dim [V^\lambda_0\otimes (\Lambda^{\even}\Lie{p}_\Lie{m}^*-\Lambda^\odd\Lie{p}_\Lie{m}^*)\otimes \Lambda^l\Lie{n}]^{K_M^0}Q_{\Lie{a}, \lambda}
\end{split}
\end{equation}
where $Q_{\Lie{a}, \lambda}=\int_{\Lie{a}^*}e^{-t(\norm{\nu}^2+\norm{\rho_{\Lie{a}}}^2-\pi_\lambda(\Omega_M))}p_\lambda(i\nu)\mass{\nu}$.
\par Lastly we see in the case $P$ is fundamental, $p_\lambda$ admits an explicit formula as in \Cref{plancherel}. We want to compute the constant $C$ this time. This can help us in deciding the $k_X$ for rather simple $X$s. To prepare for this, we choose positive root system $\Delta^+$ for $(\Lie{g}^\complex:\Lie{h}^\complex)$ and restrict it to $(\Lie{k}^\complex: \Lie{t}^\complex)$ to have a positive root system $\Delta^+_k$, where $\Lie{t}$ is the Cartan subalgebra of $\Lie{k_m}$, of $\Lie{k}$ and of $\Lie{m}$. We denote the respective half sum of positive roots to be $\rho_{G}$ and $\rho_{K}$ again. Now:
\begin{Lemma}\label{olbrich5.1}
	Let $P=MAN$ be the fundamental parabolic subgroup of $G$. Denote:
\begin{equation}
W_A:=\{k\in K\mid \Ad_k\Lie{a}=\Lie{a}\}/K_M \qquad S^d_A:=\{\exp(iX)K\mid X\in \Lie{a}\}\subset X^d
\end{equation}
then we have the constant $C$ in \nameref{plancherel} takes the following form:
\begin{equation}
\begin{split}
C=\frac{1}{|W_A|(2\pi)^{\frac{n+m}{2}}}\frac{\prod_{\alpha\in \Delta^+}\brac{\alpha, \rho_{G}}}{\prod_{\alpha\in \Delta^+_k}\brac{\alpha, \rho_{K}}}=\frac{1}{|W_A|}\frac{\vol(S^d_A)}{(2\pi)^m}\frac{1}{\vol(X^d)}
\end{split}
\end{equation}
\end{Lemma}
\begin{proof}
The first equality is explicitly computed in \cite[Theorem~27.3]{harish1976}. Note in the original text the normalization of the measure $\mass{g}$ and $\mass{\nu}$ is different from ours by a constant of $2^{\frac{n-\dim \Lie{a}_0}{2}}$ and $(2\pi)^m$  respectively. Moreover the constant $c_M$ and $c_G$ arising in the text are computed in \cite[Theorem~37.1]{harish1975}. Summing up all these we have the first equality.
\par The second equality comes from \cite[Lemma~37.4]{harish1975}, which says for all connected Lie group $K$ with maximal torus $T$ we have:
\begin{equation}
\prod_{\alpha\in \Delta^+}\brac{\alpha, \rho_{\Lie{k}}}=(2\pi)^{\frac{\dim K/T}{2}}\frac{\vol(T)}{\vol(K)}
\end{equation}
where the volume are those induced by $\brac{-,-}$ on the left-hand side and $\Delta^+$ is a positive root system associated with $(\Lie{k,t})$. Now in particular this holds for the compact dual $G^d$, which has maximal torus $H^d$ with Lie algebra of $\Lie{t}\oplus i\Lie{a}$, dual to $\Lie{h}=\Lie{t}\oplus \Lie{a}$ of $\Lie{g}$. Now apply this to the middle term, we have:
\begin{equation}
\frac{\prod_{\alpha\in \Delta^+}\brac{\alpha, \rho_{G}}}{\prod_{\alpha\in \Delta^+_k}\brac{\alpha, \rho_{K}}}=(2\pi)^{\frac{n-m}{2}}\frac{\vol(K)\vol(H^d)}{\vol(G^d)\vol(T)}=(2\pi)^{\frac{n-m}{2}}\frac{\vol(S^d_A)}{\vol(X^d)}
\end{equation}
where the last equality comes from the fact that the map $H^d/T\to X^d: hT\mapsto hK$ is an isometric embedding with the image $S^d_A$.
\end{proof}
The expression of the constant term now gives an analytic proof of Hirzebruch Proportionality:
\begin{Cor}[Hirzebruch Proportionality]\label{hirzebruch}
	If the fundamental rank of $X=G/K$ is nontrivial, then $\chi(\Gamma\backslash X)=0$. Otherwise, one has:
	\begin{equation}
	\frac{\chi(\Gamma\backslash X)}{\vol(\Gamma\backslash X)}=(-1)^{\dim(X)/2}\frac{\chi(X^d)}{\vol(X^d)}
	\end{equation}
\end{Cor}
\begin{proof}
	The first part on Euler characteristic follows directly from Part I of \Cref{olbrich1.1}. So it suffices to consider the case when $\hat{G}_d\neq \emptyset$. From \Cref{olbrich5.1} and \textbf{Weyl Dimension formula} implies the plancherel density at $\pi\in \hat{G}_d$ is:
	\begin{equation}
	p_\pi=\dim(\tau)/\vol(X^d)
	\end{equation}
	where $\tau$ is an finite-dimensional representation of $G$ such that $\chi_{\tau}=\chi_\pi$. From \Cref{knapp8.18} the infinitesimal character of $\pi\in \hat{G}_d$ are $W(\Lie{t}^\complex:\Lie{g}^\complex)$-invariant, so for fixed infinitesimal character there are $\frac{|W(\Lie{t}^\complex:\Lie{g}^\complex)|}{|W(\Lie{t}^\complex:\Lie{k}^\complex)|}$-many equivalence classes of discrete series representations. By \cite[Page~175]{bott1965}, we have $\frac{|W(\Lie{t}^\complex:\Lie{g}^\complex)|}{|W(\Lie{t}^\complex:\Lie{k}^\complex)|}=\chi(X^d)$, and now arguing in the backward direction of Part I, by \ref{olbrich(9)} and the above discussion, one has:
	\begin{equation}
	(-1)^{(\dim X)/2}\chi(\Gamma\backslash X)=b^{(2)}_{\frac{\dim X}{2}}(X)=\vol(\Gamma\backslash X)\sum_{\substack{\pi\in \hat{G}_d\\ \chi_{\pi}=\chi_{0}=\delta_G}}p_\pi=\vol(\Gamma\backslash X)\frac{\chi(X^d)}{\vol(X^d)}
	\end{equation}
	and we have the desired equality.
\end{proof}
We could of course compute $k_X(t)$ using \ref{olbrich15} for general $X$ with fundamental rank $1$, but this will be extremely complicated. In view of \Cref{3.93(4)} we can break $X$ down to irreducible cases and the perform the computation for these spaces. More explicitly, we take:
\begin{equation*}
\Gamma\backslash X=(\Gamma_1\times \Gamma_0)\backslash (X_1\times X_0)=(\Gamma_1\backslash G_1/K_1)\times (\Gamma_0\backslash G_0/K_0)
\end{equation*}
where $\Gamma=\Gamma_1\times \Gamma_0$, and $X_1$ and $X_0$ are respectively the universal cover of their quotients by $\Gamma_1$ and $\Gamma_0$, with $m(X_0)=0$ and $m(X_1)=1$. Now the product formula gives:
\begin{equation}
\rho^{(2)}(X;\vna(\Gamma))=\chi(\Gamma_0\backslash X_0)\rho^{(2)}(X_1, \vna(\Gamma_1))
\end{equation}
Apply \nameref{hirzebruch} we have:
\begin{equation*}
\rho^{(2)}(X)=\frac{\chi(\Gamma_0\backslash X_0)}{\vol(\Gamma_0\backslash X_0)}\rho^{(2)}(X_1)=\frac{(-1)^{(\dim X_0)/2}\chi(X^d_0)}{\vol(X^d_0)}\rho^{(2)}(X_1)
\end{equation*}
In particular \Cref{olbrich1.1}.IV.(a) is proved.
\par Now to prove \Cref{olbrich1.1}.III it suffices to evaluate \ref{olbrich15} for the simple Lie groups $X_1$ with $m(X_1)=1$. By classification of simple Lie groups we have 
\begin{equation*}
X_1=SL(3, \real)/SO(3)\text{ or $X_1=SO(p,q)^0/(SO(p)\times SO(q))$ for $p\leq q$ odd}
\end{equation*}
We shall handle them in parallel.
\par Before embarking on proving the main theorem, we are first to compute some constants occurring in \ref{olbrich15}. First consider the size of $\hat{M}_d$. For fundamental parabolic subgroup recall $\hat{M}_d$ are all induced from $(\hat{M_0})_d$. Since $M_0$ is linear connected reductive group with fundamental rank $0$, we can apply \Cref{discrete series} to parametrize $\hat{M}_d$ by $i\Lie{t}^*$ modulo the action of 
\begin{equation}
W_{K_M}:=N_{K_M}(\Lie{t})/T=N_{K_M}(\Lie{t})/Z_{K_M}(\Lie{t})=W(T:M) \footnote{Note here we use analytic Weyl group rather than the algebraic Weyl group of the original theorem. This is a byproduct of the induced representation when passing from $M_0$ to $M$. For details, see \cite[Proposition~12.32]{knapp2016}}
\end{equation}
Moreover, we see from \Cref{knapp8.18} the infinitesimal character are $W(\Lie{t}^\complex:\Lie{m}^\complex)$-invariant, so altogether we have for fixed infinitesimal character there are $\frac{|W(\Lie{t}^\complex:\Lie{m}^\complex)|}{|W_{K_M}|}$-many equivalence classes of discrete series representations. This term is grouped with $W_A$ in \Cref{olbrich5.1} by the following lemma:
\begin{Lemma}
	Let $G$ be a simple Lie group with fundamental rank $1$, with $P=MAN$ the fundamental parabolic group. Let $W_{\Lie{k_m}}$ be the algebraic Weyl group associated to $\Delta(\Lie{t}^\complex: (\Lie{k_m}^\complex))$. Then:
	\begin{equation}
	\frac{|W_{K_M}|}{|W_{\Lie{k_m}}|}\cdot |W_A|=2
	\end{equation}
\end{Lemma}
\begin{proof}
From \cite[Propsition~7.19(b)]{knapp2013} we see $|M/M_0|=|K_M/K_M^0|$. For $k\in K_M$, we have $\Ad_k\Lie{t}$ is a maximal torus of $\Lie{k_m}$ and thus there exists a $k^0\in K_M^0$ such that $\Ad_k\Lie{t}=\Ad_{k^0}\Lie{t}$. Hence for $K_M/K_M^0$ we can find representative $N_{K_M}(\Lie{t})$ and there are canonical isomorphism:
\begin{equation}
K_M/K_M^0\cong N_{K_M}(\Lie{t})/N_{K^0_M}(\Lie{t})\cong W_{K_M}/W_{\Lie{k_m}}
\end{equation} 
Summing up, we have $|W_{K_M}/W_{\Lie{k_m}}|$ equals to the number of connected components of $M$. Choose $\Lie{a_p}$ a maximal abelian subspace of $\Lie{p}$ containing $\Lie{a}$, with corresponding restricted roots $\Sigma=\Delta(\Lie{a}:\Lie{g})$, with respective Weyl group $W(\Sigma)$. Now from \cite[Theorem~5.17]{knapp2016} we have:
\begin{equation}
W(A_\Lie{p}:G):=N_K(\Lie{a_p})/Z_K(\Lie{a_p})=W(\Sigma)
\end{equation}
Now following the arguments of \cite[Proposition~7.85]{knapp2013} we see each element of $W_A$ have a representative in $W(A:G):=N_K(\Lie{a})/Z_K(\Lie{a})$ and can be extended to a member of $W(A_\Lie{p}:G)$. Now an case-by-case study yields that the only case in which $W(\Sigma)$ contains an nontrivial element that fixes $\Lie{a}$ is when $G=SO(p,1)^0$, in which case $M$ is connected. In all other cases, $M$ has exactly two components. This proves the lemma.
\end{proof}
This together with \Cref{olbrich5.1}, yields an more integrated expression of the constant:
\begin{equation}\label{constant term}
\frac{|W(\Lie{t}^\complex:\Lie{m}^\complex)|}{|W_{K_M}|}\cdot C=\frac{|W(\Lie{t}^\complex:\Lie{m}^\complex)|}{|W_\Lie{k_m}|}\cdot\frac{\vol(S^d_A)}{(2\pi)^m}\frac{1}{\vol(X^d)}
\end{equation}
\begin{proof}[\textbf{Proof of \Cref{olbrich1.1}.IV.(b) (c)}]
In the sequel we evaluate \ref{olbrich15} in the following steps:
\par First we evaluate the dimension term. For respective $X_1$ one has:
	\begin{equation}
	M\cong\begin{cases}
	SO(p-1, q-1) \quad &\text{if $G_1=SO(p,q)^0$}\\
	\{\begin{pmatrix}
	A &0\\0 &1
	\end{pmatrix}\mid \norm{\det A}=1\} \quad &\text{if $G_1=SL(3,\real)$}
	\end{cases}\quad \Lie{n}\cong \begin{cases}
	\real^{p+q-2}\quad &\text{if $G_1=SO(p,q)^0$}\\
	\real^2 \quad &\text{if $G_1=SL(3,\real)$}
	\end{cases}
	\end{equation}
	Moreover, the $M_0$-representations on $\Lambda^l\Lie{n}^*\otimes \complex$ is irreducible for all $l$ except when $G=SO(p,q)^0$ and $l=\frac{1}{2}\dim \Lie{n}$ takes the middle degree. This can be easily derived from Weyl dimension formula (c.f.\cite[Theorem~4.48ff]{knapp2016}) In the latter case we have two irreducible components $\Lambda^+\Lie{n}\oplus \Lambda^-{\Lie{n}}$. 
	\par Now we mimic the proof of part I, by replacing the pair $(G,K)$ by $(M_0, K^0_M)$. Then we see similarly, by arguing backwards on \Cref{borel2.5.3} we have in both cases of $X_1$:
	\begin{equation}
	\dim [V^\lambda_0\otimes \Lambda^p\Lie{p}_\Lie{m}^*\otimes \Lambda^l\Lie{n}^*]^{K_M^0}=\begin{cases} 1 \quad &\text{if $2p=\dim \Lie{m}-\dim \Lie{k}_\Lie{m}=\dim \Lie{p}_\Lie{m}$ and $\chi_{\lambda}=\chi_{\Lambda^l\Lie{n}^*}$}\\ 0 \quad &\text{if otherwise} \end{cases} 
	\end{equation}
	with $\chi$ denotes the infinitesimal character of respective representations. Then we see the dimension term in \ref{olbrich15} is in fact the Euler characteristic of the chain complex $C^*(\Lie{m}, K^0_M, E)$, which is only non-vanishing in the middle dimension. Consequently, we have:
	\begin{equation}\label{olbrich b4 17}
	\dim [V^\lambda_0\otimes (\Lambda^{\even}\Lie{p}_\Lie{m}^*-\Lambda^\odd\Lie{p}_\Lie{m}^*)\otimes \Lambda^l\Lie{n}]^{K_M^0}=\begin{cases}
	(-1)^{\frac{1}{2}\dim \Lie{p}_\Lie{m}} \quad \text{if $\chi_{\lambda}=\chi_{\Lambda^l\Lie{n}^*}$}\\
	0 \quad \text{if otherwise}
	\end{cases}
	\end{equation}
	Similarly we have the same result when replacing $\Lambda^l\Lie{n}^*$ by $\Lambda^+\Lie{n}^*$ and $\Lambda^-\Lie{n}^*$.
\par The second step is to compute $Q_{\Lie{a}, \lambda}$. This comprises of three parts, namely $\rho_{\Lie{a}}$, $\pi_{\lambda}(\Omega_M)$ and $p_\lambda(i\nu)$. 
\par To compute $\rho_{\Lie{a}}$ and $\pi_{\lambda}(\Omega_M)$, we first coin the root systems and weights of pertinent representations. We first calculate all of these for $G=SO(p,q)$. In such case $\Lie{m}=\Lie{so}(p-1,q-1)$. In the sequel we realize the maximally compact Cartan subalgebra $\Lie{h}=\Lie{t}\oplus \Lie{a}$ as follows. Denote $E_{i,j}$ to be the elementary matrix with 1 at $(i,j)$-entry and zero elsewhere. We take the following basis of $\Lie{h}$:
\begin{equation}
\begin{split}
H_i=\begin{cases}
E_{p,p+1}+E_{p+1,q} \quad &\text{for }i=1\\
\sqrt{-1}(E_{2i-3,2i-2}-E_{2i-2,2i-3})\quad &\text{for }2\leq i\leq \frac{p+1}{2}\\
\sqrt{-1}(E_{2i-1, 2i}-E_{2i,2i-1})\quad &\text{for }\frac{p+1}{2}<i\leq \frac{p+q}{2}
\end{cases}
\end{split}
\end{equation}
From now on take $n:=\frac{p+q}{2}$, Then we take $\Lie{a}, \Lie{t}$ respectively to be:
\begin{equation*}
\Lie{a}:=\real H_1 \qquad \Lie{t}:=\bigoplus_{i=2}^{n}\sqrt{-1}H_i
\end{equation*}
Then define the root systems $\{e_i\}$ of $(\Lie{h}^\complex)^*$ with respect to $H_i$, then:
\begin{equation*}
\begin{split}
\Delta(\Lie{h}^\complex:\Lie{g}^\complex)=\{\pm e_i\pm e_j, 1\leq i<j\leq n\}\\
\Delta(\Lie{t}^\complex:\Lie{m}^\complex)=\{\pm e_i\pm e_j, 2\leq i<j\leq n\}
\end{split}
\end{equation*}
Moreover, we fix positive systems of roots by:
\begin{equation*}
\begin{split}
\Delta^+(\Lie{h}^\complex:\Lie{g}^\complex)&:=\{e_i+e_j\mid i\neq j\}\sqcup \{e_i-e_j:i<j \}\\
\Delta^+(\Lie{t}^\complex:\Lie{m}^\complex)&:=\{e_i+e_j\mid i\neq j, i,j\geq 2\}\sqcup \{e_i-e_j\mid 2\leq i<j\}
\end{split}
\end{equation*}
So readily we see the half sum of restricted roots are respectively:
\begin{equation}
\begin{split}
\rho_{\Lie{a}}&=\rho_{\Lie{g}}|_\Lie{a}=\frac{1}{2}\sum_{i=2}^ne_1+e_i+e_1-e_i=(n-1)e_1=\frac{1}{2}\dim\Lie{n}\cdot e_1\\
\rho_M&=\frac{1}{2}((n-2)e_2+ \cdots+ (n-2)e_n)+((n-2)e_2+ (n-4)e_3+\cdots+(n-2(n-1))e_n)\\
&=(n-2)e_2+(n-3)e_3 +\cdots +0e_n
\end{split}
\end{equation}
Next we compute $\pi_{\lambda}(\Omega_M)$: By \ref{olbrich b4 17}, it suffices to consider all those $\pi_\lambda$ such that $\chi_{\lambda}=\chi_{\Lambda^*\Lie{n}^*}$ for $*=l, +, -$.  On the other hand since $\Lambda^*\Lie{n}^*$ are all finite-dimensional representations, we suffices to use  \cite[Proposition~5.28(b)]{knapp2013} to see:
\begin{equation}
\chi_{\Lambda^*\Lie{n}^*}(\Omega_M)=(\norm{\Lambda_*+\rho_M}^2-\norm{\rho_M}^2)
\end{equation}
 where $\Lambda_*$ is the highest weight of $\Lambda^*\Lie{n}^*$ for $*=l, +,-$. Consequently,  we suffices to evaluate the highest weight for each representations. First we need to compute the highest weight of representation from that of their compact dual $M_d=SO(p+q-2)$ and then take the effect of Cartan involution into consideration.
 \par For the natural representation of $SO(p+q-2)$ on $\Lambda^l\real^{p+q-2}$, we extend it to a representation on $\Lambda^l\complex^{p+q-2}$. In such case we have the weights of the representations are of the form:
\begin{equation*}
\Big\{\pm e_{j_1}\pm\cdots \pm e_{j_r} \mid 2\leq j_1<\cdots <j_r \qquad \text{with }r
\leq l \text{ if }l\leq n  \quad \text{or} \quad r\leq 2n-l \text{ if }l\leq n
\Big\}
\end{equation*}
Consequently we have the highest weight respectively being:
\begin{equation}
e_2+\cdots e_N \text{ if } N\leq n \quad \text{or} \quad e_2+\cdots +e_{2n-N} \text{ if }N>n
\end{equation}
Next note $e_2, \cdots, e_{n-1}$ are compact roots, and $e_n$ is noncompact root, hence the Cartan involution acts on representations via:
\begin{equation}
(\lambda_2, \cdots, \lambda_{n})\mapsto (\lambda_2, \cdots, \lambda_{n-1}, -\lambda_n)
\end{equation}
Since the last entry of $\rho_M$ is zero, we see it does not affect the norm. Altogether we have for $l< n-1$:
\begin{equation}\label{infi char of lambda_l}
\begin{split}
\chi_{\Lambda^l\Lie{n}^*}(\Omega_M)&=(\norm{\Lambda_l+\delta_M}^2-\norm{\delta_M}^2)\\
&=\norm{(n-1, \cdots, n-l, n-l-2, \cdots, 0)}^2-\norm{(n-2, \cdots, 0)}^2\\
&=(2n-3+2n-5+\cdots +2n-2l-1)\cdot \norm{e_1}^2\\
&=(2n-l-2)l\cdot \norm{e_1}^2=(\dim\Lie{n}-l)l\cdot \norm{e_1}^2
\end{split}
\end{equation}
since for all $i$, $\norm{e_i}^2=\norm{e_1}^2$ the normalization factor. Also from the above result we easily see for $l>n-1$, we have the same $\chi_{\Lambda^l\Lie{n}^*}(\Omega_M)$ by symmetry. For $*=\pm$ case, we similarly get:
\begin{equation}
\chi_{\Lambda^\pm\Lie{n}^*}(\Omega_M)=(n(n-2)+(\pm 1)^2)\cdot \norm{e_1}^2=(n-1)^2\cdot \norm{e_1}^2=(\frac{1}{2}\dim\Lie{n})^2\norm{e_1}^2
\end{equation}
Summarizing, we have for all cases of $SO(p,q)^0$, we have:
\begin{equation}
\norm{\rho_{\Lie{a}}}^2-\chi_{\lambda^*\Lie{n}^*}(\Omega_M)=((n-1)^2-l(2n-l-2))\norm{\alpha_0}^2=\frac{\dim\Lie{n}}{2}\norm{\alpha_0}^2
\end{equation}
For $G=SL(3,\real)$ we follow the same procedure. Again we fix the maximally compact Cartan subalgebra $\Lie{h}=\Lie{t}\oplus \Lie{a}$ as follows:
\begin{equation}
\Lie{a}:=\real (H_1:=\begin{pmatrix}
1 &0 &0\\ 0 &1 &0\\ 0 &0 &-2
\end{pmatrix}) \qquad \Lie{t}:=\real(H_2:=\begin{pmatrix}
0 &1&0\\ -1&0 &0\\0 &0 &0
\end{pmatrix} )
\end{equation}
and $\Lie{m}\cong \Lie{sl}_2(\real)$ with the matrix embedded as an upper left block. Next fix $f_1\in \Lie{a}^*$ and $f_2\in i\Lie{t}^*$ by $f_1(H_1)=3$ for $f_2(H_2)=i$, and fix $f_1$ as the positive restricted root of $(\Lie{g}:\Lie{a})$. Consequently one can define the positive roots by:
\begin{equation}
\Delta^+(\Lie{h}^\complex:\Lie{g}^\complex)=\{2f_2, f_1+f_2, f_1-f_2\} \qquad \Delta^+(\Lie{t}^\complex:\Lie{m}^\complex)=\{2f_2\}
\end{equation}
We see $\norm{f_2}^2=\frac{1}{3}\norm{f_1}^2$ and $\brac{f_1, f_2}=0$. Now the finite-dimensional representations of $M^0$ are parametrized by their highest weights $\{kf_2\mid k\in \nat\}$ and $\rho_{\Lie{a}}=f_1$,  $\rho_M=f_2$ and $\rho_{G}=f_1+f_2$.
\par Moreover the only nontrivial representation of $M_0$ is on $\Lambda^1\complex^2$, which corresponds to the highest weight $f_2$. Hence under the bases $(\lambda_1, \lambda_2)=\lambda_1f_1+\lambda_2f_2$, we have:
\begin{equation}
\chi_{\Lambda^1\Lie{n}^*}(\Omega_M)=\norm{(0,1)+(0,1)}^2-\norm{(0,1)}^2=3\norm{f_2}^2=\norm{f_1}^2
\end{equation}
The third step is to compute the constant term. In \ref{constant term}, we have $m=1$, and $S^d_A$ is a circle of radius inverse of the normalization factor we choose. Consequently $\frac{\vol(S^d_A)}{2\pi}=\frac{1}{\norm{\alpha_0}}$, where $\alpha_0\in \Lie{a}^*$ the unique positive root, which is $e_1$ in the case of $SO(p,q)^0$ and $f_1$ in the case of $SL(3,\real)$.
\par Now we bring \ref{constant term}, \ref{olbrich b4 17}, and the formula of $p_\lambda(i\nu)$ in \nameref{plancherel} altogether into \ref{olbrich15},
\begin{equation}\label{olbrich234}
k_X(t)=(-1)^{\frac{\dim\Lie{p_m}+\dim\Lie{n}}{2}}\frac{1}{2\norm{\alpha_0}\vol(X^d)}\cdot\frac{|W(\Lie{t}^\complex:\Lie{m}^\complex)|}{|W_\Lie{k_m}|}\sum^{\dim \Lie{n}}_{l=0}(-1)^{l+1}k_l(t)
\end{equation}
with $k_l$ explicitly to be:
\begin{equation}\label{olbrich pl}
k_l(t)=\begin{cases}
\int_{\Lie{a}^*}e^{-t(\norm{\nu}^2+(\frac{\dim\Lie{n}}{2}-l)^2\norm{\alpha_0}^2)}\prod_{\alpha\in\Delta^+}\frac{\brac{\lambda_l+i\nu, \alpha}}{\brac{\rho_{\Lie{g}},\alpha}}\mass{\nu}\quad &\text{if $G=SO(p,q)^0, l\neq \frac{\dim\Lie{n}}{2}$}\\
\int_{\Lie{a}^*}e^{-t\norm{\nu}^2}\cdot(\prod_{\alpha\in\Delta^+}\frac{\brac{\lambda_{+}+i\nu, \alpha}}{\brac{\rho_{\Lie{g}},\alpha}}+\prod_{\alpha\in\Delta^+}\frac{\brac{\lambda_{-}+i\nu, \alpha}}{\brac{\rho_{\Lie{g}},\alpha}})\mass{\nu}\quad &\text{if $G=SO(p,q)^0, l=\frac{\dim\Lie{n}}{2}$}\\
\int_{\Lie{a}^*}e^{-t(\norm{\nu}^2+(1-l)^2\norm{\alpha_0}^2)}\prod_{\alpha\in\Delta^+}\frac{\brac{\lambda_{l}+i\nu, \alpha}}{\brac{\rho_{\Lie{g}},\alpha}}\mass{\nu}\quad &\text{if $G=SL(3,\real), l=0,1,2$}
\end{cases}
\end{equation} 
To complete our computation, let $k_{P,c}(t)=\int_{-\infty}^{\infty}e^{-t(y^2+c^2)}P(iy)\mass{y}$. We note when $P$ is an even polynomial, we can apply the method of \cite[Lemma~2,3]{fried1986}. For the reader's convenience, we record it here:
\begin{Lemma}\label{fried3}
	Let $P$ be an even polynomial of dimension $2n$. Then the Mellin transform of $F(t):=\int_{\real}^{}e^{-ty^2}P(iy)\mass{y}$ with parameter $s\in \complex$, for $c>0$,
	\begin{equation}
	\mathcal{M}_s(e^{-tc^2}F(t)):=\int_{0}^{\infty}t^{s-1}e^{-tc^2}F(t)\mass{t}
	\end{equation}
	exists for $\Re(s)>2n+1$, which admits a meromorphic continuation with respect to $s$ to $\complex$ which is regular at $0$, with 
	\begin{equation}
	\mathcal{M}_0(e^{-tc^2}F(t))=-2\pi\int_0^cP(y)\mass{y}
	\end{equation}
\end{Lemma}
\begin{proof}[Proof of the Lemma]
By linearity we suffices to assume $P=y^{2a}$. Then:
\begin{equation*}
F(t)=(-\diff{}{t})^a\int_{\real}^{}e^{-ty^2}\mass{y}=(-\diff{}{t})^a\sqrt{\pi}t^{-1/2}=b_at^{-a-(1/2)}
\end{equation*}
where $b_a=\sqrt{\pi}\frac{1}{2}\cdot\frac{3}{2}\cdots \frac{2a-1}{2}$. i.e., $F(t)t^{\frac{1}{2}}$ is a polynomial $Q(t^{-1})$ of degree $n$. Next:
\begin{equation}
\mathcal{M}_s(e^{-tc^2}F(t))=b_a\int_{0}^{\infty}t^{s-a-\frac{1}{2}-1}e^{-tc^2}\mass{t}=b_a\Gamma(s-a-\frac{1}{2})c^{-2(s-a-\frac{1}{2})}
\end{equation}
Now for $c>0$ and for $\Re s>{a+\frac{1}{2}}$ the meromorphic extension of Gamma function gives the desired extension, and since when $s=0$, we have $\Gamma(-a-\frac{1}{2})$ not on negative integral points, the function is hence regular at $s=0$, with the value be:
\begin{equation*}
\begin{split}
\mathcal{M}_0(e^{-tc^2}F(t))&=b_a\Gamma(-a-\frac{1}{2})c^{2a+1}=(-1)^{a+1}\frac{2}{2a+2}\sqrt{\pi}\cdot\frac{-1}{2}\cdot\frac{-3}{2}\cdots \frac{-2a-1}{2}\cdot\Gamma(\frac{-2a-1}{2})c^{2a+1}\\
&=\sqrt{\pi}\cdot\sqrt{\pi}\cdot(-1)^{a+1}\frac{2\pi}{2a+1}c^{2a+1}=-2\pi\int_0^c(iy)^{2a}\mass{y}
\end{split}
\end{equation*}
by using $z\cdot\Gamma(z)=\Gamma(z+1)$ and $\Gamma(\frac{1}{2})=\sqrt{\pi}$. Now the general result follow from linearity.
\end{proof}
Use this lemma we then see we can save the problem of split the integral into half as in \ref{3.131}, and it makes sense to speak of $\frac{d}{ds}(\frac{1}{\Gamma(s)}\mathcal{M}_s(e^{-tc^2}F))|_{s=0}$. Moreover, since $1/\Gamma(0)=0$ and $\frac{d}{ds}\frac{1}{\Gamma(s)}|_{s=0}=1$, we have \begin{equation}
\frac{d}{ds}(\frac{1}{\Gamma(s)}\mathcal{M}_s(e^{-tc^2}F))\Big|_{s=0}=-2\pi\int_0^cP(y)\mass{y}
\end{equation}
Now it is time to evaluate $P$ explicitly, which in our case is the product term in \ref{olbrich pl}. We shall handle them case by case. In the first case, recall \ref{infi char of lambda_l} in particular gives:
\begin{equation}
\lambda_l=\sum_{i-2}^{l+1}(n-i+1)e_i+\sum_{i=l+2}^{n}(n-i)e_i
\end{equation}
for $n=\frac{\dim\Lie{n}}{2}+1$ Note for all $i$, $\norm{e_i}=\norm{e_1}$. Hence the product term in such case is
\begin{equation}
\begin{split}
P_l(i\nu\cdot e_1)&=\prod_{j=1}^{n-1}\prod_{q=j+1}^{n}\frac{\brac{\sqrt{-1}\nu \cdot e_1+\sum_{i=2}^{l+1}(n-i+1)e_i+\sum_{i=l+2}^{n}(n-i)e_i, e_j+e_q}}{\brac{\sum_{k=1}^{n}(n-k)e_k, e_j+e_q}}\\
&\cdot\prod_{j=1}^{n-1}\prod_{q=j+1}^{n}\frac{\brac{\sqrt{-1}\nu\cdot e_1+\sum_{i=2}^{l+1}(n-i+1)e_i+\sum_{i=l+2}^{n}(n-i)e_i, e_j-e_q}}{\brac{\sum_{k=1}^{n}(n-k)e_k, e_j-e_q}}\\
&=\frac{\prod_{\substack{1\leq i\leq n\\ i\neq l+1}}(-\nu^2-(n-i)^2)\prod_{\substack{1\leq j<i\leq n\\ i,j\neq l+1}}((n-j)^2-(n-i)^2)}{\prod_{1\leq j<i\leq n}((n-j)^2-(n-i)^2)}\\
&=(-1)^l\cdot\prod_{\substack{1\leq j\leq n\\ j\neq l+1}}\frac{-\nu^2-(n-j)^2}{(n-l-1)^2-(n-j)^2}
\end{split}
\end{equation}
Next for $l=\frac{\dim\Lie{n}}{2}$-case, we note $P_+(i\nu)=P_-(i\nu)$, hence it suffices to compute one of them. Now:
\begin{equation}
\begin{split}
P_+(i\nu)=\frac{\prod_{i=1}^{n-1}(-\nu^2-(n-i)^2)\prod_{1\leq j<i\leq n-1}((n-j)^2-(n-i)^2)}{\prod_{1\leq j<i\leq n}((n-j)^2-(n-i)^2)}=\prod_{i=1}^{n-1}\frac{-\nu^2-(n-i)^2}{(n-i)^2}
\end{split}
\end{equation}
Note if $c=0$, \Cref{fried3} in particular says $F$ behaves like polynomial. Hence it is standard analysis to prove this part contributes nothing to the $L^2$-torsion.
\par Now apply \Cref{fried3} to \ref{olbrich234}. There are a few more terms we can further simplify. First note $p_l=p_{\frac{\dim\Lie{n}}{2}-l}$.  Moreover, when $X=SO(p,q)^0/(SO(p)\times SO(q))$, denote $n:=\frac{\dim\Lie{n}}{2}+1=\frac{p+q}{2}$, $\Lie{m}^\complex\cong \Lie{so}(2n, \complex)$, in which case $\Lie{k}_\Lie{m}^\complex\cong \Lie{so}(p-1,\complex)\oplus \Lie{so}(q-1, \complex)$, and consequently $|W(\Lie{t}^\complex:\Lie{m}^\complex)|=(n-1)!2^{n}$, and $|W_\Lie{k_m}|=\frac{p-1}{2}!\frac{q-1}{2}!2^{n-1}$, and $\frac{|W(\Lie{t}^\complex:\Lie{m}^\complex)|}{|W_\Lie{k_m}|}=2{\frac{p+q-2}{2} \choose \frac{p-1}{2}}$. Taking all of these together, we have:
\begin{equation}\label{analytic torsion of so(p,q)}
\begin{split}
\rho^{(2)}_{\mathrm{an}}(X)
&=(-1)^{\frac{pq-1}{2}}\frac{2\pi{\frac{p+q-2}{2} \choose \frac{p-1}{2}}}{\vol(X^d)}\cdot\sum^{n-1}_{l=0}(-1)^{2l+2}\int_{0}^{n-1-l}\Big(\prod_{\substack{1\leq j\leq n\\ j\neq l+1}}\frac{\nu^2-(n-j)^2}{(n-l-1)^2-(n-j)^2}\Big)\mass{\nu}\\
\end{split}
\end{equation}
Now to prove $\rho^{(2)}$ is non-vanishing in such case, one suffices to prove each summand in the formula above have the same parity. Here we follow the strategy of \cite[5.9.1]{bergeron2013}. Denote for $0\leq l\leq n-1$, that:
\begin{equation*}
\Pi_l(\nu):=\prod_{\substack{1\leq j\leq n\\ j\neq l+1}}\frac{\nu^2-(n-j)^2}{(n-l-1)^2-(n-j)^2} \qquad Q_l(\nu):=\sum_{j=0}^l\Pi_j(\nu)
\end{equation*} 
Note $\Pi_k(\pm(n-1-j))=\delta_{jk}$ and use Lagrangian polynomial we see it is the unique even polynomial of degree $\leq 2n-2$ such that:
\begin{equation*}
Q_k(\pm (n-1-j))=\begin{cases}
1 \quad \text{if }j\leq k \\ 0 \quad \text{if }n-1\geq j> k
\end{cases}
\end{equation*}
Moreover, we have:
\begin{equation}
\sum_{l=0}^{n-1}\int_{0}^{n-l-1}\Pi_l(t)\mass{t}=\sum_{l=0}^{n-2}\int_{n-l-2}^{n-l-1}Q_l(t)\mass{t}
\end{equation}
Now it suffices to observe each integral on the right is positive. Note $Q_k'$ have no roots between $[n-k-2, n-k-1]$, since $\deg Q_k'\leq 2n-3$, and from definition of $Q_k$ we have a root on each interval $[j, j+1]$ for each integral $1-n\leq j<n-1$, with $j\neq \pm k$. Hence we have $Q_k$ is constant on $[n-k-2, n-k-1]$. Moreover, $Q_n(t)$ is a polynomial of degree $2n-2$ and equals to $1$ at $2n+2$ distinct points. Consequently, $Q_n\equiv 1$. Consequently, we have $\rho_{\mathrm{an}}^{(2)}\neq 0$, with the parity solely determined by $(pq-1)/2$.
\par To conclude the case with $SO(p,q)^0$, we shall not further burden the readers with cumbersome computation. Instead we note \ref{analytic torsion of so(p,q)} is only dependent on $p+q$. Hence it suffices to compute the results fo $SO(p,1)^0/SO(p)\cong \mathbb{H}^p$ the hyperbolic space. This was handled with great detail in \cite{schick1998}.
\par Lastly we compute $X=SL(3,\real)/SO(3)$. Recall the computation in $SO(p,q)^0$-case, we see the product part in \ref{olbrich pl} is independent of our normalization of inner product, hence it suffices to assume $\norm{f_1}=3\norm{f_2}=1$, with:
\begin{equation}
\prod_{\alpha\in \Delta^+}^{}\brac{\rho_G, \alpha}=(1+\frac{1}{3})(1-\frac{1}{3})(2\cdot\frac{1}{3})=\frac{16}{27}\\
\end{equation}
Hence we have $P_0(\nu\cdot f_1)=P_2(\nu\cdot f_1)=\frac{9\nu^2-1}{8}$, and $P_1(\nu\cdot f_1)=\frac{9\nu^2-4}{4}$. Argue as $SO(p,q)^0$-case, we see again the middle-order term does not contribute to the $L^2$-torsion, and $|W(\Lie{t}^\complex:\Lie{m}^\complex)|=|W_\Lie{k_m}|=1$, and from \ref{olbrich234} we have:
\begin{equation}
\rho^{(2)}_{\mathrm{an}}(SL(3,\real)/SO(3))=(-1)^{2+1+1}\cdot \frac{2\pi}{\vol(X^d)}\int^1_0\frac{9\nu^2-1}{8}\mass{\nu}=\frac{\pi}{2\cdot \vol(X^d)}
\end{equation}
To this point is the proof of \Cref{olbrich1.1}.IV.(b) \& (c) finished.
\end{proof}
\begin{proof}[\textbf{Proof of \Cref{olbrich1.1}.III}] This part now readily follows from \Cref{olbrich1.1}.IV since we have proved the non-vanishing result of all cases of $X_1$. Hence the proof of \Cref{olbrich1.1} is finished.
\end{proof}

\appendix

 \bibliography{/Users/hanzhicheng/Desktop/lib/Masterarbeit.bib}
\end{document}